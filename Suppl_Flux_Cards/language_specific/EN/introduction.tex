
\newtitle{Introduction}
\label{introduction}% do not translate this label

Flux Cards is a supplement intended to add depth and variety to games of \nameofthegame{}, while providing additional opportunities for narrative play. As such, it may not be ideally suited for tournaments or other type of competitive play. We recommend that you discuss with your opponent before the game and reach an agreement whether or not to use the contents of this book.

\newsubtitle{What are these Flux Cards?}

The introduction of Flux Cards and the Veil Token mechanic to define the Magic Phase was a key change to the 2.0 version of \nameofthegame{} rules. As has been noted by many members of our community, the cards open a lot of design space and options for modifying the game or exploring different magic phases, particularly vis-a-vis games of different sizes or with different numbers of players. A group of staff members in the project noticed this and came together to design some alternative sets of Flux Cards, to give players some additional tools and options for playing the game the way they want, or to just spice things up with something different for a bit of fun.

Five alternative sets of flux cards have been created and presented here: (A) Grand, (B) Warband, (C) Free For All, (D) Dark Gods and (E) Sorcerous Storm. These sets can be picked up and used with minimal modifications to the core rules.

\newsubtitle{When and why should I use them?}

The key thing to remember is that these sets are an auxiliary product created by \theninthage{} project to improve and broaden your (the players') gaming experience, so the bottom line is to use them whenever and where-ever you and your opponent agree to use them. You should also feel free to modify them to suit the needs of your own gaming group. Maybe if you come up with some great modifications of your own, consider sending them in to The \nth{9} Scroll.

Of course, there were some design goals and ideas that we had in mind during the development of these sets. We will now briefly describe our design goals and intended usage for each set, and then explain what you need to do to use each set in your games.

\newsubtitle{Set (A): Grand (page \pageref{grand})}

Set (A) has the primary goal of facilitating large games, where players can invest more points into spell casters than a standard 4500 points game, without hitting diminishing returns. As such, expect games using this Set of Flux Cards to have more epic Magic Phases that have a bigger effect on the game. Equally, if you want to simulate a game in a highly magical region of the world, or just like to use lots of magic, feel free to use these cards in standard 4500 points games too!

\newsubtitle{Set (B): Warband (page \pageref{warband})}

Set (B) is the opposite: it has the primary goal of facilitating small games. This is actually a bit tricky to do without redesigning the entire Magic Phase, so this Set is a minimal modification to the Flux Cards to allow the Magic Phase to scale better for games below 4500 points, without changing any rules or the effective game play. This relates to one of the two uses of Set (C): as an alternative set for small games, that doesn't quite play in the standard way.

\newsubtitle{Set (C): Free For All (page \pageref{ffa})}

Set (C) contains cards that give fewer \newrule{Magic Dice and} no Dispel Dice, so results in quite a different experience! One use of this is a more radical alternative for small games, as stated above. However, the main use of this Set is as a multi player \enquote{free-for-all} Set of cards, which allows players to simply and easily incorporate magic into such games, without any complicated negotiation regarding who gets the Dispel Dice during a certain Player's Turn\dots{}

\newsubtitle{Set (D): Dark Gods (page \pageref{darkgods})}

The final two Sets, (D) and (E) are not aimed at different game sizes or player numbers. Instead they are designed to offer a slightly wackier and crazier experience, and both relate to the background of \theninthage{} world.

The first of these, (D), is based around the concept of the 7 sins and Father Chaos himself. Each Flux Card is associated with one of these 8, and when the Flux Card is revealed, the player may choose to yield to the sin. If they do so, then they gain additional magical power, but at the cost of their army succumbing to the corresponding sin, and acting appropriately. If the Father Chaos card is chosen, then the player has the option to yield to two sins, for a truly powerful magic phase; but can any general control an army that has yielded to two sins ?

\textbf{Rules ---} After drawing a Flux Card, the Active Player may choose to yield to the sin on that Flux Card. If
they do, they gain 2 additional Magic Dice, except for Flux Card 2, and the effects of yielding to that sin are
applied until the start of the Active Player's next Magic Phase.

\newsubtitle{Set (E): Sorcerous Storm (page \pageref{storm})}

The final Set, (E), plays on the connections between weather and magic in certain localised regions in \theninthage{} world where the veil is weak. As the magical energy in the region fluctuates, different weather patterns can be caused, affecting the Battlefield. As such, each Flux Card has a unique effect on the board based on the weather that that Flux Card represents. Skilled Wizards can harness the weather to weaken enemy troops or bolster their own, adding an entirely new dimension to the player's strategy: this Set of Flux Cards comes with 8 entirely new weather related spells, one associated with each Flux Card, to allow players' Wizards to manipulate the Battlefield itself.

\textbf{Rules ---} Each card contains a weather effect that affects both armies until the next Flux Card is drawn. In
addition, each card contains a Weather Spell that may be cast by any of the Active Player's Wizards.

\newsubtitle{How do I use them?}

With all of these sets, the intent is that both players use a Flux Card deck of the same kind, rather than one player using e.g. the standard set and the other player using the small game set. Of course, if players are writing a specific narrative scenario, then it may be entirely appropriate for each player to use a different type of Flux Card deck, so don't let us stop you.

Sets (A)--(C) require nothing other than substituting the 8 \newrule{Flux} Cards in the Rulebook with the new set and
just doing what it says on the cards. The only subtlety is that Set (C) has no Dispel Dice, so certain Special Equipment might turn out to not be very useful.

With Set (D) the Flux Cards get replaced as above. During the game, after drawing a Flux Card, the Active Player may choose to yield to the sin on that Flux Card. If they do, they gain 2 additional Magic Dice, and the effects of yielding to that sin are applied to their army until the start of the Active Player's next Magic Phase.

With Set (E), the players need to replace the standard Flux Cards with the Weather Flux Cards, and they will also need the set of Weather Spells \newrule{(see next page)}. Each Flux Card contains a weather effect that affects both armies until the next Flux Card is drawn. In addition, each card contains a Weather Spell that may be cast by any of the Active Player's Wizards during the \newrule{Magic P}hase in which the Flux Card was drawn.
