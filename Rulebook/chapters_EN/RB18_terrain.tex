\part{Terrain}
\label{terrain}

\section{Terrain Types}
\label{terrain_types}

\subsection{Dangerous Terrain (X)}
\label{dangerous_terrain}

A model must take a Dangerous Terrain Test if it is in contact with a Terrain Feature which counts as Dangerous Terrain at any point during its March, Charge, Failed Charge, Flee, Pursuit, or Overrun Move. To do this, roll a number of D6 depending on the model's Size and Model Rules:

\begin{center}\begin{tabular}{lcccc}%
\hline
& \textbf{\sizestandard} & \textbf{\sizelarge} & \textbf{\sizegigantic} & \textbf{\chariot}\tabularnewline
Number of D6 rolled & 1 & 2 & 3 & +1 \tabularnewline
\hline
\end{tabular}\end{center}

For each dice equal to or below X rolled (where X is the value stated in brackets), the model suffers a hit with Armour Penetration 10 that wounds automatically.

Note that:
\begin{itemize}[label={-}]
\item Dangerous Terrain Tests are taken as soon as the model is in contact with the relevant Terrain. When it does not matter exactly when a model dies, take all Dangerous Terrain Tests for a unit at once.
\item Hits suffered from Dangerous Terrain Tests are distributed \rewordedrule{onto} the model's Health Pool.
\item A model never takes more than one Dangerous Terrain Test for the same Terrain Feature during a single move, but it might have to take several Dangerous Terrain Tests caused by different pieces of Terrain or abilities.
\end{itemize}

\subsection{Impassable Terrain}
\label{impassable_terrain}

Models cannot move into or through Impassable Terrain. 

\subsection{Opaque Terrain}
\label{opaque_terrain}

Line of Sight cannot be drawn through Opaque Terrain, but can be drawn into and out of it.

\subsection{Covering Terrain}
\label{covering_terrain}

Like models, Terrain Features may contribute to Cover when obscuring a fraction of a unit's Footprint from the enemy shooter's Line of Sight (see \totalref{cover}, and remember that Cover modifiers only apply if more than half of the target unit's Footprint is obscured by Cover).

For the purpose of counting as Cover, Terrain Features may distinguish:

\begin{itemize}[label={-}]
\item Targets obscured \textbf{behind} the Terrain Feature. These units must have more than half of their Footprint off the Terrain Feature.
\item Targets obscured \textbf{inside} the Terrain Feature. These units must have more than half of their Footprint inside the Terrain Feature.
\item Targets obscured \textbf{behind and/or inside} the Terrain Feature: there is no need to determine where more than half of these units' Footprint lies (as long as it is obscured).
\end{itemize}

Models always ignore any Terrain they are inside when they draw Line of Sight.

\newpage
\section{Terrain Features}
\label{terrain_features}

A Terrain Feature is a topographical area on the Battlefield that may be a mixture of Dangerous, Opaque, or Covering Terrain and may possess its own rules.

\subsection{Open Terrain}
\label{open_terrain}

Open Terrain doesn't normally have any effect upon Line of Sight, Cover modifiers, or Movement. All parts of the board that are not covered by any other kind of Terrain are considered to be Open Terrain.

\subsection{Buildings}
\label{buildings}

\begin{tableterrain}
Types & Buildings are \hyperref[opaque_terrain]{Opaque Terrain}. Buildings are \hyperref[impassable_terrain]{Impassable Terrain}, with the exceptions listed below.\newline
Buildings must have a rectangular base. \\

Cover & Buildings contribute to \hyperref[covering_terrain]{Hard Cover}. \\

Entering a Building & Units comprised entirely of Large and/or Standard Size Infantry​ and/or Beast models with a Boundary Rectangle no bigger than \SI{125}{\milli\meter} deep and wide can enter Buildings. \vspace*{5pt}\par
A unit can enter a Building by being deployed in or Advance Moving into an ungarrisoned Building, or by destroying a Garrisoning unit in combat (see Close Combat with a Building below). To enter a Building with an Advance Move, all models must start the Advance Move within their March Rate of the Building, and the unit must move to within \distance{1} of the Building. Remove the unit from the Battlefield. This unit is said to be Garrisoning the Building.\vspace*{5pt}\par
While Garrisoning a Building, the unit is considered to be of Gigantic Size for Line of Sight and Cover purposes, to occupy the Boundary Rectangle of the Building, and to keep the formation (number of ranks and files) it had when entering the Building. The unit is considered to have its Front Arc facing towards all directions. Every model can draw Line of Sight from any point of the Building (and in any direction). The Garrisoning unit cannot move, other than \hyperref[reform]{Reforming} or leaving the Building (see below).\vspace*{5pt}\par
The Boundary Rectangle of a unit Garrisoning a building can never exceed a width and depth of \SI{125}{\milli\meter}.\\

Set Ablaze & Units Garrisoning a Building gain \hyperref[flammable]{\textbf{Flammable}}. \\

Safe Position & Units Garrisoning a Building gain \hyperref[fearless]{\textbf{Fearless}}, \hyperref[stubborn]{\textbf{Stubborn}}, and are automatically destroyed when Broken. They may not Flee.\\

Leaving a Building & The only way to leave a Building is by making a special type of \hyperref[reform]{Reform}: the unit is deployed in any legal formation, with all models fully within their March Rate from the Building and with at least one model \distance{1} from the Building. The unit cannot move further until the end of this Movement Phase (not even if the unit has \hyperref[light_troops]{Light Troops}, and this cannot be made as a part of a \hyperref[swift_reform]{Swift Reform}).\\

Shooting and Measuring from a Building & Up to 10 models may shoot from a Building (regardless of the Garrisoning unit's formation). Units Garrisoning a Building are considered to be in \hyperref[covering_terrain]{Hard Cover}. To measure distances from and to any model of a Garrisoning unit, measure to and from the closest point on the Footprint of the Building.\\

Assaulting a Building & Declaring and resolving charges against a unit in a Building is done following the normal rules with the following exceptions. Move the charging unit into contact with the Building, aligning and maximising contact as normal, treating the Building as a single model. The Building cannot be moved when aligning units. The Garrisoning unit is always considered to be charged in the Front Facing.\\
\end{tableterrain}

\begin{tableterrain}
Close Combat with a Building & Assaulting and Garrisoning units do not attack each other. Instead they attack the Building, as if it was a single enemy model. Assaulting units attack it as normal (base contact and using Supporting Attacks). Only models in the first rank of the Garrisoning unit can attack. Garrisoning models cannot issue or accept \hyperref[duels]{Duels}. The building has Defensive Skill 5. \hyperref[stomp_attacks]{Stomp Attacks} and \hyperref[impact_hits]{Impact Hits} cannot be used against a Building.\vspace*{5pt}\par
\rewordedrule{For rules affecting units or models in base contact, Assaulting and Garrisoning units are considered to be in base contact with one another. Every model in the Garrisoning unit's first rank is considered to be in base contact with every model in the Assaulting units' first ranks and vice versa.}\vspace*{5pt}\par
Hits scored by Assaulting units are distributed onto the Garrisoning unit. Hits scored by the Garrisoning unit are distributed \rewordedrule{onto} Assaulting units. If there are more than one Assaulting unit, the owner of the Garrisoning unit divides the hits as evenly as possible between all Assaulting units.\vspace*{5pt}\par
Calculate the Combat Score and roll Break Tests as normal.\vspace*{5pt}\par
If the Garrisoning unit is destroyed (remember that a Breaking Garrisoning unit is destroyed), one of the Assaulting units may choose to enter the Building, provided it has the right Size, Type, and Boundary Rectangle, and it is not Engaged in Combat. If the Assaulting unit does not enter the Building, follow the rules for Assaulting units that stop fighting (see below).\vspace*{5pt}\par
If the Assaulting unit Breaks, it Flees as normal. The Garrisoning unit cannot Pursue.\par
If there are still units left Engaged in the Combat, the winning side may choose what each Assaulting unit will do: continue, or stop fighting. Assaulting units Engaged in Combat with other units than the Garrisoning unit must continue fighting.\vspace*{3pt}\par
\hspace*{0.3cm}- An Assaulting unit that continues fighting will remain Engaged in Combat with the Garrisoning unit. Both units can \hyperref[combat_reform]{Combat Reform} if able to (a Garrisoning unit's new formation must not exceed \SI{125}{\milli\meter} wide or deep).\par
\hspace*{0.3cm}- An Assaulting unit that stops fighting is nudged \distance{1} directly away from the Building (centre of Building to Centre of unit), or as far as possible without breaking the \hyperref[unit_spacing]{Unit Spacing} rule, and then performs either a \hyperref[post_combat_pivot]{Post-Combat Pivot} or a \hyperref[post_combat_reform]{Post-Combat Reform}.
\\
\end{tableterrain}

\subsection{Cliffs}
\label{cliffs}

\begin{tableterrain}%
    Types&%
    Cliffs are \hyperref[opaque_terrain]{Opaque Terrain} and \hyperref[impassable_terrain]{Impassable Terrain}.%
    \\
    Cover&
    Cliffs contribute to \hyperref[covering_terrain]{Hard Cover}.\\
\end{tableterrain}

\subsection{Fields}
\label{fields}

\begin{tableterrain}
Types& Fields are \hyperref[covering_terrain]{Covering Terrain} for units \textbf{inside} them.\\
Cover &Fields contribute to \hyperref[covering_terrain]{Soft Cover}, except for models with \hyperref[towering_presence]{Towering Presence}.\\
\end{tableterrain}

\subsection{Forests}
\label{forests}

\begin{tableterrain}
Types & Forests are \hyperref[covering_terrain]{Covering Terrain} for units \textbf{inside and/or behind them}, and \hyperref[dangerous_terrain]{Dangerous} \hyperref[dangerous_terrain]{Terrain (1)} for Cavalry, Constructs, and units making a \hyperref[fly]{Flying Movement}.\\
Cover & Forests contribute to \hyperref[covering_terrain]{Soft Cover}. \\
Broken Ranks & Units with more than half of their Footprint in a Forest can never be \hyperref[steadfast]{Steadfast}.\\
Stubborn & Units with more than half of their Footprint in a Forest and consisting entirely of Infantry with \hyperref[light_troops]{Light Troops} excluding models with \hyperref[towering_presence]{Towering Presence} and/or \hyperref[fly]{Fly}, are \hyperref[stubborn]{\textbf{Stubborn}}.\\
\end{tableterrain}

\subsection{Hills}
\label{hills}

\begin{tableterrain}
Types& Hills are \hyperref[opaque_terrain]{Opaque Terrain} for units \textbf{behind} them.\\
Cover & Hills contribute to \hyperref[covering_terrain]{Soft Cover} for targets behind \textbf{but partially on} them.\newline
Hills contribute to \hyperref[covering_terrain]{Hard Cover} for targets behind \textbf{and entirely off} them.\\
Elevated Position & Models with more than half their base on a Hill are considered to be Elevated. Ignore all intervening non-Elevated models if you are:
\begin{itemize}[label={-}]
\item drawing Line of Sight to or from Elevated models.
\item determining Cover when shooting with:
\begin{itemize}[label={\textbullet}]
\item Elevated models.
\item non-Elevated models at units which have more than half of their models Elevated.
\end{itemize}
\end{itemize}\\
Charging Downhill & A unit initiating a Charge Move with more than half of its Footprint on a Hill towards an enemy with more than half of its Footprint off a Hill must reroll failed Charge Range rolls.\\
\end{tableterrain}

\subsection{Ruins}
\label{ruins}

\begin{tableterrain}
Types & Ruins are \hyperref[covering_terrain]{Covering Terrain} for units \textbf{inside} them, \hyperref[dangerous_terrain]{Dangerous Terrain~(2)} for Cavalry and Constructs without \hyperref[skirmisher]{Skirmisher}, and \hyperref[dangerous_terrain]{Dangerous Terrain~(1)} for any other unit without \hyperref[skirmisher]{Skirmisher}.\\
Cover & Ruins contribute to \hyperref[covering_terrain]{Hard Cover}, except for models with \hyperref[towering_presence]{Towering Presence}.\\
\end{tableterrain}

\subsection{Walls}
\label{walls}

\begin{tableterrain}
Types & Walls are \hyperref[covering_terrain]{Covering Terrain} for models \textbf{behind} them while Defending the Wall (see below), and \hyperref[dangerous_terrain]{Dangerous Terrain~(2)} for Constructs.\\
Cover & Walls contribute to \hyperref[covering_terrain]{Hard Cover}, except for models with \hyperref[towering_presence]{Towering Presence}.\\
Defending a Wall & In order to Defend a Wall, more than half of a unit's Front Facing must be in contact with it.\\
Combat & Units Defending a Wall gain \hyperref[distracting]{\textbf{Distracting}} against Close Combat Attacks from charging enemies in their Front Facing. \\
\end{tableterrain}

\subsection{Water}
\label{water}

\begin{tableterrain}
Types & Water is \hyperref[dangerous_terrain]{Dangerous Terrain~(1)} for Standard Size models on foot.\\
Broken Ranks & Units with more than half of their Footprint in Water can never be \hyperref[steadfast]{Steadfast}. \\
Doused Flames & All Melee Attacks against or by models in units with more than half of their Footprint inside a Water Terrain Feature lose \hyperref[flaming_attacks]{Flaming Attacks} (if they had it).\\
\end{tableterrain}


