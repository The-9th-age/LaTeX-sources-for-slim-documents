\part{Melee Phase}
\label{melee_phase}

In the Melee Phase, the players' models Engaged in Combat must strike.

\section{Engaged in Combat}
\label{engaged_in_combat}

Units are considered Engaged in Combat if one or more models in the unit is in base contact with an enemy unit. If a unit is Engaged in Combat, all models in the unit are also Engaged in Combat. Units that are Engaged in Combat cannot move unless specifically stated (such as during \hyperref[combat_reform]{Combat Reforms} or when \hyperref[break_test]{Breaking}).

\subsubsection{First Round of Combat}

Certain rules only apply to the first Round of Combat. A unit's first Round of Combat is:

\begin{itemize}[label={-}]
\item the Round of Combat after it successfully charges an enemy unit.
\item the Round of Combat after it is successfully charged by an enemy unit if previously unengaged.
\end{itemize}

\section{Melee Phase Sequence}
\label{melee_phase_sequence}

Each Melee Phase is divided into the following steps.\par

\hspace*{0.3cm}
\begin{tabular}{c | m{14cm}}
1 & Start of the Melee Phase. Apply any \hyperref[no_longer_engaged]{No Longer Engaged}.\\
2 & The Active Player chooses a combat to fight that has not already been fought during this Melee Phase.\\
3  & Fight a Round of Combat. \\
4 & Repeat steps 2 and 3. \\
5 & Once all units that were Engaged in Combat at the start of the phase have fought, the Melee Phase ends.\\
\end{tabular}
\par

A combat is defined as a group of opposing units, which are all connected through base contact. Normally, this would be two units pitted against one another, but it could also be several units against a single enemy unit or a long chain of units from both sides. Complete all actions in the Round of Combat Sequence of all units involved in the chosen combat before moving on to the next combat.

\section{No Longer Engaged}
\label{no_longer_engaged}

A unit follows the rules described in \totalref{no_more_foes} if it was Engaged in Combat previously, but had all of its opponents moved or removed between the previous Movement Phase and this Melee Phase (and base contact could not be maintained through nudging, following the instructions under \totalref{dropping_out_of_combat}). That unit may do a \hyperref[post_combat_pivot]{Post-Combat Pivot} or \hyperref[post_combat_reform]{Post-Combat Reform} (or \hyperref[overrun]{Overrun} if it just charged) before any combats are fought. This cannot be done if the unit has moved since the opposing units were removed (e.g. with a Magical Move).

\newpage
\section{Round of Combat Sequence}
\label{round_of_combat_sequence}

Each Round of Combat is divided into the following steps:

\begin{center}
\begin{tabular}{c|m{14cm}}
1 & Start of the Round of Combat.\\
2 & Choose Weapon (see \totalref{close_combat_weapons}).\\
3 & \hyperref[make_way]{Make Way} (see \totalref{characters}).\\
4 & \hyperref[issuing_a_duel]{Issue and Accept Duels} (see \totalref{duels}).\\
5 & Determine the Initiative Order. \\
6 & Roll Melee Attacks, starting with the first Initiative Step:
	\begin{enumerate}[parsep=0cm,itemsep=0.05cm, topsep=3pt]
		\item Allocate attacks.
		\item Roll to hit, to wound, saves, and remove casualties.
		\item Move to the next Initative Step.
 	\end{enumerate}
\tabularnewline[-12pt]
7 & Calculate which side wins the Round of Combat. Losers roll Break Tests.\\
8 & Roll \hyperref[panic_test]{Panic Tests} for friendly units within \distance{6}. \\
9 & Restrain or Pursue? \\
10 & Roll Flee Distances. \\
11 & Roll Pursuit Distances. \\
12 & Move Fleeing units. \\
13 & Move Pursuing units. \\
14 & \hyperref[post_combat_pivot]{Post-Combat Pivots} and/or \hyperref[post_combat_reform]{Post-Combat Reforms}. \\
15 & \hyperref[combat_reform]{Combat Reforms}. \\
16 & End of the Round of Combat. Proceed to the next combat.\\
\end{tabular}
\end{center}

\subsection{Initiative Order}
\label{initiative_order}

Melee Attacks are performed in Rounds of Combat. These occur during the Melee Phase. All Melee Attacks have a specific Agility value which corresponds to the Agility of their model part, unless explicitly specified otherwise (such as Impact Hits which are performed at Initiative Step 10, or Crush Attacks at Initiative Step 0). 
\par
Each Round of Combat is fought in a strict striking order, decided at the beginning of the round. The order starts with all attacks that have Agility 10 and then works downwards from the highest to the lowest Agility. Agility above 10 is resolved at Agility 10, and Agility below 0 is resolved at Agility 0.
\par
This process is referred to as Initiative Order. The Initiative Order in a combat is determined immediately before any attacks are made. Take into account all modifiers that affect the Agility of attacks that may be performed in this Round of Combat to sort these attacks in Initiative Steps. Once the Initiative Order has been determined for a Round of Combat, it cannot be changed by effects that alter the Agility of attacks during that Round of Combat.
\par
Once the Initiative Order has been established, at each Initiative Step, all attacks from this step that meet the necessary requirements (see \totalref{which_models_can_attack}) strike simultaneously.

\subsection{Charging Momentum}
\label{charging_momentum}

Charging models gain +1 Agility.

\subsection{Which Models can Attack}
\label{which_models_can_attack}

Models in base contact with an enemy attack when it is their turn in the Initiative Order (remember that models are considered to be in base contact across empty gaps). Models from both sides attack in each player's Melee Phase.

\subsubsection{Supporting Attacks}
\label{supporting_attacks}

Models in the second rank, and not in base contact with any enemies, can perform Close Combat Attacks across models in the first rank directly in front of them. These are called Supporting Attacks. A model part that performs Supporting Attacks \textbf{always} has a maximum Attack Value of X, where X is defined by the Size of the model (see \totalref{model_classification}).

Figure \ref{figure/empty_gaps} illustrates this subsection.

\newcommand{\figEGA}{a)}
\newcommand{\figEGB}{b)}

\begin{figure}[!htbp]
\centering
\def\svgwidth{0.7\textwidth}
\input{pics/empty_gaps.pdf_tex}
\caption{Attacks over empty gaps.\captionpar
Models colour-coded with a darker shade can all strike. Models with a bold frame count as being in base contact with an enemy (models are considered to be in base contact across empty gaps). Models colour-coded with a lighter shade cannot make Supporting Attacks.\captionpar
a) The magenta unit is in Line Formation and thus both the second and third rank can make Supporting Attacks. The green unit is not Engaged in its Front Facing; its models cannot make Supporting Attacks to their Flank or Rear (they could only strike across the first rank).\captionpar
b) The turquoise models with a bold frame are considered to be in base contact with the model on the round base, since this round base is directly in front of them.
}
\label{figure/empty_gaps}
\end{figure}

\subsection{Allocating Attacks}
\label{allocation_attacks}

At each Initiative Step, before any attacks are rolled, Close Combat Attacks must first be allocated towards enemy models. If a model is in base contact with more than one model, it can choose which model to attack. Attacks can be allocated towards models with different Health Pools, i.e. \rnf{} models, Champions, and Characters (see \totalref{attacks}). The number of Close Combat Attacks that a model can make is equal to its Attack Value. Equipment, Attack Attributes, spells, etc. can change this number. If a model has an Attack Value above one, it can allocate its Close Combat Attacks at will to\rewordedrule{wards} different targets in base contact. If a model is making Supporting Attacks, it can allocate its attacks as if it was in the first rank of the unit (in the same file). If a model could either strike at models in base contact or make Supporting Attacks, it must allocate its attacks \rewordedrule{towards} models in base contact. Allocate all attacks at a given Initiative Step before rolling any to-hit rolls.


\newpage
\subsubsection{Swirling Melee}
\label{swirling_melee}

If a \rnf{} model is in base contact with an enemy, or in a position to make Supporting Attacks, it may  choose to allocate any number of its Close Combat Attacks towards \rnf{} models instead of Characters or Champions in a unit that the attacker could normally allocate attacks towards (ignoring Duels). I.e. \rnf{} models are not limited to only allocating Close Combat Attacks towards models in base contact but can also allocate them towards \rnf{} models of the same unit instead. Note that Swirling Melee cannot be used by Characters.

Figure \ref{figure/allocate_attacks} illustrates how attacks can be allocated in a complex case.

\newcommand{\figAHCharOne}{$C_{1} $}
\newcommand{\figAHCharTwo}{$C_{2} $}
\newcommand{\figAHCharThree}{$C_{3} $}
\newcommand{\figAHChamp}{Ch}

\begin{figure}[!htbp]
\begin{minipage}{0.48\textwidth}
\def\svgwidth{\textwidth}
\input{pics/allocate_attacks.pdf_tex}
\end{minipage}\hfill\begin{minipage}{0.51\textwidth}
\caption{Example for Allocating Attacks.\vspace*{10pt}\newline
The Champion of the magenta unit (Ch) and Character $C_{2} $ are locked in a Duel (indicated by the chess pattern). This means that they can only strike at each other. The magenta and green models can strike at the \rnf{} models of the other unit. The models with a bold frame can strike at Characters/Champions. The models in fainter colours with dashed frames cannot strike at all. Character $C_{1} $ cannot strike because the only model it is in base contact with is a Champion that is locked in a Duel. If $C_{1} $ was a \rnf{} model, it could strike the magenta \rnf{} models.}
\label{figure/allocate_attacks}
\end{minipage}
\end{figure}


\subsection{Rolling to Hit}
\label{rolling_to_hit}
\cftaddtitleline{toc}{section}{\normalfontsize{}\hspace*{0.5em}\hyperref[rolling_to_hit]{Rolling to Hit}}{\arabic{page}}

To make to-hit rolls, roll a D6 for each Close Combat Attack. The comparison of the Offensive Skill of the attacking model part and the Defensive Skill of the model the attack was allocated towards determines the needed roll to successfully hit the target. See table \ref{table/close_combat_to_hit_table} below. 
\par
To-hit modifiers can alter this to-hit roll. Unless specified otherwise, a to-hit modifier applies to both Shooting and Close Combat to-hit rolls. Close Combat to-hit rolls always fail on a natural roll of \result{1}, while they are always successful on a natural roll of \result{6}.
\par
Example: a model has Offensive Skill 3, Attack Value 2, and is equipped with Paired Weapons which gives it a total of three attacks. The model may allocate two attacks towards a model with Defensive Skill 2, which hit on 3+, and one towards a model with Defensive Skill 8, which hits on 5+. If one or more hits are scored, the player may then follow the procedure described under \totalref{attacks}.

\begin{table}[!htbp]
\centering
\begin{tabular}{r l}
  %\toprule
  Offensive Skill - Defensive Skill & Needed roll to hit\\
  \midrule
  4 or more & 2+ \\
  1 to 3 & 3+ \\
  0 to -3 & 4+\\
  -4 to -7 & 5+\\
  -8 or less & 6+\\
  %\bottomrule
\end{tabular}
\caption{The Close Combat to-Hit Table.}
\label{table/close_combat_to_hit_table}
\end{table}

\newpage
\subsection{Dropping out of Combat}
\label{dropping_out_of_combat}

Removing casualties may cause units to drop out of base contact with their foe. When this happens, units are nudged back into combat using the following procedure:
\begin{enumerate}
\item The unit that is going to drop out of combat while not suffering casualties is moved the minimum amount needed to keep the units in base contact.
\item If this will not bring the units back into contact, move the unit suffering casualties the minimum amount needed to keep the units in base contact.
\end{enumerate}
A nudged unit can only be moved straight forward, backward, to either side, or a combination of two of these directions (first one, then the other). Units that are in base contact with other enemy units can never be nudged in this way. Nudged units cannot move through other units or Impassable Terrain, but they are allowed to move within \distance{1} of other units in the same combat. Nudge moves cannot be used to change the Facing in which any unit is fighting (which means that if the unit was attacked in the Flank before the nudge move, this must still be true after the nudge move). If several units drop out of combat at the same time, move them in the order that allows the maximum number of units to stay in combat. If this number is equal, the Active Player decides the order.\par

If nudging either unit does not manage to bring the units back into contact with each other, the unit drops out of combat. Any units that are no longer Engaged in Combat follow the rules given under \totalref{no_more_foes}.
\par

If base contact was lost due to units deliberately moving out of combat (e.g. as a result of the effect of a spell), do not nudge the units back together. Instead follow the rules given under \totalref{no_more_foes}.

\section{Duels}
\label{duels}

\subsection{Issuing a Duel}
\label{issuing_a_duel}

Characters and Champions Engaged in a Combat may issue a Duel at step 4 of the Round of Combat Sequence (see \totalref{round_of_combat_sequence}). The Active Player may nominate one of their Characters or Champions and Issue a Duel, provided that there is at least one enemy Champion or Character able to accept it (this enemy model's unit must be in base contact with the challenging model's unit and there must be no ongoing Duel in this combat; see below). If no Duel was issued, the Reactive Player may nominate one of their Characters or Champions and issue a Duel.

\subsection{Accepting or Refusing a Duel}
\label{accepting_and_refusing_a_duel}

If a Duel was issued, the opposing player may now choose one of their own Characters or Champions Engaged in the same Combat to accept the Duel and fight the Character or Champion that issued the Duel. The model that accepts the Duel must be in a unit that is in base contact with the model that issued the Duel, or its unit.

If a Duel isn't accepted it is said to be refused. The player issuing the Duel now nominates one of their opponent's Characters, which could have Accepted the Duel, if there is any (note that Champions cannot be nominated). The chosen model:
\begin{itemize}[label={-}]
\item has its Discipline \textbf{set} to 0, and it loses \hyperref[stubborn]{Stubborn} (if it has it). Both effects last until the end of the Player Turn in which the combat ends or until the Character Accepts or Issues a Duel.
\item cannot make any Melee Attacks during this Round of Combat.
\item loses \hyperref[rally_around_the_flag]{Rally Around the Flag} (if it has it).
\item in case of a \hyperref[bsb]{Battle Standard Bearer}, doesn't add +1 to its side's Combat Score during this Round of Combat.
\end{itemize}

\subsection{Fighting a Duel}
\label{fighting_a_duel}

If the Duel was accepted, the model that Issued the Duel and the model that Accepted the Duel count as being in base contact with each other (even if their bases are not physically touching each other) and must allocate all their Close Combat Attacks towards each other. Melee Attacks made towards a unit as a whole (such as \hyperref[breath_attack]{Breath Attacks}, \hyperref[impact_hits]{Impact Hits}, \hyperref[grind_attacks]{Grind Attacks}, \hyperref[stomp_attacks]{Stomp Attacks}) can only be distributed onto the opposing duellist. Melee Attacks made at specific models (such as all models in base contact) are unaffected and work as normal. No other model can allocate attacks towards either of these models, and attacks/hits from Melee Attacks can never be distributed onto a model that is fighting a Duel. If one of the models is killed in the Melee Phase before the other model had a chance to make all its Melee Attacks (this is a common situation with Characters with more than one Agility value, such as a rider and its mount, or a model with Stomp Attacks), any of the attacks not yet carried out can and must be directed at the killed model, as if it was still alive and in base contact, in order to get an Overkill bonus.
\par
If one of the models is killed, Breaks, or if the combat ends for any reason (including being divided through Splitting Combat), the Duel ends at the end of the \rewordedrule{p}hase. If neither model is killed and both their units are still Engaged with one another at the beginning of the next Round of Combat, the Duel continues. No other Duel can be issued in the same combat before the Duel ends. 

\subsection{Overkill}
\label{overkill}

During a Duel, any excess Health Point losses caused count towards the Combat Score, up to a maximum of +3.


\section{Winning a Round of Combat}
\label{winning_a_round_of_combat}

\subsection{Combat Score}

Once all Initiative Steps have passed (i.e. all models have had a chance to attack), the winner of this Round of Combat is determined. This is done by calculating each side's Combat Score. Simply add up all Combat Score bonuses. The side with the higher Combat Score wins the combat and the side with the lower Combat Score loses the combat. If there is a tie, both sides are treated as winners. Bonuses are described below and summarised in table \ref{table/combat_score}.

\subsubsection{Lost Health Points on enemy units: +1 for each Health Point}

Each player adds up the number of Health Points lost from their opponent's units (Engaged in the same Combat) during this Round of Combat. This includes enemies that were Engaged in the Combat but dropped out or were completely wiped out during this Round of Combat.

\subsubsection{Overkill: +1 for each Health Point (maximum +3)}

In a Duel, excess Health Points lost by an opponent after it was killed are counted towards the Combat Score. A maximum of +3 can be added to your Combat Score due to Overkill. Note that excess lost Health Points are only counted when in a Duel. In all other situations, excess lost Health Points count for nothing.

\subsubsection{Charge: +1}

Each side with one or more charging models receives +1 to their Combat Score.

\subsubsection{Rank Bonus: +1 for each rank (maximum +3)}

Each side adds +1 to their Combat Score for each \hyperref[full_ranks]{Full Rank} after the first in a single unit, up to a maximum of +3. Only count this for a single unit per side. (Use the unit that gives the highest Rank Bonus). Units in \hyperref[line_formation]{Line Formation} cannot add Rank Bonus to their Combat Score.

\subsubsection{Standards: +1 for each Standard and Battle Standard Bearer}

Each side adds +1 to their Combat Score for each Standard Bearer and Battle Standard Bearer Engaged in Combat at the end of the Round of Combat.





\subsubsection{Flank Bonus: +1 or +2}

Each side adds +1 to their Combat Score if they have one or more units fighting an enemy in the enemy's Flank. If at least one of these units (that are fighting an enemy in its Flank) has one or more Full Ranks, add +2 instead.
 
\subsubsection{Rear Bonus: +2 or +3}

Each side adds +2 to their Combat Score if they have one or more units fighting an enemy in the enemy's Rear. If at least one of these units (that are fighting an enemy in its Rear) has one or more Full Ranks, add +3 instead.

\subsubsection{Combat Score Summary}

\begin{table}[!htbp]
\centering
\begin{tabular}{r l }
\toprule
Health Points Lost by Enemy Units & +1 for each Health Point\\
Overkill & +1 for each Health Point (maximum +3)\\
Charge & +1 \\
Rank Bonus & +1 for each rank (maximum +3)\\
Standard & +1 for each Standard and Battle Standard Bearer \\
Flank Bonus & +1 or +2 \\
Rear Bonus & +2 or +3\\
\bottomrule
\end{tabular}
\caption{Combat Score Summary.}
\label{table/combat_score}
\end{table}



\section{Break Test}
\label{break_test}

Each unit on the side that lost the Round of Combat must take a Break Test. The order is chosen by the losing player. A Break Test is a Discipline Test with a negative modifier to its Discipline equal to the Combat Score difference (i.e. if the Combat Score was 6 to 3, the units on the losing side take Break Tests with a -3 modifier to their Discipline). If the test is failed, the unit Breaks and Flees. If the test is passed, the unit remains Engaged in the Combat.

\subsubsection{Steadfast}
\label{steadfast}

Any units that have more Full Ranks than each of the enemy units Engaged in the same Combat ignore Discipline modifiers from the Combat Score difference when rolling Break Tests (and tests to \hyperref[combat_reform]{Combat Reform}). 

\subsubsection{Disrupted Ranks}
\label{disrupted_ranks}

A unit cannot use the Steadfast rule if it is Engaged in Combat in its Flank or Rear with an enemy unit with at least 2 Full Ranks.

\subsubsection{No More Foes}
\label{no_more_foes}

Sometimes a unit kills all enemy units in base contact and finds itself no longer Engaged in Combat (so it cannot provide Combat Score Bonuses such as Standards or Flank). These units always count as winning the combat, and can either make an \hyperref[overrun]{Overrun} (if they were charging), a \hyperref[post_combat_pivot]{Post-Combat Pivot}, or a \hyperref[post_combat_reform]{Post-Combat Reform}.

When this happens in multiple combats, the Combat Score resulting from lost Health Points by the unit and its opponents counts, but all other Combat Score bonuses are ignored. Note that the unit itself doesn't take a Break Test since it always counts as if on the winning side.

\subsubsection{Splitting Combat}
\label{splitting_combat}

If due to removing casualties, two or more disconnected subgroups of opposing units are created (see figure \ref{figure/splitting_combat}), resolve the Combat normally (accounting for every unit that took part in this Round of Combat), checking any remaining base contact for the purpose of Rear and Flank Bonuses. In the next Melee Phase, each subgroup will be treated as a different combat.

\newcommand{\figSplitA}{a)}
\newcommand{\figSplitB}{b)}

\begin{figure}[!htbp]
\centering
\def\svgwidth{0.7\textwidth}
\input{pics/splitting_combat.pdf_tex}
\caption{Splitting Combat.\captionpar
The purple Infantry unit suffers casualties, which results in the yellow unit no longer being in base contact. Neither the purple nor yellow unit can be nudged back into base contact since they are in base contact with other enemies, (see \totalref{dropping_out_of_combat}). Calculate Combat Score here as one single combat (the pink unit grants a Flank Bonus). In the next Player Turn, this situation will count as two separate combats.}
\label{figure/splitting_combat}
\end{figure} 

\section{Pursuits and Overruns}
\label{pursuits_and_overruns}

Before moving broken units, each unit that is in base contact with the broken unit(s) may declare a Pursuit of a single broken unit (each Pursuing unit may choose any eligible enemy unit to Pursue). Determine the direction of the Flee Move as follows:
\begin{itemize}[label={-}]
\item If the broken unit is in contact with a single enemy unit, its \rewordedrule{Flee Move} will be directed away from that unit.
\item If the broken unit is in contact with more than one enemy unit, the owner of the enemy units must declare which of those units the \rewordedrule{Flee Move} will be directed away from.
\end{itemize}

To be able to Pursue a broken enemy, the unit cannot be Engaged with any non-broken enemy units and must be in base contact with the broken unit. Units can elect not to Pursue, but must then pass a Discipline Test to succeed in restraining themselves; if the test is failed, the unit must Pursue anyway. If the test is passed, the unit may do either a \hyperref[post_combat_pivot]{Post-Combat Pivot} or a \hyperref[post_combat_reform]{Post-Combat Reform}.

\subsubsection{Overrun}
\label{overrun}

A unit that fought its first Round of Combat after charging  can choose to make a special Pursuit Move called Overrun (instead of a Post-Combat Pivot), if all units in base contact were wiped out (including units being removed from play as a result of \hyperref[unstable]{Unstable} or something similar). Overruns follow the rules for moving Pursuing units, except that step 1. Pivot is ignored (i.e. Overruns are straight forward) and no Discipline Test is required for restraining from Pursuit.

\subsection{Roll for Flee and Pursuit Distances}

Every broken unit now rolls 2D6 to determine their Flee Distance, and each unit that has declared a Pursuit now rolls 2D6 to determine its Pursuit Distance. If any Pursuing unit rolls an \textbf{equal or higher} Pursuit Distance than the Flee Distance of the unit it is Pursuing, the Fleeing unit is immediately destroyed. Remove that unit from the game with no saves of any kind allowed. If several units are Fleeing from the same combat, the units move in the same order as their Flee Distance was rolled (the player controlling the units chooses in which order they roll the Flee Distance). The Active Player chooses which player will roll for their Pursuing units first. Each player chooses the order of their own Pursuing units. 

\subsection{Flee Distance and Fleeing Units}

Each broken unit that was not captured and destroyed will now Flee directly away from the previously determined enemy unit in base contact. Pivot the Fleeing unit so that its Rear Facing is parallel with the Facing it was engaged in (of the enemy unit the \rewordedrule{Flee Move} is directed away from), and then move the Fleeing unit directly forward a number of inches equal to the Flee Distance rolled earlier. Use the rules for \hyperref[flee_moves]{Flee Moves} (with the exception that units that are Engaged in the same combat do not cause Dangerous Terrain Tests).

\subsection{Pursuit Distance and Pursuing Units}
\label{pursuit_distance_and_pursuing_units}

Each Pursuing unit now performs a Pursuit Move, which is divided into three consecutive steps. If several units are Pursuing from the same combat, they move in the same order as their Pursuit Distance was rolled.

All enemy units that Fled from the combat involving the Pursuing unit are treated as Impassable Terrain for the Pursuit Move. All friendly units that were part of the same combat this Round of Combat are treated as Open Terrain for steps 1 and 2 of the Pursuit Move.

\paragraph{1. Pivot}

Before Pivoting, check which Arc the Pursuing unit is in for each enemy unit that may be charged later in this process. If the Pursuit Move will lead to a charge, it will be in the Facing determined at this point.

The Pursuing unit then Pivots (around its Centre) so that it is facing the same direction as the Pursued unit, or if destroyed, the direction the Pursued unit would have had, had it not been destroyed. Ignore the \hyperref[unit_spacing]{Unit Spacing} rule during this Pivot.

After the Pivot, one of the four situations below will arise. If more than one is applicable, apply the uppermost one.

\begin{itemize}[label={-}]
\item If the Front Facing of the Pursuing unit would overlap the Board Edge, the unit Pursues off the Table (see \totalref{pursuing_off_the_table}).
\item If the Front Facing of the Pursuing unit would overlap an enemy unit's Boundary Rectangle, it charges that unit. If there is more than one possible target, the pivoting unit chooses which to charge.  Remove the Pursuing unit from the Battlefield and then place it back on the Battlefield with its Front Facing in base contact with its target, in the Facing determined before the Pivot, maximising the number of Engaged models as normal but keeping the Centre of the unit as close as possible to its starting position while doing so. If there is not enough room to place the Pursuing unit, treat the enemy unit as Impassable Terrain instead.
\item If the Front Facing of the Pursuing unit would overlap with the Boundary Rectangle of a friendly unit (that was not part of the same combat) or Impassable Terrain, the unit instead Pivots so that it faces as close as possible towards the direction of the the Pursued unit, while abiding by the Unit Spacing rule (normally this means stopping \distance{1} away from the obstruction), and then moves no further (i.e. ignore steps 2 and 3).
\item If the Front Facing of the Pursuing unit touches neither another unit's Boundary Rectangle nor Impassable Terrain, proceed to step 2. Note that only the Front Facing needs to be clear: units, Impassable Terrain, or the Board Edge overlapping with other parts of the unit are ignored during steps 1-3.
\end{itemize}

\paragraph{2. Forward Ahead}

Without moving the Pursuing unit, check what the first obstacle (Board Edge, Boundary Rectangle, or Impassable Terrain) within the rectangle directly ahead of the unit formed by its Front Facing and the rolled Pursuit Distance would be. The Unit Spacing rule is ignored when doing this check and for all movement during Forward Ahead.

\paragraph{2.1. Board Edge}

If the first obstacle would be the Board Edge, move the unit straight forward until it comes into contact with the Board Edge and then follow the rules for \hyperref[pursuing_off_the_table]{Pursuing off the Table}.

\paragraph{2.2. Enemy Unit}

If the first obstacle would be an enemy unit, the Pursuing unit Declares a Charge against said unit, using its Pursuit Distance roll as its Charge Range. The Pursuing unit immediately performs a Charge Move (following all the normal \hyperref[move_chargers]{Move Chargers} rules) towards the Facing determined before the Pivot. The charged enemy may not perform any Charge Reactions (not even if already Fleeing).
\par
When a Pursuit Move leads to a charge, if the Pursuing unit joins a combat that has already been fought or was created during this Melee Phase, it will be resolved in the next Melee Phase (with the charging unit still counting as charging). If the Pursuing unit joins a combat that wasn't created during this Melee Phase and that hasn't been fought yet, the unit will have a chance to fight and Pursue again this phase.
\par
If the charge is not possible to complete, treat the enemy unit as Impassable Terrain and proceed to 2.3.

\paragraph{2.3. Friendly Unit\rewordedrule{,} Impassable Terrain\rewordedrule{, or No Obstacle}}

If the first obstacle\rewordedrule{,} \removedreworded{(}if any\removedreworded{)}\rewordedrule{,} would not be an enemy unit, the Pursuing unit now moves its Pursuit Distance directly forward. If this brings the Front Facing of the unit into base contact with the Boundary Rectangle of a friendly unit (that was not part of the same combat) or Impassable Terrain, the unit stops.

\paragraph{3. Legal Position?}

At the end of the Pursuit Move, check if the unit is in a legal position. It cannot be in base contact with a unit it didn't Declare a Charge against, and it must abide by the \hyperref[unit_spacing]{Unit Spacing} rule, including against friendly units that were part of the same combat. If the unit is not in a legal position, backtrack the move to the unit's last legal position where it obeys the Unit Spacing rule.

Figure \ref{figure/simple_pursuit} shows a simple example of a Pursuit Move, figure \ref{figure/two_units_pursuit} illustrates a case where two units are Pursuing into a new fight, and figure \ref{figure/pursuit} introduces more complex cases.

\subsection{Pursuing off the Table}
\label{pursuing_off_the_table}

When a unit Pursues off the Table, it will go off the Battlefield and will come back during the controlling player's next Movement Phase, using the rules for arriving \hyperref[ambush]{Ambushers}, with the following exceptions:
\begin{itemize}[label={-}]
\item It automatically arrives.
\item It must be placed with its back rank centred on a point at which it contacted the Board Edge, or as close as possible. 
\item It must arrive in the same formation as it left.
\item It does not count as destroyed at the end of the game, nor does it lose \hyperref[scoring]{Scoring}.
\end{itemize}


\subsection{Post-Combat Pivot}
\label{post_combat_pivot}

An unengaged unit may Pivot around its Centre and may reorganise models with the \hyperref[front_rank]{Front Rank} rule (they must still be in legal positions). This move is made after the Pursuing and Fleeing units have been moved.

\subsection{Post-Combat Reform}
\label{post_combat_reform}

An unengaged unit may perform a Reform manoeuvre. If this is done, the unit doesn't count as \hyperref[scoring]{Scoring} for claiming Secondary Objectives during this Player Turn, and the unit may not Declare any Charges in the following Player Turn. This move is made after the Pursuing and Fleeing units have been moved.


\newcommand{\figSimplePursA}{a)}
\newcommand{\figSimplePursB}{b)}

\begin{figure}[!htbp]
\begin{minipage}{0.52\textwidth}
\def\svgwidth{\textwidth}
\input{pics/simple_pursuit.pdf_tex}
\end{minipage}\hfill\begin{minipage}{0.45\textwidth}
\caption{Simple example of a Pursuit.\captionpar
a) The purple unit Breaks from combat. It Pivots to face away from the green unit, and then moves the Flee Distance forward.\captionpar
b) The green unit Pursues. It Pivots to face the same direction as the purple unit, and then moves the Pursuit Distance forward.}
\label{figure/simple_pursuit}
\end{minipage}
\end{figure}

\newcommand{\figTwoPursA}{a)}
\newcommand{\figTwoPursB}{b)}
\newcommand{\figTwoPursC}{c)}
\newcommand{\figTwoPursFirstPursDist}{First Pursuit Distance (blue unit)}
\newcommand{\figTwoPursSecPursDist}{Second Pursuit Distance (purple unit)}
\newcommand{\figTwoPursFleeDist}{\begin{minipage}{0.25\unitlength}\begin{center}Flee Distance (yellow unit)\end{center}\end{minipage}}

\begin{figure}[!htbp]
\begin{minipage}{0.48\textwidth}
\def\svgwidth{\textwidth}
\input{pics/two_units_pursuit.pdf_tex}
\end{minipage}\hfill\begin{minipage}{0.49\textwidth}
\caption{Example of two units Pursuing into the same enemy unit.\captionpar
a) The yellow unit loses the combat, Breaks, and Flees \distance{7}. The owner of the winning units chooses to roll for the blue unit's Pursuit Distance first. The purple unit's Pursuit Distance is \distance{8}, so it is equal or higher than the yellow unit's Flee Distance: the Fleeing unit is immediately destroyed.\vspace*{1cm}\captionpar
b) As the blue unit's Pursuit Distance was rolled first, it is moved first. The first obstacle within the rectangle directly ahead is an enemy unit, so the blue unit Declares a Charge against this green unit and moves using the normal rules for Moving Chargers (one Wheel allowed, Maximising Contact). The purple unit isn't considered yet.\vspace*{1cm}\captionpar
c) Now the purple unit performs its Pursuit Move (its Pursuit Distance was rolled second). The first obstacle within the rectangle directly ahead is the enemy green unit, because the blue unit is treated as Open Terrain (it was in the same combat). The purple unit Declares a Charge and moves in contact with the green unit. It can move through the blue unit (treated as Open Terrain). Note that this situation is not treated as a Multiple Charge, so the number of models in contact doesn't have to be maximised overall.}
\label{figure/two_units_pursuit}
\end{minipage}
\end{figure}

\newcommand{\figPursA}{a)}
\newcommand{\figPursB}{b)}
\newcommand{\figPursC}{c)}
\newcommand{\figPursD}{d)}
\newcommand{\figPursTextA}{\begin{minipage}{0.45\unitlength}\begin{center}\normalfontsize{The Front Facing of the Pursuing unit touches a friendly unit's Boundary Rectangle.}\end{center}\end{minipage}}
\newcommand{\figPursTextB}{\begin{minipage}{0.45\unitlength}\begin{center}\normalfontsize{The Front Facing of the Pursuing unit touches an enemy unit's Boundary Rectangle.}\end{center}\end{minipage}}
\newcommand{\figPursTextC}{\begin{minipage}[b]{0.4\unitlength}\normalfontsize{The first obstacle would be an enemy unit.}\end{minipage}}
\newcommand{\figPursTextD}{\begin{minipage}[b]{0.4\unitlength}\normalfontsize{The first obstacle would \textbf{not} be an enemy unit.}\end{minipage}}

\begin{figure}[!htbp]
\begin{minipage}{0.53\textwidth}
\def\svgwidth{\textwidth}
\input{pics/pursuit.pdf_tex}
\end{minipage}\hfill\begin{minipage}{0.44\textwidth}
\caption{Examples of Pursuits.\captionpar
a) The purple unit is in the green unit's Flank. The green unit wins combat, the purple unit Breaks and Flees. The green unit Pursues. Pivoting the green unit would make its Front Facing overlap with a friendly unit (light green). The Pivot is instead made as close as possible to the intended direction and the Pursuit Move ends.\vspace*{1.5cm}\captionpar
b) The purple unit is in the green unit's Flank. The green unit wins combat, the purple unit Breaks and Flees, the green unit Pursues. Pivoting the green unit would make its Front Facing overlap with an enemy unit (pink). The green unit is removed from the Battlefield and then placed back on the Battlefield with its Front Facing in base contact with the charged pink unit's Front Facing.\vspace*{3.5cm}\captionpar
c) The blue unit Breaks and Flees from the orange unit. No obstructions are encountered during the Pivot. The first obstacle the orange unit would encounter during its move ahead is the turquoise unit. The orange unit must now perform a Charge Move against the turquoise unit, Maximising Contact as usual.\vspace*{4cm}\captionpar
d) The blue unit Breaks and Flees from the orange unit. No obstructions are encountered during the Pivot. The first obstacle the orange unit would encounter during its move ahead is an Impassable Terrain. The orange unit is moved into contact with the Impassable Terrain. However, this position breaks the Unit Spacing rule. The orange unit's Pursuit move is backtracked until its last legal position.\vspace*{1cm}}
\label{figure/pursuit}
\end{minipage}
\end{figure}

\clearpage
\section{Combat Reform}
\label{combat_reform}

Each unit still Engaged in a Combat after all Fleeing and Pursuing units have moved (and after Post-Combat Pivots and Post-Combat Reforms have been performed) now performs a Combat Reform.

\begin{itemize}[label={-}]
\item Units on the losing side of the combat must pass a Discipline Test in order to do so. Apply the same modifiers as for the previous \hyperref[break_test]{Break Test} (i.e. apply the Combat Score difference, unless the unit is \hyperref[steadfast]{Steadfast}).
\item Units Engaged in more than one Facing (e.g. in both Front and Flank) can never perform any Combat Reforms.
\item After all Combat Reform tests have been taken, the Active Player decides which player performs their Combat Reforms first. After this player has completed all Combat Reforms with all their units (one at a time, in any order), the opponent Combat Reforms their units.
\item Each player may choose not to Combat Reform one or more of their units.
\end{itemize}

When performing a Combat Reform, remove a unit from the Battlefield and place it back, following these restrictions:

\begin{enumerate}
\item The unit must be placed in a legal formation (abiding by the \hyperref[unit_spacing]{Unit Spacing} rule, etc.).
\begin{itemize}[label={-}]
\item The unit is allowed to come within \distance{0.5} of units Engaged in the same Combat, but it cannot move into base contact with enemy units that it was not in base contact with before the Combat Reform.
\end{itemize}
\item The unit must be placed in base contact with the same enemy unit(s) as it was before the Combat Reform, and in the same Facing of the enemy unit(s).
\item All models in the unit must be placed with their centre within their March Rate from their position before the Combat Reform.
\item Characters that were in base contact with an enemy must still be after the Combat Reform.
\begin{itemize}[label={-}]
\item This applies to both enemy and friendly Characters.
\item A Character may end up in base contact with different enemy models than it was before the Combat Reform.
\end{itemize}
\item After each Combat Reform, there must be at least as many models of the Combat Reforming unit in base contact with enemy models as there were before.
\begin{itemize}[label={-}]
\item These don't have to be the same models.
\end{itemize}
\end{enumerate}

Furthermore, after a player has completed all their Combat Reforms, the same enemy models that were in base contact with opposing models before the Combat Reform must still be in base contact after the Combat Reform (but they may be Engaged with different models or even units).
