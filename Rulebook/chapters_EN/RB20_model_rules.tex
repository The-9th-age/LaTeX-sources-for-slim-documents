\part{Model Rules}
\label{model_rules}

These are rules that are applied to individual models or model parts, as described on their unit profile. Model Rules are divided into the following categories: Universal Rules, Character, Personal Protections, Armour Equipment, Weapons, Attack Attributes, Special Attacks.

\vspace*{10pt}
\hypertarget{specialrulestable}{}
\setlength\columnseprule{0.5pt}
\begin{multicols}{3}
\hbadness=10000 % remove warnings for underfull lines
\specialruletablesubtitle{universal_rules}{Universal Rules}
\specialruletableentry{ambush}{Ambush}
\specialruletableentry{bsb}{Battle Standard Bearer}
\specialruletableentry{bodyguard}{Bodyguard}
\specialruletableentry{channel}{Channel}
\specialruletableentry{chariot}{Chariot}
\specialruletableentry{commanding_presence}{Commanding Presence}
\specialruletableentry{engineer}{Engineer}
\specialruletableentry{fear}{Fear}
\specialruletableentry{fearless}{Fearless}
\specialruletableentry{feigned_flight}{Feigned Flight}
\specialruletableentry{fly}{Fly}
\specialruletableentry{frenzy}{Frenzy}
\specialruletableentry{front_rank}{Front Rank}
\specialruletableentry{ghost_step}{Ghost Step}
\specialruletableentry{insignificant}{Insignificant}
\specialruletableentry{light_troops}{Light Troops}
\specialruletableentry{magic_resistance}{Magic Resistance}
\specialruletableentry{massive_bulk}{Massive Bulk}
\specialruletableentry{not_a_leader}{Not a Leader}
\specialruletableentry{protean_magic}{Protean Magic}
\specialruletableentry{rally_around_the_flag}{Rally Around the Flag}
\specialruletableentry{random_movement}{Random Movement}
\specialruletableentry{scoring}{Scoring}
\specialruletableentry{scout}{Scout}
\specialruletableentry{skirmisher}{Skirmisher}
\specialruletableentry{stand_behind}{Stand Behind}
\specialruletableentry{strider}{Strider}
\specialruletableentry{stubborn}{Stubborn}
\specialruletableentry{supernal}{Supernal}
\specialruletableentry{swiftstride}{Swiftstride}
\specialruletableentry{tall}{Tall}
\specialruletableentry{terror}{Terror}
\specialruletableentry{towering_presence}{Towering Presence}
\specialruletableentry{unbreakable}{Unbreakable}
\specialruletableentry{undead}{Undead}
\specialruletableentry{unstable}{Unstable}
\specialruletableentry{vanguard}{Vanguard}
\specialruletableentry{war_machine}{War Machine}
\specialruletableentry{war_platform}{War Platform}
\specialruletableentry{wizard_apprentice}{Wizard Apprentice}
\specialruletableentry{wizard_adept}{Wizard Adept}
\specialruletableentry{wizard_master}{Wizard Master}
\specialruletableentry{wizard_conclave}{Wizard Conclave}

\vspace*{-5pt}
\specialruletablesubtitle{characters}{Characters}
\specialruletableentry{make_way}{Make Way}

\vspace*{-5pt}
\specialruletablesubtitle{personal_protections}{Personal Protections}
\specialruletableentry{aegis}{Aegis}
\specialruletableentry{cannot_be_stomped}{Cannot be Stomped}
\specialruletableentry{distracting}{Distracting}
\specialruletableentry{flammable}{Flammable}
\specialruletableentry{hard_target}{Hard Target}
\specialruletableentry{parry}{Parry}
\specialruletableentry{fortitude}{Fortitude}

\vspace*{-5pt}
\specialruletablesubtitle{armour_equipment}{Armour Equipment}
\specialruletablesubtitle{close_combat_weapons}{Close Combat Weapons}
\specialruletablesubtitle{shooting_weapons}{Shooting Weapons}
\specialruletablesubtitle{artillery_weapons}{Artillery Weapons}
\specialruletableentry{bolt_thrower}{Bolt Thrower}
\specialruletableentry{cannon}{Cannon}
\specialruletableentry{catapult}{Catapult}
\specialruletableentry{flamethrower}{Flamethrower}
\specialruletableentry{volley_gun}{Volley Gun}
\specialruletableentry{the_misfire_table}{The Misfire Table}

\vspace*{-5pt}
\specialruletablesubtitle{attack_attributes}{Attack Attributes}
\specialruletableentry{accurate}{Accurate}
\specialruletableentry{area_attack}{Area Attack}
\specialruletableentry{battle_focus}{Battle Focus}
\specialruletableentry{crush_attack}{Crush Attack}
\specialruletableentry{devastating_charge}{Devastating Charge}
\specialruletableentry{divine_attacks}{Divine Attacks}
\specialruletableentry{fight_in_extra_rank}{Fight in Extra Rank}
\specialruletableentry{flaming_attacks}{Flaming Attacks}
\specialruletableentry{harnessed}{Harnessed}
\specialruletableentry{hatred}{Hatred}
\specialruletableentry{inanimate}{Inanimate}
\specialruletableentry{lethal_strike}{Lethal Strike}
\specialruletableentry{lightning_reflexes}{Lightning Reflexes}
\specialruletableentry{magical_attacks}{Magical Attacks}
\specialruletableentry{move_or_fire}{Move or Fire}
\specialruletableentry{multiple_wounds}{Multiple Wounds}
\specialruletableentry{penetrating}{Penetrating}
\specialruletableentry{poison_attacks}{Poison Attacks}
\specialruletableentry{quick_to_fire}{Quick to Fire}
\specialruletableentry{reload}{Reload!}
\specialruletableentry{toxic_attacks}{Toxic Attacks}
\specialruletableentry{unwieldy}{Unwieldy}
\specialruletableentry{volley_fire}{Volley Fire}
\specialruletableentry{weapon_master}{Weapon Master}

\vspace*{-5pt}
\specialruletablesubtitle{special_attacks}{Special Attacks}
\specialruletableentry{breath_attack}{Breath Attack}
\specialruletableentry{grind_attacks}{Grind Attacks}
\specialruletableentry{impact_hits}{Impact Hits}
\specialruletableentry{stomp_attacks}{Stomp Attacks}
\specialruletableentry{sweeping_attack}{Sweeping Attack}
\end{multicols}

\subsubsection{Duplicated Model Rules}

Sometimes a model or model part may have the same Model Rule more than once, for example when a model gains a Model Rule that it already had. In such a case, the effects of the duplicated Model Rule do not stack and do not offer any additional benefit, unless specifically noted otherwise. If the duplicated Model Rule has different values in brackets (X), use the version with the highest value. If X is the result of a dice roll, you may instead choose which version to use (before rolling any dice). If X is not a value, the Model Rules are not considered to be duplicates of the same Model Rule and both rules are applied (e.g. Hatred (against Infantry) and Hatred (against Cavalry) are considered two different Model Rules, apply the effects of both).

\section{Universal Rules}
\label{universal_rules}

If at least one model part has a Universal Rule, it affects the entire combined model. For example, if a Character with a Universal Rule mounts a Character mount without this Universal Rule, the combined model (Character and mount) are both affected by the Universal Rule.

\subsubsection{Ambush}
\label{ambush}

You may choose to not deploy units with Ambush, but instead let them Ambush by bringing them into play later on. Declare which units will be Ambushing during step 8 of the \hyperref[the_pre_game_sequence]{Pre-Game Sequence} (after Spell Selection), starting with the player that picked the Deployment Zone. Deploy your army as usual, but without the Ambushing units. Starting with your Player Turn 2, at the end of step 2 of the \hyperref[the_movement_phase_sequence]{Movement Phase Sequence} (after moving units with \hyperref[random_movement]{Random Movement}), roll a dice for each of your Ambushing units. After rolling for all Ambushing units, all units that rolled 3+ enter the Battlefield from any Board Edge. Place the arriving units with their entire rear ranks touching the Board Edge.

\begin{itemize}[label={-}]
\item Ambushing models can neither March Move during the Movement Phase in which they arrive, nor can they voluntarily end that Movement Phase further away from the Board Edge than their March Rate.
\item Ambushing models count as having moved during the turn they arrive on the Battlefield.
\item If an Ambushing unit has not entered the Battlefield before the end of the game (e.g. due to failing all its 3+ rolls), the unit counts as destroyed.
\item An Ambushing unit that enters the Battlefield on Game Turn 4 or later loses \hyperref[scoring]{Scoring}.
\item An Ambushing Character may be deployed within an Ambushing unit that it is allowed to join (declare this when declaring which units are Ambushing). Roll only one dice for the combined unit.
\item Until arriving on the Battlefield, Ambushing units cannot perform any actions at all, and all Special Equipment, rules, and abilities don't work while off the table.
\end{itemize}

\subsubsection{Battle Standard Bearer - \oneofakind{}}
\label{bsb}

An army may only include a single Battle Standard Bearer. The model gains \hyperref[rally_around_the_flag]{\textbf{Rally Around the Flag}} and \hyperref[not_a_leader]{\textbf{Not a Leader}}. If the model has the option to buy Special Equipment, it is allowed to purchase up to two Banner Enchantments.

\subsubsection{Bodyguard (X)}
\label{bodyguard}

When a Character is joined to a unit in which at least one model has Bodyguard, that Character gains \hyperref[stubborn]{\textbf{Stubborn}}. When Characters or Character types are stated in brackets, Bodyguard only works for the specified Characters or Character types.

\subsubsection{Channel (X)}
\label{channel}

During step 3 of the \hyperref[magic_phase_sequence]{Magic Phase Sequence}, each of the Active Player's models with Channel may add X Veil Tokens to the owner's Veil Token pool. This Universal Rule is cumulative, adding X Veil Tokens of each independent instance of Channel to the model's total Channel value (so e.g. a model with Channel (1) and Channel (2) is treated like a model with Channel (3)).

\subsubsection{Chariot}
\label{chariot}

The model must roll an additional D6 when taking \hyperref[dangerous_terrain]{Dangerous Terrain Tests}. A model with Chariot can only be part of a unit consisting entirely of models with Chariot, unless noted otherwise.

\subsubsection{Commanding Presence}
\label{commanding_presence}

All \hyperref[the_general]{Generals} have the Commanding Presence Universal Rule. All units within \distance{12} of a friendly non-Fleeing model with Commanding Presence may \textbf{borrow} the Discipline of the model with Commanding Presence, instead of using their own Discipline (this ability follows all the normal rules for using a \hyperref[using_borrowed_characteristics]{Borrowed Characteristic}, meaning that effects modifying the Discipline of the model with Commanding Presence are applied before borrowing the model's Discipline; this borrowed Discipline may then be further modified).

\subsubsection{Engineer (X+)}
\label{engineer}

Once per Shooting Phase, an unengaged Engineer may select a single \hyperref[war_machine]{War Machine} within \distance{6} that has not fired yet to gain the following effects:

\begin{itemize}[label={-}]
\item Replace the Aim of one of the War Machine's Artillery Weapons with the value given in brackets (X+).
\item You may reroll the roll on the \hyperref[the_misfire_table]{Misfire Table}.
\item You may reroll the dice (all of them or none) for determining the number of hits of a \hyperref[flamethrower]{Flamethrower} Artillery Weapon.
\end{itemize}

\subsubsection{Fear}
\label{fear}

Models in units in base contact with one or more enemy models with Fear suffer -1 Discipline. At the start of each Round of Combat, such units must take a \hyperref[performing_a_discipline_test]{Discipline Test}, called a Fear Test. If this test is failed, the models in the unit are \hyperref[shaken]{Shaken} and Close Combat Attacks made by the models in the unit suffer -1 to hit, while Close Combat Attacks allocated \rewordedrule{towards} the models in the unit gain +1 to hit. These effects apply until the end of the Round of Combat. Models that have Fear themselves are immune to the effects of Fear.

\subsubsection{Fearless}
\label{fearless}

If more than half of a unit's models are Fearless, the unit automatically passes \hyperref[panic_test]{Panic Tests} and cannot declare a Flee Charge Reaction (unless already Fleeing). Models that are Fearless are also immune to the effects of \hyperref[fear]{Fear}.

\subsubsection{Feigned Flight}
\label{feigned_flight}

Models in a unit consisting solely of models with Feigned Flight do not become \hyperref[shaken]{Shaken} if their unit voluntarily chooses Flee as Charge Reaction and passes its \hyperref[rally_fleeing_units]{Rally Test} in its next Player Turn. The \hyperref[reform]{Reform} after Rallying in this case does not prevent the unit from moving nor from shooting (but it still counts as having moved). This rule does not apply if a unit fails to rally on the next friendly Player Turn or involuntarily Flees (e.g. as a result of a failed \hyperref[panic_test]{Panic Test} or if it was already Fleeing when being charged).

\subsubsection{Fly (X, Y)}
\label{fly}

The model gains \hyperref[light_troops]{\textbf{Light Troops}} and \hyperref[swiftstride]{\textbf{Swiftstride}}. Units composed entirely of models with Fly may use Flying Movement during \hyperref[move_chargers]{Move Chargers} moves, \hyperref[advance_move]{Advance Moves}, and \hyperref[march_move]{March Moves}. When a unit uses Flying Movement, substitute its models' Advance Rate with the first value given in brackets (X), and their March Rate with the second value given in brackets (Y). A unit using Flying Movement ignores all Terrain Features and units during the Flying Movement. Note that:

\begin{itemize}[label={-}]
\item It must abide by the \hyperref[unit_spacing]{Unit Spacing} rule at the end of the move (unless charging, when the normal exceptions to the Unit Spacing rule apply).
\item It is affected by the Terrain Features from which it takes off and in which it lands.
\item All modifiers to ground movement values also apply to a model's Fly values (unless specified otherwise).
\item When Declaring a Charge with a unit with Fly, you must declare if the unit will use Flying Movement for the Charge Move.
\end{itemize}

\subsubsection{Frenzy}
\label{frenzy}

The model gains \hyperref[fearless]{\textbf{Fearless}}. At the start of the Charge Phase, each of your units with at least one model with Frenzy that could Declare a Charge against an enemy unit within the unit's Advance Rate +\distance{7} must take a \hyperref[performing_a_discipline_test]{Discipline Test}, called a Frenzy Test. If the test is failed, the whole unit must Declare a Charge this Player Turn if possible. Frenzy Tests and Discipline Tests to restrain from Pursuing taken by units with at least one model with Frenzy are subject to \hyperref[maximised_roll]{Maximised Roll}.

\begin{itemize}[label={-}]
\item When measuring if a unit must take a Frenzy Test, use the lowest available Advance Rate among the unit's models.
\item If the unit has \hyperref[fly]{Fly} and there is more than one Advance Rate available, you must use the type of movement (ground or Fly) that has the highest chance of completing the charge.
\item When a unit is forced to Declare a Charge due to a failed Frenzy Test, it is not forced to charge the enemy unit that triggered the Frenzy Test.
\end{itemize}



\subsubsection{Front Rank}
\label{front_rank}

Front Rank specifies where in a unit the model may be placed and how the model moves inside its unit. A model with Front Rank can either have a Matching Base or a Mismatching Base.

\paragraph{Matching Bases}
\label{matching_bases}

Matching Base refers to one of the following:

\begin{itemize}[label={-}]
\item The model has the same base size as the \rnf{} models in its unit.
\item The model's base is the same size as two or more (whole number) of the \rnf{} models' bases (such as a \num{40x40} \si{\milli\meter} base in a \num{20x20} \si{\milli\meter} unit).
\end{itemize}

A model with Matching Base must always be placed as far forward as possible in its unit. Normally this means that it must be placed in the first rank. If the first rank is occupied by models with Front Rank, it is placed in the second rank instead. If this rank is also occupied by models with Front Rank, it is placed in the third rank, and so on. If the model has a larger base than the \rnf{} models, it is considered to be in all ranks its base occupies for the purposes of calculating \hyperref[full_ranks]{Full Ranks}. For calculating the number of models in the unit's ranks (e.g. for Full Ranks, \hyperref[line_formation]{Line Formation}, \hyperref[area_attack]{Area Attack}, \hyperref[penetrating]{Penetrating}), the large base counts as the number of models it replaces. A model cannot join a unit that has more than one rank if its base is wider than the unit it wishes to join, nor can a unit \hyperref[reform]{Reform} into a formation that is narrower than any model joined to the unit. If a model with a Matching Base has a longer base than the \rnf{} models in the unit, the unit is allowed to have more than one incomplete rank if all incomplete ranks after the first consist entirely of models with such bases (for instance the rear parts of long bases such as War Platforms are allowed to form several incomplete ranks).

When making an Advance Move, March Move, or Reform with a unit that includes models with Front Rank, these models can be reorganised into a new position (still as far forward as possible) as part of the move. This counts towards the distance moved by the unit (measure the distance from the starting position to the ending position of the centre of the model with Front Rank to determine how far it has moved).

If a model with Front Rank leaves a unit or is removed as a casualty, the gap it leaves must be filled with models from other ranks, possibly moving up models with Front Rank, if this means they are moved to positions further forward. If more than one model with Front Rank could move forward, the owner of the models chooses which model to move. If all models with Front Rank already are as far forward as possible, fill any empty gaps with \rnf{} models from the back ranks. Sometimes models with Front Rank must be redistributed in order for all such models to be as far forward as possible. When this happens, move as few models as possible in order to have all models with Front Rank as far forward as possible.

\paragraph{Mismatching Bases}
\label{mismatching_bases}

Anything that is not a Matching Base is a Mismatching Base (such as a \num{50x75} \si{\milli\meter} base inside a \num{25x50} \si{\milli\meter} unit).

A model with Mismatching Base is placed in base contact to the side of the unit, aligned with its front. Only two Mismatching Bases can be joined to a single unit (one at each side). These models are considered to be only in the first rank, but are ignored when counting the number of models in each rank in order to establish the number of \hyperref[full_ranks]{Full Ranks} and whether or not a unit is in \hyperref[line_formation]{Line Formation}. They form a file of one model.

When making an Advance Move, March Move, or Reform with a unit that includes models with Mismatching Bases, these models can be reorganised into a new position (i.e. moved to the other side of the unit) as part of the move. This counts towards the distance moved by the unit (measure the distance from the starting position to the ending position of the centre of the model with Front Rank to determine how far it has moved).

Figure \ref{figure/front_rank} shows how models with Front Rank can be placed in a complex case.

\newcommand{\figFRA}{a)}
\newcommand{\figFRB}{b)}

\begin{figure}[!htbp]
\centering
\def\svgwidth{0.6\textwidth}
\input{pics/front_rank.pdf_tex}
\caption{Illustration of the \frontrank{} rule.\captionpar
Yellow models have Front Rank, green models do not.\captionpar
a) A Character on a Mismatching Base is placed next to the unit. Characters on Matching Bases are placed inside the unit, as far forward as possible. This unit is considered to have 3 Full Ranks.\captionpar
b) When a model with Front Rank joins the unit, the models with Front Rank in the second rank must be moved to the side in order to have all models with Front Rank as far forward as possible.
}
\label{figure/front_rank}
\end{figure}

\subsubsection{Ghost Step}
\label{ghost_step}

The model may choose to treat all Terrain Features as \hyperref[open_terrain]{Open Terrain} for movement purposes, but must abide by the \hyperref[unit_spacing]{Unit Spacing} rule upon the completion of its moves.

\subsubsection{Insignificant}
\label{insignificant}

Units consisting entirely of models with Insignificant do not cause \hyperref[panic_test]{Panic Tests} on friendly units without Insignificant. Only Insignificant Characters can join units with Insignificant \rnf{} models.

\subsubsection{Light Troops}
\label{light_troops}

\rewordedrule{A unit composed entirely of models with Light Troops may \hyperref[reform]{Reform} any number of times during Advance Moves and March Moves. The following rules apply to the movement of Light Troops:}

\begin{itemize}[label={-}]
\item \rewordedrule{No model can end its movement with its centre further away from its starting position than its March Rate.}
\item For measuring the distance travelled by a model, check the path the model would have taken if it was alone and measure the movement around any obstructions (abiding by the Unit Spacing rule). Note that the unit nevertheless must abide by the \hyperref[unit_spacing]{Unit Spacing} rule (including all Reforms).
\item If a model performed any action during the movement (such as a \hyperref[sweeping_attack]{Sweeping Attack}), the distance moved is measured from its starting position to the point on the Battlefield where it performed that action and then to its final position.
\end{itemize}

\rewordedrule{In addition,}
\begin{itemize}[label={-}]
\item \rewordedrule{Units composed entirely of models with Light Troops} may shoot even after March Moving or Reforming earlier that Player Turn.
\item Units with more than half of their models with Light Troops always count as having 0 \hyperref[full_ranks]{Full Ranks}.
\item Infantry Characters gain Light Troops while joined to Infantry units of the same Size with Light Troops.
\end{itemize}

\removedrule{Characters lose Light Troops while joined to units with one or more models without Light Troops (if they had it).}

\subsubsection{Magic Resistance (X)}
\label{magic_resistance}

Learned Spells and Bound Spells that are targeting at least one enemy unit with one or more models with Magic Resistance suffer a -X modifier to their casting roll (where X is given in brackets). This is an exception to the Casting and Dispelling Modifier rule. If there are different X values that could be used, use the highest value.

\subsubsection{Massive Bulk}
\label{massive_bulk}

If the model is mounted by a Character, ignore the rider's Armour Equipment (including Armour Enchantments) and Personal Protections, unless specifically stated otherwise (such as Armour Enchantments that affect the bearer's model). 

\subsubsection{Not a Leader}
\label{not_a_leader}

The model cannot be the \hyperref[the_general]{General}.

\subsubsection{Protean Magic}
\label{protean_magic}

During \hyperref[spell_selection]{Spell Selection}, the Wizard must select its spells between the \hyperref[learned_spells]{Learned Spell} 1 of the Paths it has access to and the \hyperref[hereditary_spells]{Hereditary Spell} of its army. This rule overrides the Spell Selection rules connected to being a Wizard \hyperref[wizard_apprentice]{Apprentice}, \hyperref[wizard_adept]{Adept}, or \hyperref[wizard_master]{Master}.

\subsubsection{Rally Around the Flag}
\label{rally_around_the_flag}

All units within \distance{12} of a friendly non-Fleeing model with Rally Around the Flag may reroll failed \hyperref[performing_a_discipline_test]{Discipline Tests}.


\subsubsection{Random Movement (X)}
\label{random_movement}

At the end of step 2 of the \hyperref[the_movement_phase_sequence]{Movement Phase Sequence} (after Rally Fleeing Units), the unit must move using the rules for \hyperref[pursuit_distance_and_pursuing_units]{Pursuing units}, with the following exceptions (that only apply in the Movement Phase, unless specified otherwise):

\begin{itemize}[label={-}]
\item It \textbf{always} moves the distance stated in brackets (X), which is also used for Flee Distance and Pursuit Distance (including \hyperref[overrun]{Overruns}).
\item It can choose which direction to rotate towards before rolling the Pursuit Distance.
\item It cannot move off the Board Edge.
\item It does not take \hyperref[dangerous_terrain]{Dangerous Terrain Tests} unless charging.
\end{itemize}


There are several restrictions connected with Random Movement:

\begin{itemize}[label={-}]
\item The unit cannot move normally in the Movement Phase (Advance, March, Reform) and cannot Declare Charges in the Charge Phase. Whenever it requires a March Rate (e.g. when \hyperref[post_combat_reform]{Post-Combat Reforming}), use the potential maximum value of X as its March Rate.
\item The unit cannot perform \hyperref[magical_move]{Magical Moves}.
\item The unit loses \hyperref[swiftstride]{Swiftstride} and can never gain it.
\item Characters with Random Movement can only join units with Random Movement (by moving into contact with them during the Movement Phase or by being deployed inside), and units with Random Movement can only be joined by Characters with Random Movement.
\item Units with Random Movement cannot enter \hyperref[buildings]{Buildings}.
\item A unit with Random Movement cannot move in the same phase as it arrives on the Battlefield as \hyperref[ambush]{Ambusher}.
\item If the unit has several instances of Random Movement, use the one with the lowest average (choose in case of a tie).
\end{itemize}

\subsubsection{Scoring}
\label{scoring}

Units with at least one model with Scoring are considered to be Scoring Units, which are used for winning Secondary Objectives (see \totalref{secondary_objectives}). Every army needs Scoring Units to be able to complete Secondary Objectives, which is why units with Scoring are marked in the Army Books with a special pennant icon:

\begin{center}
\textcolor{white}{debug}\includegraphics[width=2.5cm]{../Layout/pics/logo_scoring.png}\textcolor{white}{debug}
\end{center}


Scoring can be lost during the game:
\begin{itemize}[label={-}]
\item A unit that is Fleeing loses Scoring for as long as it if Fleeing. 
\item An \hyperref[ambush]{Ambushing} unit that enters the Battlefield on Game Turn 4 or later loses Scoring.
\item A unit that has performed a \hyperref[post_combat_reform]{Post-Combat Reform} loses Scoring until the start of the following Player Turn.
\item A \hyperref[vanguard]{Vanguarding} unit loses Scoring until the end of Game Turn 1.
\end{itemize}

\subsubsection{Scout}
\label{scout}

Units with Scout may be deployed using Special Deployment rules. All units that will be deployed using the Scout rule must be declared at step 8 of the Pre-Game Sequence (Declare Special Deployment), starting with the player that picked the Deployment Zone. Scout deployment is conducted on Step 5 of the Deployment Phase (Deploy Scout Units). If both players have Scouting units, alternate unit placement (one unit at a time), starting with the player who first completed their normal deployment. Units with Scout have three deployment options:
\begin{itemize}[label={-}]
\item Completely inside your Deployment Zone, using the normal deployment rules.
\item Anywhere on the Battlefield at least \distance{18} away from enemy units.
\item Anywhere on the Battlefield entirely within a Building, Field, Forest, Ruins, or Water Terrain Feature and at least \distance{12} away from enemy units.
\end{itemize}
Scouting units that are placed not completely inside their Deployment Zone may not Declare Charges in the first Player Turn of the first Game Turn (there are no Scout charge restrictions after the first Player Turn).

\subsubsection{Skirmisher}
\label{skirmisher}

The model can always use Shooting Attacks from any rank (models with Skirmisher are not limited to shooting from first and second rank).

Units with at least one \rnf{} model with Skirmisher are formed into a skirmish formation. They are not placed in base contact with each other. Instead, models are placed with a \SI{12.5}{\milli\meter} distance \rewordedrule{(roughly half an inch)} between them. This gap is considered part of the unit for Cover purposes, and will have the same Size as the models in the unit. Other than this gap between models, units with Skirmisher follow the normal rules for forming units and therefore have a Front, two Flank, a Rear Facing, can perform Supporting Attacks, and so on. Units in skirmish formation gain \hyperref[hard_target]{\textbf{Hard Target}} and \hyperref[light_troops]{\textbf{Light Troops}} and never block Line of Sight (remember that this also affects Cover and that they can never contribute to Hard Cover).

Units in skirmish formation can only be joined by Characters that have both the same Type and the same Size as the unit. Unless a Character has the exact same base size as all \rnf{} models in the unit, it is considered Mismatched for the purpose of placement within the unit. The unit ceases to be in skirmish formation when all \rnf{} models with Skirmisher are wiped out: immediately contract their skirmish formation into a normal formation, without moving the centre of the first rank. Nudge any unit as normal to maintain base contact if possible.

Figure \ref{figure/skirmisher} illustrates this rule.

\newcommand{\figSkirmiA}{a)}
\newcommand{\figSkirmiB}{b)}
\newcommand{\figSkirmiDist}{\normalfontsize \SI{12.5}{\milli\meter}}
\newcommand{\figSkirmiCharOne}{$C_{1}$}
\newcommand{\figSkirmiCharTwo}{$C_{2}$}

\begin{figure}[!htbp]
\centering
\def\svgwidth{\textwidth}
\input{pics/skirmisher.pdf_tex}
\caption{Skirmishers.\captionpar
a) An example of a Skirmishing unit with a joined Mismatching Character.\captionpar
b) The same unit in combat. Models with bold frame can attack a Character (either $C_{1}$ or $C_{2}$).\newline Models with dashed frame cannot attack at all.}
\label{figure/skirmisher}
\end{figure}

\subsubsection{Stand Behind}
\label{stand_behind}

The model can be placed anywhere in its unit (it doesn't have to be placed as far forward as possible, even if it has \hyperref[front_rank]{Front Rank}). It cannot be placed further forward inside a unit than any model with Front Rank without Stand Behind. Ignore Stand Behind for models with Mismatching Bases.

\subsubsection{Strider}
\label{strider}

The model automatically passes \hyperref[dangerous_terrain]{Dangerous Terrain} Tests taken due to Terrain. If more than half of a unit's models have Strider, the unit never loses \hyperref[steadfast]{Steadfast} due to Terrain. Sometimes Strider is linked to a specific type of Terrain, stated in brackets. In this case, models with Strider are considered Striders only when interacting with such type of Terrain.

\subsubsection{Stubborn}
\label{stubborn}

A unit with at least one model with Stubborn ignores any Combat Score penalties to its Discipline when taking \hyperref[break_test]{Break Tests} or \hyperref[combat_reform]{Combat Reform} Discipline Tests.

\subsubsection{Supernal}
\label{supernal}

\rewordedrule{The model gains \hyperref[magical_attacks]{\textbf{Magical Attacks}}. In addition, the model gains \hyperref[unstable]{\textbf{Unstable}}, with the following exception: w}hen a unit consisting entirely of models with Supernal loses a combat, the unit must take a \hyperref[break_test]{Break Test} (\hyperref[stubborn]{Stubborn} or \hyperref[steadfast]{Steadfast} units ignore modifiers from Combat Score difference as normal). If the Break Test is passed, ignore all Health Points that would be lost due to Unstable. If the Break Test is failed, follow the rules for Unstable as normal.

\subsubsection{Swiftstride}
\label{swiftstride}

If a unit is composed entirely of models with Swiftstride, its rolls for Charge Range, Flee Distance, Pursuit Distance, and Overrun Distance are subject to \hyperref[maximised_roll]{Maximised Roll}.

\subsubsection{Tall}
\label{tall}

\hyperref[line_of_sight]{Line of Sight} drawn to or from a model with Tall is not blocked by models of the same Size (as the model with Tall), unless the intervening model also has Tall. Remember that this also affects \hyperref[cover]{Cover} (if a model blocks Line of Sight it contributes to Hard Cover, otherwise only to Soft Cover).

\subsubsection{Terror}
\label{terror}

The model gains \hyperref[fear]{\textbf{Fear}} and is immune to the effects of Terror. When a unit with one or more models with Terror Declares a Charge, its target must take a \hyperref[panic_test]{Panic Test}. If the test is failed, the target of the charge must declare a Flee Charge Reaction, if able to do so. 

\subsubsection{Towering Presence}
\label{towering_presence}

The model gains \hyperref[tall]{\textbf{Tall}} and can never be joined or join a unit (unless it is a \hyperref[war_platform]{War Platform}). A model with Towering Presence increases its \hyperref[rally_around_the_flag]{Rally Around the Flag} and \hyperref[commanding_presence]{Commanding Presence} ranges by \distance{6}.

\subsubsection{Unbreakable}
\label{unbreakable}

The model gains \hyperref[fearless]{\textbf{Fearless}} and its unit automatically passes all \hyperref[break_test]{Break Tests}. Characters with Unbreakable can only join units consisting entirely of models with Unbreakable. Units with one or more models with Unbreakable can only be joined by Characters with Unbreakable.

\subsubsection{Undead}
\label{undead}

The model gains \hyperref[unstable]{\textbf{Unstable}}. Models with Undead cannot perform March Moves, unless their unit starts the March Move within the range of a friendly model's \hyperref[commanding_presence]{Commanding Presence}. The only Charge Reaction a unit with one or more models with Undead can make is Hold.

When units consisting entirely of models with Undead lose Health Points due to \hyperref[unstable]{Unstable}, the number of lost Health Points can be reduced in some situations. Apply the modifiers in the following order:
\begin{enumerate}
\item If the unit is \hyperref[stubborn]{Stubborn}, halve the number of lost Health Points (round fractions up).
\item If the unit is \hyperref[steadfast]{Steadfast}, reduce all lost Health Points above 12 to 12.
\item If the unit receives \hyperref[rally_around_the_flag]{Rally Around the Flag}, reduce the number of lost Health Points by the unit's current Rank Bonus. Units with no Rank Bonus reduce the number of Health Points lost by 1 instead.
\item Apply all other modifiers (from Special Equipment, Model Rules, spells, etc.) afterwards.
\end{enumerate}

\subsubsection{Unstable}
\label{unstable}

The model gains \hyperref[fearless]{\textbf{Fearless}}. \rewordedrule{A unit with one or more models with Unstable does not take a \hyperref[break_test]{Break Test} when losing a combat, but instead} it loses one Health Point (without any saves allowed) for each point of Combat Score by which it lost the combat. 

The Health Points losses are distributed in the following order:
\begin{enumerate}
\item \rnf{} models, excluding Champions.
\item Champion.
\item Characters. Distributed by the owner of the unit, as evenly as possible.
\end{enumerate}

Only Characters with Unstable can join units with one or more models with Unstable, and Characters with Unstable cannot join units with models without Unstable​.

\subsubsection{Vanguard}
\label{vanguard}

After Deployment (including units with \hyperref[scout]{Scout}), models with Vanguard may perform a \distance{12} move. The move is performed as a combination of \hyperref[advance_move]{Advance Move} and/or \hyperref[reform]{Reforms}, as in the Movement Phase, including any actions and restrictions the unit would normally have (such as \hyperref[pivots_and_wheels]{Wheeling}, joining units, leaving units, and so on). The \distance{12} distance is used instead of the unit's Advance Rate and March Rate. This move cannot be used to move within \distance{12} of enemy units. This is decreased to \distance{6} for enemy units which have either Scouted or Vanguarded. Units that have moved in this way lose Scoring until the end of Game Turn 1 and may not Declare Charges in the first Player Turn (if their side has the first turn). If both players have units with Vanguard, alternate moving units one at a time, starting with the player that finished deploying last (note that this is an exception to the rules for Simultaneous Effects). Instead of moving a unit, a player may declare to not move any more Vanguarding units.

\subsubsection{War Machine}
\label{war_machine}

The model gains \hyperref[move_or_fire]{\textbf{Move or Fire}}. The model cannot Pursue (which does not prevent it from being affected by Random Movement), Declare Charges, or Declare Flee as Charge Reaction. Characters can never join units with War Machine, and Characters with War Machine cannot join units at all.

When a War Machine fails a \hyperref[panic_test]{Panic Test}, instead of Fleeing it is \hyperref[shaken]{Shaken} until the end of the next Player Turn. War Machines that fail a \hyperref[break_test]{Break Test} are automatically destroyed. War Machines and units Engaged in Combat with them cannot make \hyperref[combat_reform]{Combat Reforms}.

When a unit charges a War Machine, it can move into base contact by having its Front Facing contact any point of the War Machine's base (it must still maximise the number of models in base contact, see \totalref{contact_between_objects} and figure \ref{figure/contact} page \pageref{figure/contact}). No Align Move is allowed. Ignore the War Machine's Facing, as it does not have any due to its round base.

\newpage
\subsubsection{War Platform}
\label{war_platform}

Unless selected as a mount for a Character, a model with War Platform gains \hyperref[not_a_leader]{\textbf{Not a Leader}} and \hyperref[characters]{\textbf{Character}}, with the following exceptions:

\begin{itemize}[label={-}]
\item It does not count towards the Characters Category (for Army List creation).
\item It does not count as Character when Deploying Units (it may still be deployed inside units).
\item It cannot \hyperref[issuing_a_duel]{Issue Duels}, \hyperref[accepting_and_refusing_a_duel]{Accept Duels}, or \hyperref[make_way]{Make Way}.
\item It can perform \hyperref[swirling_melee]{Swirling Melee}.
\item It does not count as Character regarding \hyperref[bodyguard]{Bodyguard} and \hyperref[multiple_wounds]{Multiple Wounds}, unless the War Platform is specifically mentioned in the Bodyguard rule.
  \end{itemize}
  
The model can join units even if it has \hyperref[towering_presence]{Towering Presence}, and having \hyperref[chariot]{Chariot} does not prevent it from joining units without Chariot. Additionally, it does not prevent Characters without Chariot from joining a unit containing a model with War Platform and Chariot. When joined to a unit, it must always be placed in the centre of the first rank, possibly pushing back other models with \hyperref[front_rank]{Front Rank}, and must keep its position in the centre of the first rank at all times (as long as it is joined to the unit). If two positions are equally central (e.g. in a unit with an even number of models in the first rank and a \hyperref[war_platform]{War Platform} replacing an uneven number of models per rank), the War Platform can be placed in either of these positions. If the War Platform cannot be placed in the centre of the the first rank for any reason, the model cannot join the unit. A War Platform can never join a unit with Mismatching Bases and only a single War Platform can be in the same unit unless noted otherwise.

\subsubsection{Wizard Apprentice}
\label{wizard_apprentice}

The Wizard selects its spells as described in \hyperref[spell_selection]{Spell Selection}.

\subsubsection{Wizard Adept}
\label{wizard_adept}

The Wizard gains \hyperref[channel]{\textbf{Channel (1)}} and selects its spells as described in \hyperref[spell_selection]{Spell Selection}.

\subsubsection{Wizard Master}
\label{wizard_master}

The Wizard gains \hyperref[channel]{\textbf{Channel (1)}} and a +1 modifier to its casting rolls, and selects its spells as described in \hyperref[spell_selection]{Spell Selection}.

\subsubsection{Wizard Conclave}
\label{wizard_conclave}

The Champion of a unit with Wizard Conclave gains +1 Health Point in addition to the normal Characteristics increases associated with being a \hyperref[champion]{Champion}, and is a \hyperref[wizard_adept]{\textbf{Wizard Adept}}. This Champion may select up to two spells from predetermined spells given in brackets after Wizard Conclave. This overrides the \hyperref[spell_selection]{Spell Selection} rules connected to being a Wizard Adept.

\section{Characters}
\label{characters}

Character is a special type of Universal Rule. Unless specifically noted otherwise, any model bought as part of the Characters Category of an Army Book has the Character Universal Rule. A model with this rule is referred to as a Character.

All Characters gain the \hyperref[front_rank]{\textbf{Front Rank}} Universal Rule.

\subsubsection{Lone Characters}

All Characters can operate as a unit consisting of just a single model. In this case, follow the normal rules for units.

\subsubsection{Characters Joined to Units}

Characters can operate as part of other units, by joining them. This can be done either by deploying the Character in the unit or by moving into contact with the unit during the Movement Phase. Units that are Engaged in Combat or Fleeing cannot be joined.

Characters can join other Characters to form a unit consisting only of Characters.

Units that are formed by Characters joining \rnf{} models or other Characters are referred to as combined units.

When a Character joins a unit, it must move into a legal position (see \totalref{front_rank}). A Character can freely choose any legal position it can reach with its move, moving through the unit it joins, possibly replacing other models (including models with Front Rank). Move displaced models as little as possible in order to keep all models in legal positions. If the Character does not have a sufficient move to reach its desired position, it cannot join the unit.

When a Character joins a unit with just a single rank, the owner can choose to either displace a model to the second rank, or to expand the unit's width and place the displaced model at either side of the first rank.

If a unit is joined by a Character, the unit cannot move any further in the same Movement Phase. For determining which model counts as having moved or Marched (for purposes of shooting etc.), the Character and the unit are treated individually during the Player Turn in which the Character joined the unit. For example, if the unit hasn't moved and the Character has Marched into the unit to join in, the Character counts as having Marched, the rest of the unit counts as not having moved at all.

Once joined to a unit, the Character is considered as part of the unit for all rules purposes.

\subsubsection{\rnf{} Models in a Combined Unit Wiped out}

If a combined unit has all its \rnf{} models killed, leaving one or more Characters behind, the remaining Characters will stay a combined unit, which is still considered to be the same unit for ongoing effects (such as \hyperref[lasts_one_turn]{Lasts one Turn} spells) and \hyperref[panic_test]{Panic} (the unit has not been destroyed; the Characters in this combined unit may require a Panic Test due to suffering \SI{25}{\percent} or more casualties). They are treated as a new unit for \hyperref[rally_fleeing_units]{Rally Tests} (i.e. Characters that were part of combined Fleeing units at \SI{25}{\percent} or less of their starting number of Health Points take Rally Tests on their normal Discipline).

\subsubsection{Leaving a Combined Unit}

A Character can leave a combined unit in the Charge Phase and in the Movement Phase if it is able to move (i.e. if it isn't Engaged in Combat, hasn't already moved, isn't Fleeing, etc.). In both cases, any game effects that would affect the combined unit (such as Banner Enchantments) remain in effect for the Character until the end of the phase unless noted otherwise (e.g. Lasts one Turn spells). The Character ignores models from the combined unit for movement purposes and may make a Flying Movement (if it has \hyperref[fly]{Fly}). Characters leaving a unit do not affect whether or not this unit counts as having moved (e.g. for purposes of shooting).

\subsubsection{Charging out of a Unit}

Declare a Charge with a Character in a combined unit (during the Charge Phase as normal) and apply the following rules:

\begin{itemize}[label={-}]
\item Use the Character's model for determining Line of Sight and distance to the enemy unit.
\item As soon as the Character Declares a Charge, it is considered a separate unit (i.e. it uses its own Advance Rate, all hits from Stand and Shoot Charge Reactions will hit the Character, in case of a Flee Charge Reaction the enemy unit flees away from the Character, etc.).
\item Ignore the unit the Character was part of when determining Line of Sight and cover for Stand and Shoot Charge Reactions.
\item The unit itself (including other Characters in the unit) cannot Declare Charges in the same Player Turn.
\item If the charge is successful, move the Character out of the unit and charge as normal.
\item If the charge is not successful, the Character makes a \hyperref[failed_charge]{Failed Charge Move} out of the unit. If the Failed Charge Move is too short to place the Character outside \distance{1} of the combined unit, the Character is no longer considered a separate unit and remains in the combined unit. All the models in the combined unit are \hyperref[shaken]{Shaken} until the end of the Player Turn.
\end{itemize}



\subsubsection{Advance/March Moving out of a Combined Unit}

A Character counts as part of the unit until it has physically left it. If a Character does not have enough movement to be placed at least \distance{1} away from the unit, it cannot leave the unit. A Character cannot leave a unit and rejoin it in the same phase. If one or more Characters want to March Move out of a combined unit, apply the following rules:

\begin{itemize}[label={-}]
\item If you need to take a March Test, before moving any models, declare which models the test is taken for.
\item A single March Test can be taken for either one Character, several Characters, or the entire unit (all Characters and \rnf{} models).
\item If the test was taken for Characters only, and all Characters that took the test leave the unit, the remaining models in the unit are free to not perform a March Move. If they wish to March however, they have to take a new March Test.
\item If any Character that took the test elects to remain in the unit, the entire unit must March Move, and must perform a March Test (if it hasn't already).
\end{itemize}

\subsubsection{Distributing Hits \rewordedrule{onto} Combined Units}

When an non-Close Combat Attack hits a combined unit, there are two possibilities for distributing hits:

\vspace*{10pt}
\begin{minipage}[t]{0.42\textwidth}
  \begin{center}
    \textbf{%
    Characters are of the same Type\\%
    -and-\\%
    same Size\\%
    -and-\\%
    There are 5 or more \rnf{} models in the unit%
    }
  \end{center}
  
  \vspace*{10pt}
  All hits are distributed onto the \rnf{} \hyperref[health_pools]{Health Pool}, Characters cannot suffer any hits.
\end{minipage}%
\hfill%
\begin{minipage}[t]{0.42\textwidth}
  \begin{center}
    \textbf{%
      Characters are of a different Type\\%
      - or-\\%
      different Size\\%
      -or-\\%
      There are 4 or less \rnf{} models in the unit%
    }
  \end{center}
   \vspace*{10pt} 
  The player making the attack distributes hits onto the \rnf{} \hyperref[health_pools]{Health Pool} and Characters. All simultaneous hits must be distributed as equally as possible, meaning that no model can take a second hit until all models have taken a single hit, and so on.
\end{minipage}

  \vspace*{5pt}
If a unit of 5 or more \rnf{} models contains several Characters of both the same and different Type or Size, Characters with the same Type and Size as the \rnf{} models are ignored for the hit distributions. Note that hits are never distributed onto Champions.

\subsubsection{Make Way}
\label{make_way}

At step 3 of the \hyperref[round_of_combat_sequence]{Round of Combat Sequence}, any Character placed in the first rank and not in base contact with an enemy model may move into contact with an enemy model. This enemy model must be in base contact with the Character's unit, and it must be attacking the Character's unit in its Front Facing. To perform a Make Way move, the Character switches position with another model (or models) in its unit; these models cannot be Characters. Characters with Mismatching Bases can never perform a Make Way move.

\section{Personal Protections}
\label{personal_protections}

If at least one model part has a Personal Protection, the entire combined model follows the rules of the Personal Protection, unless the model's mount is of Gigantic Size (and therefore has the Universal Rule \hyperref[massive_bulk]{Massive Bulk}). In this case, only the mount's Personal Protections are applied. For example, if a Character with Distracting mounts a horse (Standard Size), the combined model is affected by Distracting. If the Character instead mounts a dragon (Gigantic Size), the combined model is not affected by Distracting.

\subsection{Conditional Application}
\label{PP_conditional_application}

Personal Protections may only work against certain attacks, which are then stated in brackets after \enquote{against}. There may already be some piece of information relative to the rule specified between brackets, as in Aegis (4+): then the conditions for the rule to work are written in the same brackets, after a comma. This can e.g. be certain kinds of attacks or attacks with a given Attack Attribute (Aegis (4+, against Melee Attacks) or Aegis (2+, against Flaming Attacks)).

\subsubsection{Aegis (X)}
\label{aegis}

Aegis is a \hyperref[special_saves]{Special Save}. A model must reroll successful Aegis Saves against attacks with \hyperref[divine_attacks]{Divine Attacks}.

\subsubsection{Cannot be Stomped}
\label{cannot_be_stomped}

For the purposes of \hyperref[stomp_attacks]{Stomp Attacks} from enemy model, a model with Cannot be Stomped is never considered of Standard Size.

\subsubsection{Distracting}
\label{distracting}

Close Combat Attacks allocated towards a model with Distracting suffer a -1 to-hit modifier. This to-hit modifier cannot be combined with any other negative to-hit modifiers. 



\subsubsection{Flammable}
\label{flammable}

Attacks with the \hyperref[flaming_attacks]{Flaming Attacks} Attack Attribute must reroll failed to-wound rolls against a model with Flammable. 

\subsubsection{Hard Target}
\label{hard_target}

Shooting Attacks targeting a unit that has more than half of its models with Hard Target suffer a -1 to-hit modifier. This rule is cumulative, allowing an additional -1 to-hit modifier for each instance of Hard Target.



\subsubsection{Parry}
\label{parry}

Parry can only be used against \hyperref[melee_attacks]{Close Combat Attacks} from the Front Facing. The model gains +1 Defensive Skill, or its Defensive Skill is \textbf{always} equal to the Offensive Skill of the attacker, whichever is higher.

\subsubsection{Fortitude (X)}
\label{fortitude}

Fortitude is a \hyperref[special_saves]{Special Save}. Fortitude Saves cannot be taken against attacks with \hyperref[lethal_strike]{Lethal Strike} that rolled a natural 6+ to wound, or against attacks with \hyperref[flaming_attacks]{Flaming Attacks}.

\section{Armour Equipment}
\label{armour_equipment}

There are 2 different types of Armour Equipment. A model can only ever be equipped with one piece of armour of each type (i.e. an optional Suit of Armour replaces a model's default Suit of Armour if applicable). The types of armour below are also referred to as mundane armour:

\subsection{Suits of Armour}

\begin{itemize}[label={-}]
\item Light Armour: +1 Armour.
\item Heavy Armour: +2 Armour.
\item Plate Armour: +3 Armour.
\end{itemize}

\subsection{Shields}

\begin{itemize}[label={-}]
\item Shield: +1 Armour.
\end{itemize}

Several weapons prevent simultaneously using a Shield against Melee Attacks (see \totalref{weapons}).





\newpage
\section{Weapons}
\label{weapons}

Weapons are divided into three categories: Close Combat Weapons, Shooting Weapons, and Artillery Weapons. The weapons listed below are also referred to as mundane weapons.

\subsection{Close Combat Weapons}
\label{close_combat_weapons}

Close Combat Weapons are used in Close Combat and can confer various benefits and drawbacks to the model's Close Combat Attacks. The rules for a Close Combat Weapon are only applied when using the weapon in question (i.e. they don't apply to \hyperref[special_attacks]{Special Attacks}, such as Stomp Attacks, or when using a different weapon).

\subsubsection{Choosing a Close Combat Weapon}

If a model has more than one Close Combat Weapon, it must choose which one to use in the first Round of Combat, at step 2 of the \hyperref[round_of_combat_sequence]{Round of Combat Sequence}. It must then continue to use the same weapon for the duration of that combat. All \rnf{} models in a unit must \textbf{always} choose the same Close Combat Weapon, unless they are forced to use enchanted weapons.

\begin{center}
\renewcommand*{\arraystretch}{2}
\begin{tabular}{>{\raggedleft}p{3cm} p{12.5cm}}
  \toprule
  Weapon & Rules \\
    \textbf{\gw} & Attacks made with a Great Weapon gain +2 Strength, +2 Armour Penetration, and  strike at \hyperref[initiative_order]{Initiative Step} 0 (regardless of the wielder's Agility). A model using this weapon cannot simultaneously use a Shield against Melee Attacks. \\
  \textbf{\hw} & All models come equipped with a Hand Weapon as their default equipment. If a model has any Close Combat Weapon other than a Hand Weapon, it cannot choose to use the Hand Weapon (unless specifically stated). Hand Weapons wielded by models on foot can be used alongside a Shield, then giving \hyperref[parry]{\textbf{Parry}}.\\
  \textbf{\halberd} & Attacks made with a Halberd gain +1 Strength and +1 Armour Penetration. A model using this weapon cannot simultaneously use a Shield against Melee Attacks. \\
    \textbf{\lance} & Attacks made with a Lance and allocated towards models in the wielders' Front Facing gain \hyperref[devastating_charge]{\textbf{Devastating Charge (+2 Strength, +2 Armour Penetration)}}. Infantry cannot use Lances.\\
  \textbf{Light Lance} & Attacks made with a Light Lance and allocated towards models in the wielders' Front Facing gain \hyperref[devastating_charge]{\textbf{Devastating Charge (+1 Strength, +1 Armour Penetration)}}. Infantry cannot use Light Lances.\\
  \textbf{\pw} & The wielder gains +1 Attack Value when using Paired Weapons. Attacks made with Paired Weapons gain +1 Offensive Skill and ignore \hyperref[parry]{Parry}. A model using this weapon cannot simultaneously use a Shield against Melee Attacks (while Paired Weapons are often modelled as two Hand Weapons, they are considered a separate weapon category for rules purposes).\\
  \textbf{\spear} & Attacks made with a Spear gain \hyperref[fight_in_extra_rank]{\textbf{Fight in Extra Rank}} and +1 Armour Penetration. Close Combat Attacks from model parts wielding a Spear gain +2 Agility and an additional +1 Armour Penetration in the first Round of Combat provided their unit is not charging and is not Engaged either in their Flank or Rear Facing. Cavalry, Beasts, and Constructs cannot use Spears.\\
  \bottomrule
\end{tabular}
\end{center}

\clearpage
\subsection{Shooting Weapons}
\label{shooting_weapons}

Shooting Weapons are used for making Shooting Attacks. Each model part can normally only use one Shooting Weapon per phase even if it is equipped with more than one, and all non-Champion \rnf{} models in a unit must use the same Shooting Weapon. Each Shooting Weapon has a maximum range, a Strength, and an Armour Penetration value, and can have one or more \hyperref[attack_attributes]{Attack Attributes}. Attack Attributes listed for a Shooting Weapon only apply to the Shooting Attacks made with that weapon.

\begin{center}
\renewcommand*{\arraystretch}{2}
\begin{tabular}{r c c c c l}
  \toprule
  Weapon & Range & Shots & Strength & Armour Penetration & Attack Attributes \\
  \textbf{Bow} & \distance{24} & 1 & 3 & 0 & \hyperref[volley_fire]{Volley Fire} \\
  \textbf{Longbow} & \distance{30} & 1 & 3 & 0 & \hyperref[volley_fire]{Volley Fire} \\
  \textbf{Crossbow} & \distance{30} & 1 & 4 & 1 & \hyperref[unwieldy]{Unwieldy} \\
  \textbf{Handgun} & \distance{24} & 1 & 4 & 2 & \hyperref[unwieldy]{Unwieldy} \\
  \textbf{Pistol} & \distance{12} & 1 & 4 & 2 & \hyperref[quick_to_fire]{Quick to Fire} \\
  \textbf{Throwing Weapons} & \distance{8} & 2 & Same as user & Same as user & \hyperref[accurate]{Accurate}, \hyperref[quick_to_fire]{Quick to Fire}  \\
  \bottomrule
\end{tabular}
\end{center}

\subsection{Artillery Weapons}
\label{artillery_weapons}

Artillery Weapons are a special kind of Shooting Weapons. These weapons are often installed on \hyperref[war_machine]{War Machines}, but can on other occasions be carried by \hyperref[chariot]{Chariots} or Gigantic Beasts, or contained within \hyperref[special_equipment]{Special Equipment}. Artillery Weapons are Shooting Weapons that always have \hyperref[reload]{\textbf{Reload!}}. Artillery Weapons have specific profiles for Range, Shots, Strength, Armour Penetration, and Attack Attribute rules, which you will find in their description. Some Artillery Weapons may have further rules detailed as follows.

\subsubsection{Bolt Thrower}
\label{bolt_thrower}

Bolt Thrower attacks have \hyperref[penetrating]{\textbf{Penetrating}}.

\subsubsection{Cannon}
\label{cannon}

Cannon attacks have \hyperref[penetrating]{\textbf{Penetrating}} and ignore to-hit modifiers from Soft Cover and Hard Cover. They gain a +1 to-hit modifier when targeting units consisting entirely of models of Gigantic Size that do not benefit from Cover. On a natural to-hit roll of \result{1} a Cannon Misfires: roll on the \hyperref[the_misfire_table]{Misfire Table} (table \ref{table/misfire} page \pageref{table/misfire}) and apply the corresponding result (a to-hit roll resulting in a Misfire cannot be rerolled).

\subsubsection{Catapult (X)}
\label{catapult}

Catapult attacks ignore to-hit modifiers from Soft Cover and Hard Cover. Resolve Catapult attacks as follows:

\begin{itemize}[label={-}]
\item On a natural to-hit roll of \result{1}, it Misfires: roll on the \hyperref[the_misfire_table]{Misfire Table} (table \ref{table/misfire} page \pageref{table/misfire}) and apply the corresponding result (a to-hit roll resulting in a Misfire cannot be rerolled).
\item On a successful to-hit roll, the attack gains \hyperref[area_attack]{Area Attack (X)}. Resolve the attack with the Strength and Armour Penetration written in the Catapult's description.
\item On any other to-hit result, roll to hit with a new Catapult attack, labelled Partial Hit (ignore any Misfire). If it hits, this attack gains \hyperref[area_attack]{Area Attack (X-1)}, all hits are at half Strength and half Armour Penetration (rounding fractions up), it loses all benefits from the Strength and Armour Penetration written in square brackets and/or Attack Attributes written in square brackets (if any; see \hyperref[area_attack]{Area Attack}). If it misses, no further attack can be generated this way.
\end{itemize}

\subsubsection{Flamethrower}
\label{flamethrower}

Flamethrowers do not roll to hit. Instead, roll a D6 (this is not considered a to-hit roll). On a natural roll of \result{1}, it Misfires: roll on the \hyperref[the_misfire_table]{Misfire Table} (table \ref{table/misfire} page \pageref{table/misfire}) with a -1 modifier and apply the corresponding result. On any other natural result the attack is successful. Check in which Arc of the target half or more of the attacker's base is in (randomise in case of a tie).
\begin{itemize}[label={-}]
\item If the attacker is in the Front or the Rear Arc, the attack causes D6 hits, +D3 hits for each rank after the first up to a maximum of +4D3.
\item If the attacker is in either Flank Arc, the attack causes a D6 hits, +D3 hits for each file after the first up to a maximum of +4D3.
\end{itemize}
The total number of hits cannot exceed the number of models in the unit.

Some Flamethrowers have a higher Strength, Armour Penetration, and/or additional Attack Attributes stated in curly brackets (such as Strength 4 \{5\}, Armour Penetration 1 \{2\}, \{\hyperref[multiple_wounds]{Multiple Wounds (D3)}\}). If so, use the Strength, Armour Penetration, and Attack Attributes inside such brackets when shooting at a target within Short Range.

\subsubsection{Volley Gun}
\label{volley_gun}

The number of shots fired by a Volley Gun is a random number. When rolling for the number of shots for a Volley Gun attack, if a single natural \result{6} is rolled (after any reroll), this attack suffers a -1 to-hit modifier; instead, if two or more natural \result{6} are rolled, the attack fails and the Volley Gun Misfires: roll on the \hyperref[the_misfire_table]{Misfire Table} (table \ref{table/misfire} page \pageref{table/misfire}) and apply the corresponding result.

\subsection{The Misfire Table}
\label{the_misfire_table}

A to-hit roll resulting in a Misfire cannot be rerolled. When an Artillery Weapon misfires, roll a D6 and consult the table \ref{table/misfire} below (a result of \result{0} or less may happen when there is a negative modifier to the roll, as for the \hyperref[flamethrower]{Flamethrower}).

\vspace*{10pt}
\begin{table}[!htbp]
\centering
\begin{tabular}{M{2cm}m{12cm}}
\textbf{Dice Result} & \centering\textbf{Misfire Effect} \tabularnewline
\midrule
\textbf{0\newline (or less)} & \textbf{Explosion!}\vspace*{3pt}\newline 
All models within \distance{D6} of the misfiring model suffer a hit with Strength 5 and Armour Penetration 2. The shooting model is then destroyed. Remove it as a casualty. \tabularnewline
\textbf{1-2} & \textbf{Breakdown}\vspace*{3pt}\newline 
The model cannot shoot anymore with the weapon for the rest of the game. \tabularnewline
\textbf{3-4} & \textbf{Jammed}\vspace*{3pt}\newline
The Artillery Weapon may not shoot in the controlling player's next Player Turn. If the model is a \hyperref[war_machine]{War Machine}, instead the model is \hyperref[shaken]{Shaken} until the end of the controlling player's next Player Turn.\tabularnewline
\textbf{5+} & \textbf{Malfunction}\vspace*{3pt}\newline
The shooting model loses a Health Point with no saves of any kind allowed. \tabularnewline
\bottomrule
\end{tabular}
\caption{Misfire Effects.}
\label{table/misfire}
\end{table}

\clearpage
\section{Attack Attributes}
\label{attack_attributes}

Attack Attributes can either be given to a model part, to a weapon, to a \hyperref[special_attacks]{Special Attack}, or to a spell. Attack Attributes are divided into three sub-categories that define what attacks they affect.

\subsubsection{Close Combat}

A model part with an Attack Attribute with this keyword applies the rules of the Attack Attribute to all its Close Combat Attacks.

\subsubsection{Shooting}

A model part with an Attack Attribute with this keyword applies the rules of the Attack Attribute to all its Shooting Attacks that are not \hyperref[special_attacks]{Special Attacks}.

\subsubsection{Attacks \&{} Weapons}

Attack Attributes with these keywords that are given to a \hyperref[special_attacks]{Special Attack}, a spell, or a weapon are always applied to the attacks coming from that Special Attack, spell, or weapon. The rule is not applied to any other attacks made by a model with such a Special Attack, spell, or weapon.

Note that an Attack Attribute with Attacks \&{} Weapons and Close Combat keywords could be applied to a Shooting Weapon.

\subsection{Conditional Application}
\label{conditional_application}

Attack Attributes may only work against certain enemies, which are then stated in brackets after \enquote{against}. There may already be some piece of information relative to the rule specified between brackets, as in \hyperref[multiple_wounds]{Multiple Wounds~(2)}: then the conditions for the rule to work are written in the same brackets, after a comma. This can e.g. be all models from a given Army Book, with a given Model Rule, of a given Size, or of a given Type. If the Attack Attribute is effective against more than one type of enemy, they are separated by commas. If no comma but instead \enquote{and} is used, this means that the rule works only against enemies that fulfil all criteria. For example, Multiple Wounds~(2, against Large and Beasts, Gigantic) means that Multiple Wounds can be used against models that are both Large and Beasts, as well as against models that are Gigantic.

Attack Attributes with Conditional Application can only be applied when the affected attacks are either allocated \rewordedrule{towards} or distributed \rewordedrule{onto} a Health Pool where all models fulfil the requirements.

\subsubsection{Accurate - Attacks \&{} Weapons, Shooting}
\label{accurate}

The attack doesn't suffer the -1 to-hit modifier for shooting at \hyperref[long_range]{Long Range}.

\subsubsection{Area Attack (X) - Attacks \&{} Weapons}
\label{area_attack}

When the attack hits a unit, choose up to X different ranks of this unit; these must be the ranks resulting in the maximum amount of hits. For each rank selected this way: the unit suffers X hits, to a maximum equal to the number of models in this rank. A single Area Attack can never cause more hits than there are models in the unit.

Some Area Attacks have a higher Strength and/or additional Attack Attributes stated in square brackets (such as Strength 3 [7], [\hyperref[multiple_wounds]{Multiple Wounds (D3)}]). If so, a single hit from this attack, chosen by the attacker, uses the Strength value and Attack Attributes in brackets. The bracketed values and Attack Attributes are not applied to any other hits.

Figure \ref{figure/area_attack} illustrates a few examples of Area Attacks.

\newcommand{\figAAOneHit}{\normalfontsize{1 Hit}}
\newcommand{\figAATwoHits}{\normalfontsize{2 Hits}}
\newcommand{\figAAThreeHits}{\normalfontsize{3 Hits}}
\newcommand{\figAATotalOneHit}{Total: 1 Hit}
\newcommand{\figAATotalFourHits}{Total: 4 Hits}
\newcommand{\figAATotalSixHits}{Total: 6 Hits}
\newcommand{\figAAAreaAttack}[1]{\areaattack{#1}}

\begin{figure}[H]
\begin{minipage}{0.55\textwidth}
\centering
\def\svgwidth{\textwidth}
\input{pics/area_attack.pdf_tex}
\end{minipage}\hfill\begin{minipage}{0.42\textwidth}
\caption{A few examples of Area Attacks.\captionpar
Green unit is hit by an \areaattack{2}. There are more than two ranks with more than 2 models.\newline
Number of hits: $ 2 + 2 = 4 $.\captionpar
Orange unit is hit by an \areaattack{3}. First rank has more than 3 models, but the second rank only has a single model. There is no third rank.\newline
Number of hits: $ 3 + 1 = 4 $.\captionpar
Purple unit is hit by an \areaattack{4}. Since it's a single model, there will always only be a single rank with a single model.\newline
Number of hits: $ 1 $.\captionpar
Blue unit is hit by an \areaattack{4}. First and second rank have 3 models. There is no third or fourth rank.\newline
Number of hits: $ 3 + 3 = 6 $.%
}
\label{figure/area_attack}
\end{minipage}
\end{figure}

\subsubsection{Battle Focus - Attacks \&{} Weapons, Close Combat}
\label{battle_focus}

If the attack hits with a natural to-hit roll of \result{6}, the attack causes two hits instead of one.

\subsubsection{Crush Attack - Attacks \&{} Weapons, Close Combat}
\label{crush_attack}

At the end of step 4 of the \hyperref[round_of_combat_sequence]{Round of Combat Sequence} (just after Issue and Accept Duels), the model part may announce that it will use its Crush Attack this Round of Combat. It performs a single Close Combat Attack, which cannot be made as a \hyperref[supporting_attacks]{Supporting Attack}, is resolved at \hyperref[initiative_order]{Initiative Step} 0, has Strength 10, Armour Penetration 10 (regardless of user's Agility, Strength, and Armour Penetration), and \hyperref[multiple_wounds]{Multiple Wounds (D3+1)}. Crush Attacks never benefit from any weapons or other Attack Attributes the model part may have. The model part cannot make any other Close Combat Attacks during this Round of Combat (but can still use its Special Attacks such as Stomp Attacks or Impact Hits).

\subsubsection{Devastating Charge (X) - Attacks \&{} Weapons, Close Combat}
\label{devastating_charge}

A charging model part with Devastating Charge, or using a weapon with Devastating Charge, gains the Model Rules and Characteristics modifiers stated in brackets. For example, a charging model part with Devastating Charge (+1 Strength, \hyperref[poison_attacks]{Poison Attacks}) gains +1 Strength and Poison Attacks when it is charging. This rule is cumulative: a model part with several instances of Devastating Charge applies all Attack Attributes and Characteristics modifiers from all of them when charging.

\subsubsection{Divine Attacks - Attacks \&{} Weapons, Close Combat}
\label{divine_attacks}

Successful \hyperref[aegis]{Aegis} Saves taken against the attack must be rerolled.

\subsubsection{Fight in Extra Rank - Attacks \&{} Weapons, Close Combat}
\label{fight_in_extra_rank}

Model parts with Fight in Extra Rank, or using a weapon with Fight in Extra Rank, can make \hyperref[supporting_attacks]{Supporting Attacks} from an additional rank (normally, this means that models with Fight in Extra Rank will be able to make Supporting Attacks from the third rank). This rule is cumulative, allowing an additional rank to make Supporting Attacks for each instance of Fight in Extra Rank.

\subsubsection{Flaming Attacks - Attacks \&{} Weapons, Close Combat, Shooting}
\label{flaming_attacks}

The attacks ignores \hyperref[fortitude]{Fortitude} Saves and must reroll failed to-wound rolls against models with \hyperref[flammable]{Flammable}.

\subsubsection{Harnessed - Close Combat}
\label{harnessed}

Model parts with Harnessed cannot make \hyperref[supporting_attacks]{Supporting Attacks} and cannot use weapons. Shooting Weapons carried by model parts with Harnessed can be used by other model parts of the same model (as long as they do not have Harnessed or \hyperref[inanimate]{Inanimate}). A model with at least one model part with Harnessed is considered to be mounted.

\subsubsection{Hatred - Attacks \&{} Weapons, Close Combat}
\label{hatred}

During the first Round of Combat, failed to-hit rolls from attacks with Hatred must be rerolled.

\subsubsection{Inanimate - Close Combat}
\label{inanimate}

Model parts with Inanimate cannot make Close Combat Attacks and cannot use Shooting Weapons. Shooting Weapons carried by model parts with Inanimate can be used by other model parts of the same model (as long as they do not  have \hyperref[harnessed]{Harnessed} or Inanimate).

\subsubsection{Lethal Strike - Attacks \&{} Weapons, Close Combat}
\label{lethal_strike}

An attack with Lethal Strike that wounds with a natural to-wound roll of \result{6} has its Armour Penetration \textbf{set} to 10 and ignores \hyperref[fortitude]{Fortitude} Saves.

\subsubsection{Lightning Reflexes - Attacks \&{} Weapons, Close Combat}
\label{lightning_reflexes}

The attack gains a +1 to-hit modifier if it is a Close Combat Attack. Model parts with this Attack Attribute wielding Great Weapons do not gain this +1 to-hit modifier, but strike with the Great Weapon at the \hyperref[initiative_order]{Initiative Step} corresponding to their normal Agility instead of always striking at Initiative Step 0.

\subsubsection{Magical Attacks - Attacks \&{} Weapons, Close Combat, Shooting}
\label{magical_attacks}

The attack doesn't have any special effects. However, the Attack Attribute interacts with other rules, such as \hyperref[aegis]{Aegis} (X, against Magical Attacks). Model parts with Magical Attacks also apply the Attack Attribute to their \hyperref[special_attacks]{Special Attacks} (such as \hyperref[stomp_attacks]{Stomp Attacks}, \hyperref[impact_hits]{Impact Hits} and \hyperref[breath_attack]{Breath Attack}).

\subsubsection{Move or Fire - Attacks \&{} Weapons, Shooting}
\label{move_or_fire}

The attack may not be used if the attacking model has made an \hyperref[advance_move]{Advance Move}, \hyperref[march_move]{March Move}, \hyperref[reform]{Reform}, or \hyperref[pivots_and_wheels]{Pivot} during the current Player Turn.

\subsubsection{Multiple Wounds (X) - Attacks \&{} Weapons, Close Combat}
\label{multiple_wounds}

Unsaved wounds caused by the attacks are multiplied into the value given in brackets (X). If the value is a dice (e.g. Multiple Wounds (D3)), roll one dice for each unsaved wound with Multiple Wounds. The amount of wounds that the attack is multiplied into can never be higher than the Health Points Characteristic of the target (excluding Health Points lost previously in the battle). For example, if a Multiple Wounds (D6) attack wounds a unit of Trolls (HP 3) and rolls a \result{5} for the multiplier, the number of unsaved wounds is reduced to 3, even if the Troll unit has already lost one or two Health Points previously in battle.

If Clipped Wings is stated after the X value in brackets, any unsaved wound caused by the attack against a model with \hyperref[fly]{Fly} is multiplied into X+1 instead of X.

\subsubsection{Penetrating - Attacks \&{} Weapons}
\label{penetrating}

When the attack hits, check in which Arc of the target half or more of the attacker's base is (randomise in case of a tie). The attack causes a number of hits equal to the number of ranks of its target if the attacker is in the Front or the Rear Arc, or a number hits equal to the number of files of its target if the attacker is in either Flank Arc. In either case, the number of affected ranks or files cannot exceed 5, and no model can suffer more than one hit from a single attack with Penetrating.

Some Penetrating attacks have a higher Strength and/or additional Attack Attributes stated in square brackets (e.g. Strength 3 [6], [\hyperref[multiple_wounds]{Multiple Wounds (D3)}]). If so, a single hit from this attack, chosen by the attacker, uses the Strength value and Attack Attributes in brackets. The bracketed values and Attack Attributes are not applied to any other hits.

\subsubsection{Poison Attacks - Attacks \&{} Weapons, Close Combat, Shooting}
\label{poison_attacks}

If the attack successfully hits with a natural to-hit roll of \result{6}, it automatically wounds with no to-wound roll needed. Shooting Attacks using \hyperref[hopeless_shots]{Hopeless Shot} can only automatically wound if the first to-hit roll is a natural \result{6}. Note that the second to-hit roll must still be successful in order to hit the target. If the attack can be turned into more than one hit (e.g. a hit with \hyperref[area_attack]{Area Attack}, \hyperref[battle_focus]{Battle Focus}, or \hyperref[penetrating]{Penetrating}), only a single hit, chosen by the attacker,  automatically wounds. All other hits must roll to wound as normal. 

\subsubsection{Quick to Fire - Attacks \&{} Weapons, Shooting}
\label{quick_to_fire}

The attack doesn't suffer the -1 to-hit modifier for Moving and Shooting.

\subsubsection{Reload! - Attacks \&{} Weapons, Shooting}
\label{reload}

The attack cannot be used for a Stand and Shoot Charge Reaction.

\subsubsection{Toxic Attacks - Attacks \& Weapons, Close Combat}
\label{toxic_attacks}

The attack has its Strength \textbf{always set} to 3 and its Armour Penetration \textbf{always set} to 10.

\subsubsection{Unwieldy - Attacks \&{} Weapons, Shooting}
\label{unwieldy}

The attack suffers an additional -1 to-hit modifier for Moving and Shooting (for a total of -2). When combined with \hyperref[quick_to_fire]{Quick to Fire}, the attack can only ignore the normal -1 to-hit modifier from Moving and Shooting, not the additional -1 to-hit modifier from Unwieldy.

\subsubsection{Volley Fire - Attacks \&{} Weapons, Shooting}
\label{volley_fire}

If at least one model in a unit can draw Line of Sight to the target, then all model parts using Volley Fire in the same unit ignore all intervening models of their own Size or smaller for Line of Sight and Cover purposes. In addition, unless making a Stand and Shoot Charge Reaction, models in a unit in \hyperref[line_formation]{Line Formation} that has not moved during this Player Turn may shoot from one additional rank (usually this means that they can shoot from the first three ranks).

\subsubsection{Weapon Master - Close Combat}
\label{weapon_master}

At the beginning of each Round of Combat, model parts with Weapon Master may choose which weapon they fight with. This includes selecting to use a Hand Weapon even if they have other weapons. If armed with a weapon with a Weapon Enchantment, the model part must still use it.

\section{Special Attacks}
\label{special_attacks}

A model part with Special Attacks can make a special type of attack specified by the corresponding rules. Attacks made using Special Attacks cannot be affected by weapons or \hyperref[attack_attributes]{Attack Attributes}, unless stated otherwise.

\subsubsection{Breath Attack (X)}
\label{breath_attack}

A model part with Breath Attack can use it only once during the game. If a model has more than one Breath Attack, it can only use one Breath Attack in a single phase. It can be used either as a Shooting Attack or as a Special Attack in Close Combat.
\begin{itemize}[label={-}]
\item As a Shooting Attack (normally in the Shooting Phase): choose a target using the \hyperref[shooting_with_a_unit]{normal rules} for Shooting Attacks (it is allowed for a Stand and Shoot Charge Reaction), except it can be used even if the model Marched previously in this Player Turn. A model with both a Breath Attack and a Shooting Weapon can use both in the same Shooting Phase, however only against the same target. The attack has a range of \distance{6}. 
\item As a Special Attack in Close Combat: the attack is made at the model part's Agility. Declare that you are using the Breath Attack when allocating attacks, and choose a unit in base contact to attack with it.
\end{itemize}  
No matter if it is used as a Shooting or Melee Attack, the target of the Breath Attack suffers 2D6 hits. The Strength, Armour Penetration, and Attack Attributes (if any) of these hits are given within brackets, such as in Breath Attack (Strength 4, Armour Penetration 1, \hyperref[flaming_attacks]{Flaming Attacks}). When several models in the same unit have this Special Attack, roll for the number of hits separately for each model.

\subsubsection{Grind Attacks (X)}
\label{grind_attacks}

A model part with Grind Attacks resolves these attacks at its own Agility. It must choose an enemy unit in base contact with it. The chosen enemy unit suffers a number of hits equal to the value stated in brackets (X). These hits are resolved with the model part's own Strength and Armour Penetration.

If a model has both Grind Attacks and \hyperref[impact_hits]{Impact Hits}, it may only use one of these rules in the same Round of Combat (its controlling player may choose which). When several model parts in the same unit have Grind Attacks and when X is a random number (e.g. Grind Attacks (2D3)), roll for the number of hits separately for each model part.

\subsubsection{Impact Hits (X)}
\label{impact_hits}

At \hyperref[initiative_order]{Initiative Step} 10, a charging model part with Impact Hits must choose an enemy unit that is in base contact with the attacking model's Front Facing. This unit suffers a number of hits equal to the value stated in brackets (X). These hits are resolved with the attacking model part's Strength and Armour Penetration.

If a model has both \hyperref[grind_attacks]{Grind Attacks} and Impact Hits, it may only use one of these rules in the same Round of Combat (its controlling player may choose which). In multipart models, only model parts that also have \hyperref[harnessed]{Harnessed} or \hyperref[inanimate]{Inanimate} can use their Impact Hits. When several models in the same unit have Impact Hits, and when X is a random number (e.g. Impact Hits (D6)), roll for the number of hits separately for each model part.

\subsubsection{Stomp Attacks (X)}
\label{stomp_attacks}

At \hyperref[initiative_order]{Initiative Step} 0, a model part with Stomp Attacks must choose an enemy model of Standard Size in base contact with it. The chosen model's unit suffers a number of hits equal to the value stated in brackets (X). These hits can only be distributed onto models of Standard Size (ignore models of a different Size when distributing hits). They are resolved with the model part's own Strength and Armour Penetration.

In multipart models, only model parts that also have \hyperref[harnessed]{Harnessed} can use Stomp Attacks. When several models in the same unit have this Special Attack, and when X is a random number (e.g. Stomp Attacks (D6)), roll for the number of hits separately.

\subsubsection{Sweeping Attack}
\label{sweeping_attack}

This attack may be used by units containing models with Sweeping Attack. When the unit \hyperref[advance_move]{Advance Moves} or \hyperref[march_move]{March Moves}, you may nominate a single unengaged enemy unit that the unit with Sweeping Attack moved through or over during this move (meaning their bases were overlapping, even partially). The whole unit makes the Sweeping Attack against the nominated enemy unit, which is resolved when the March or Advance Move is completed. Follow the description in the unit profile. These attacks hit automatically and count as ranged Special Attacks. Each Sweeping Attack can only be performed once per Player Turn.

