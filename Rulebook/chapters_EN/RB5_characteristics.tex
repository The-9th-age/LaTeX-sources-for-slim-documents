
\part{Characteristics}
\label{characteristics}

\section{The Characteristics Profiles}
\label{the_characteristics_profiles}

Each unit entry has three (or more) Characteristics Profiles: Global Characteristics, Defensive Characteristics, and Offensive Characteristics.

\subsection{Global Characteristics Profile}

Each model has three Global Characteristics:

\begin{center}
\begin{tabular}{M{0.75cm}M{2.8cm}M{12cm}@{}}
\hline
\textbf{\AdvanceRateInitials} & \AdvanceRate{} & \flufffont{The speed of the model when it Advance Moves, in inches per turn.} \tabularnewline
\textbf{\MarchRateInitials} & \MarchRate{} & \flufffont{The speed of the model when it March Moves, in inches per turn.} \tabularnewline
\textbf{\DisciplineInitials} & \Discipline{} & \flufffont{Shows the model's Discipline and ability to stand and fight.} \tabularnewline
\hline
\end{tabular}
\end{center}

\subsection{Defensive Characteristics Profile}

Each model has four Defensive Characteristics:

\begin{center}
\begin{tabular}{M{0.75cm}M{2.8cm}M{12cm}@{}}
\hline
\textbf{\HealthPointsInitials} & \HealthPoints{} & \flufffont{When the model loses this many Health Points, it is removed as a casualty.} \tabularnewline
\textbf{\DefensiveSkillInitials} & \DefensiveSkill{} & \flufffont{How well the model avoids being hit in melee.} \tabularnewline
\textbf{\ResilienceInitials} & \Resilience{} & \flufffont{A high Resilience withstands blows more easily.} \tabularnewline
\textbf{\ArmourInitials} & \Armour{} & \flufffont{The innate Armour of the model.} \tabularnewline
\hline
\end{tabular}
\end{center}

\subsection{Offensive Characteristics Profile}

If a model consists of more than one model part, each model part has its own set of Offensive Characteristics. Each model part has five Offensive Characteristics:

\begin{center}
\begin{tabular}{M{0.75cm}M{2.8cm}M{12cm}@{}}
\hline
\textbf{\AgilityInitials} & \Agility{} & \flufffont{Models parts with a higher Agility strike first in melee.} \tabularnewline
\textbf{\AttackValueInitials} & \AttackValue{} & \flufffont{The number of times the model part can strike in a round of combat.} \tabularnewline
\textbf{\OffensiveSkillInitials} & \OffensiveSkill{} & \flufffont{How good the model part is at landing hits in melee.} \tabularnewline
\textbf{\StrengthInitials} & \Strength{} & \flufffont{The higher the Strength, the easier it is to wound other models.} \tabularnewline
\textbf{\ArmourPenetrationInitials} & \ArmourPenetration{} & \flufffont{The higher the Armour Penetration, the easier it is to penetrate the opponent's armour.} \tabularnewline
\hline
\end{tabular}
\end{center}

\subsection{Random Characteristics}

Some model parts have random values for one or more of their Characteristics (e.g. Attack Value D6+1). Roll for the value each time just before it is needed. In the case of Defensive and Offensive Characteristics, the rolled value is used for all simultaneous attacks (such as Shooting Attacks from a single unit or Melee Attacks at the same Initiative Step). When several models in the same unit have a random value for a Characteristic, roll separately for each model. Advance Rate, March Rate, and Health Points cannot be random (a random Advance Rate means the model has the Random Movement Universal Rule).

\section{Performing a Characteristic Test}
\label{performing_a_characteristic_test}

To perform a Characteristic Test, roll a D6. If the result is less than or equal to the tested Characteristic, the test is passed. Otherwise the test is failed. A test always fails on the result of \result{6}.

When a model with more than one value for a Characteristic (such as a rider and its horse) is called upon to take a Characteristic Test, take a single test for the combined model, using the highest value available. When a unit as a whole takes a Characteristic Test, the highest value is used.

\subsection{Performing a Discipline Test}
\label{performing_a_discipline_test}

To perform a Discipline Test, the player rolls 2D6 and compares the result with the Discipline Characteristic of a model. If the result is less than or equal to the Discipline value, the test is passed. Otherwise, the test is failed. If a unit takes a Discipline Test and more than one Discipline value is available (e.g. when a Character joins a unit), the player may choose which Discipline value to use.

Many different game mechanics call for a Discipline Test, such as performing a Panic Test or a Break Test. All such game mechanics are Discipline Tests, regardless of any additional rules and modifications described in the relevant sections of the Rulebook.

\section{Using Borrowed Characteristics}
\label{using_borrowed_characteristics}

In certain situations a model may borrow or use the Characteristic of another model. In this case the value of the borrowed Characteristic is taken after applying any modification from equipment, spells, or Model Rules to which the owner of the Characteristic is subject to. Modifications from equipment, spells, or Model Rules that affect the unit will then be applied to this (following the rules in \totalref{priority_of_modifiers}).

\section{Priority of Modifiers}
\label{priority_of_modifiers}

The modifiers of Characteristics, dice rolls, or other values are applied in a strict order (whenever you see the terms set/borrow/always/never used in \textbf{bold} in such a modifier, this is indicative for its priority), following table \ref{table/priority_of_modifiers} below:

\begin{table}[!htbp]
\centering
  \begin{tabular}{M{0.5cm} m{7cm} m{8cm}}
    & Priority of Modifiers & Examples \\
    \midrule
    1 & Values \textbf{set} to a certain number and values \textbf{borrowed}. If a borrowed Characteristic is modified, apply these modifiers before borrowing the Characteristic. & \enquote{Attacks made with this weapon are \textbf{set} to hit on 4+.}\newline \enquote{All units within \distance{12} may \textbf{borrow} this model's Discipline.} \\
    2 & Multiplication and division. Round fractions up. & \enquote{All attacks made against this model are performed at half Strength.} \\
    3 & Addition and subtraction. & \enquote{The wearer gains +2 Attack Value.} \\
    4 & Rolls \textbf{always} or \textbf{never} succeeding or failing on certain results, and Characteristics \textbf{always} or \textbf{never} set to a certain value or range of values. & \enquote{The target can \textbf{never} be wounded on better than a natural 5+.}\newline \enquote{The wielder \textbf{always} has Resilience 5.}\newline \enquote{Attacks made with this weapon \textbf{always} have minimum Strength 4.} \\
  \end{tabular}
 \caption{Priority of Modifiers.}
 \label{table/priority_of_modifiers}
\end{table}

When several modifiers within a group are to be applied to a value (e.g. Characteristic values), apply them in the order that results in the lowest value.

When several modifiers within a group are to be applied to a dice roll (e.g. for \hyperref[aegis]{Aegis} Saves, to-hit rolls, to-wound rolls), apply them in the order that results in the lowest success chance of the roll.

\newpage
After all modifications via multiplication, division, addition, or subtraction:

\begin{itemize}[label={-}]
\item \textbf{Agility} and \textbf{Attack Value} cannot be modified to lower than 1,
\item the value of all other Characteristics cannot be modified to lower than 0,
\item \textbf{Armour} cannot be modified to exceed a maximum of 6,
\item \textbf{Agility} and \textbf{Discipline} cannot be modified to exceed a maximum of 10
\end{itemize}

(unless noted otherwise).



