
\part{Introduction}

\section{What is \nameofthegame{}?}

\nameofthegame{}, often simply called The 9th Age, is a community-made miniatures war game in which two grand armies clash in an epic battle for power or survival. Each army can be composed of simple foot soldiers, skilled archers, armour-clad mounted knights, powerful wizards, legendary heroes, epic monsters, huge dragons, and more. The game takes place on a 4 by 6 foot battlefield and uses six-sided dice to resolve different actions such as charging into battle, letting arrows loose, or casting spells.

All relevant rules, as well as feedback and suggestions, can be found/given here:
\begin{center}
\href{https://www.the-ninth-age.com/}{https://www.the-ninth-age.com/}
\end{center}

\protectednewrule{Recent rule changes are colour-coded like this sentence}, \protectedrewordedrule{while recent rewordings are colour-coded like this one}. All changes can be found in the change log:
\begin{center}
\href{https://www.the-ninth-age.com/archive.html}{https://www.the-ninth-age.com/archive.html}
\end{center}

\begin{itemize}[label={-}]
\item

Keywords and titles main terms are highlighted by starting with a capital letter, as in Round of Combat, unless they are too common, as \enquote{model} or \enquote{unit}.

\item

\textbf{Bold font} is used to highlight essential words and Model Rules given by another Model Rule (intricate rules). It is used in Army Books to highlight Model Rules that are defined in the unit profile.

\item

\textit{Italic font} is used for spell names, background text, or repetitive text (as in tables).

\item
\greytextcolor{Grey-coloured text} is used for figure and table captions, item restrictions, and repetitive text.
\item

(Brackets) are used for clarifications and explanations of the actual rules, and for defining parameters of some model rules.

\end{itemize}

\begin{hidewhenprinted}\textcolor{linkcolour}{The electronic version of this document has clickable hyperlinks colour-coded like this sentence, and the page footer displays hyperlinks to key sections.}\end{hidewhenprinted}


\begin{center}
\begin{framed}
\license
\end{framed}
\end{center}

\noindent \latexcredit

\section{The Scale of the Game}

Playing tabletop war games is often an exercise in abstract thought, especially when it comes to mass battle games like \theninthage{}. As such there is no prescribed scale while playing \theninthage{}; a single miniature could represent a single, a dozen, or even a hundred warriors. We believe the timescale of the game to be even more arbitrary than the scale of the game. Hence, no quantitative value can be assigned to a game turn or turn sub-phase.
