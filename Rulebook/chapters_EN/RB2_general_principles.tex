
\part{General Principles}
\label{general_principles}

\section{Turn}
\label{turn}

\nameofthegame{} is a turn based game, however there is no set length of time that a single turn represents. The action of moving in the Movement Phase could take several minutes of real time, while casting spells in the Magic Phase or shooting in the Shooting Phase could be near instantaneous events. Likewise the actions of two units clashing in the Melee Phase could represent only a few heartbeats in real time, while a Duel between two mighty individuals could be a drawn combat lasting minutes or more.

A standard game lasts for 6 Game Turns, each divided into two Player Turns. At the beginning of the game, one player has the first Player Turn, in which they use their units to perform various actions, such as moving, casting spells, or charging, while their opponent gets to react. After this, the other player has their first Player Turn. When this comes to an end, Game Turn 1 is complete. In Game Turn 2, the first player now has their second Player Turn, and so on, until both players have completed 6 Player Turns. This marks the end of the game.

\subsection{Player Turn}
\label{player_turn}

Each Player Turn is divided into five phases, performed in the following order:

\hspace*{0.3cm}
\begin{tabular}{c|l}
1 & Charge Phase \tabularnewline
2 & Movement Phase \tabularnewline
3 & Magic Phase \tabularnewline
4 & Shooting Phase \tabularnewline
5 & Melee Phase \tabularnewline
\end{tabular}

\subsection{Active and Reactive Player}
\label{active_and_reactive_player}

The Active Player is the player whose turn it currently is.

The Reactive Player is the player whose turn it currently is not.

\subsection{Simultaneous Effects}
\label{simultaneous_effects}

Whenever two or more effects occur at the same time, resolve effects controlled by the Active Player first. For effects that happen during the Pre-Game Sequence, consider the player that picked their Deployment Zone to be the Active Player. For effects that happen during the Deployment Phase, consider the player that finished deploying first to be the Active Player. Each player is free to decide in which order their own simultaneous abilities resolve. If there is a choice involved (such as abilities that may or may not be activated), the Active Player must declare the use of their abilities before the Reactive Player. Once both players have declared the use and order of their abilities, resolve their effects, starting with those of the Active Player. For example, if both players have abilities that may be activated at the beginning of the Magic Phase, the player whose Magic Phase it is must choose first whether or not they are using their abilities and in which order. Then the Reactive Player may choose to use their abilities or not. After this, the effects of the abilities from both sides are resolved, starting with the Active Player's abilities.

\section{Dice}
\label{dice}

\subsection{Rolling Dice}
\label{rolling_dice}

In \nameofthegame{}, dice are often used to determine random outcomes. The most commonly used type is the six-sided dice, named D6, with a range from 1 to 6. The effects of a dice roll are often dependent on whether the rolled value is equal to or higher than a set value (such as a dice roll that is successful if the dice rolls \result{3} or higher). This is often referred to as a 3+ (or 2+, 4+, 6+, etc.). Sometimes you are called upon to roll more than one of these dice at the same time. This is represented by a number before the type of dice rolled, such as 3D6, which means to roll 3 six-sided dice and add the results together. On other occasions, a dice roll may be modified by adding or subtracting a number, such as D6+1. In such cases, simply add the relevant number to, or subtract it from, the result of the roll. A natural roll on a D6 refers to the value of the dice, before any modifiers are applied. Lastly, some effects in the game call for rerolling certain dice, such as \enquote{failed to-wound rolls}, or \enquote{Aegis Save results of \result{1}}. When you encounter such situations, reroll the relevant dice. Note that rerolling a dice is not considered a modifier.

\textbf{Dice can only be rerolled once.} The second result is final, no matter the cause, source, or result.

\subsubsection{Rolling a D3}

The game sometimes requires the roll of a D3. This is performed by rolling a D6 and then halving the result, rounding up, so that the result can only be 1, 2 or 3. If the game requires a natural \result{1} or a natural \result{6} when rolling D3, it always refers to the value of the D6 before halving.

\subsubsection{Maximised Roll and Minimised Roll}
\label{maximised_roll}

For dice rolls subject to Maximised Roll, roll one additional D6 and discard the lowest D6 rolled. For dice rolls subject to Minimised Roll, roll one additional D6 and discard the highest D6 rolled. These rules are cumulative (e.g. for a roll affected by two instances of Maximised Roll, you roll two additional D6 and discard the two lowest D6 rolled). The results of the discarded D6 are ignored for all intents and purposes.

\subsection{The Direction Dice and Random Directions}
\label{the_direction_dice_and_random_directions}

The Direction Dice is a special six-sided dice with all sides marked with an arrow. Certain rules may ask the player to determine a random direction. In such cases, roll the Direction Dice and then use the direction in which the arrow points.

\paragraph{Representing the Direction Dice with a Standard D6}

Alternatively the Direction Dice can be represented by rolling a standard six-sided dice and using the side with a single dot (i.e. the \result{1}) to represent the direction of an arrow as depicted in figure \ref{figure/deviation_dice}. If rolling a result of \result{1} or \result{6} (\result{1} and \result{6} are on opposite faces on a standard dice), use the central dot in the \result{5} to represent the direction of the arrow instead.

\begin{figure}[!htbp]
\centering
\includegraphics[width=2cm]{pics/deviation_dice.png}
\caption{Representing the Direction Dice with a Standard D6.}
\label{figure/deviation_dice}
\end{figure}

