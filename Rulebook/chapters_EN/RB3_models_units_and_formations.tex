
\part{Models, Units and Formations}
\label{models_unit_and_formations}

\section{Models}
\label{models}

Models represent fighting warriors, monsters, and spell casters. Everything that stands on the same base is considered the same model (e.g. a dragon and its rider or a cannon and its three crewmen are considered a single model).

Players are welcome to interpret the scale as they like, as the distances used in the rules do not seem realistic if the scale of 1:1 compared to the actual size of the miniatures is used for the game. The scale of miniatures most commonly used for \theninthage{} ranges from 1:70 to 1:50 when compared to real-life sized equivalents for human-sized creatures. Many units are commonly represented by \SI{25}{mm}-\SI{32}{mm} miniatures (a common form of measuring human miniature size is measuring the model's height to the eyes).

\theninthage{} does not officially support any particular product line, and you are welcome to play with whatever scale and miniatures you and your opponent have agreed upon. However, it is very important to make sure you mount your models (regardless of scale or size) on the correct base size for the unit entry.

Just as we can imagine that the combatants in the game are actually smaller than the miniatures that represent them, so we can imagine that a single miniature does not necessarily represent a single warrior. We could imagine a unit of 10 elite elven warriors representing exactly 10 elves or some other group size like 20, 50, or 100. At the same time a unit of 10 weedy goblin runts could just represent 10 goblins, but is more likely to represent some larger group of 100, 200, or 500 of the little creatures. At this point one may ask about characters and monsters. These models are meant to represent exceptional individuals and especially potent creatures that are worth entire regiments on their own. It may be easier to come to terms with a miniature of a character representing not just the character itself but also their bodyguards and assorted staff that might follow such a hero to battle.

\subsection{Bases}
\label{bases}

All models are placed on a rectangular or round base. The extent of the model is considered to be its base. Base sizes are given as two measurements in millimetres: front-width x side-length (e.g. most horse riders' bases are \num{25x50} \si{\milli\meter}). In some rare cases models have round bases. In these cases, only a single measurement is given: the diameter of the base (e.g. a standard War Machine base is a round \SI{60}{\milli\meter} base).

\subsection{Multipart Models}
\label{multipart_models}

Models with more than one Offensive Profile are called Multipart Models. For example, a cavalry model is a model with two parts (the rider and its mount), a chariot can be described as a single model consisting of five parts (two horses, two crewmen, and one chariot chassis), while a normal foot soldier is a model which consists of a single part. Each part of such a model has its own Offensive Profile and can be referred to as a model part. Sometimes a model has multiple identical parts. In this case, the model part name is followed by a number in brackets. For example, a chariot might have three charioteers, which would be noted as \enquote{Charioteer (3)}.

Whenever a rule, ability, spell, and so on affects a model, all parts of the model are affected, unless the rule specifically states it only affects a model part. When attacking or shooting, each model part of the multipart model uses its own Characteristics and weapons.

\section{Units}
\label{units}

All models are part of a unit. A unit is either a group of models deployed in a formation consisting of ranks (along the width of the unit) and files (along the length of the unit) or a single model operating on its own.

Whenever a rule, ability, spell, and so on affects a unit, all models in the unit are affected. When forming a unit, all models in the unit must be perfectly aligned in base contact with each other and face the same direction. Models in a unit that are not in the first rank must be positioned so that another model is directly in front of them. All ranks must always have the same width, except the last rank which can be shorter than the other ranks; if so, this is called an incomplete rear rank. Note that it's perfectly fine for the last rank to have empty gaps in it, as long as the models are aligned with those of the other ranks. Following these rules, you are free to field your units in whatever formation, as few or as many files wide as you wish, but this may affect rules that interact with the unit (like \totalref{line_formation}).

\subsection{Rank-and-File}
\label{rank_and_file}

Normal models in a unit are called Rank-and-File models (\rnf{}). All models except Characters are \rnf{} models.

\subsection{Full Ranks}
\label{full_ranks}

If there are enough models in a rank it is considered a Full Rank. How many models are needed is determined by the Size of the unit (see \totalref{model_classification}. Units of Standard Size need 5 models, Large need 3 models, and Gigantic need 1 model.

\subsection{Close Formation \&{} Line Formation}
\label{line_formation}

Units are normally considered in Close Formation. Units in ranks of 8 or more models are instead considered to be in Line Formation (see \totalref{melee_phase} for details on their in-game effects). Units in Line Formation gain the \hyperref[fight_in_extra_rank]{\textbf{Fight in Extra Rank}} Attack Attribute, but cannot add Rank Bonus to their Combat Score.

\subsection{Health Pools}
\label{health_pools}

All Health Points of a unit are part of one or more Health Pools. The Health Points of all non-\hyperref[champion]{Champion} \rnf{} models of a unit form a separate Health Pool, while the Champion and each \hyperref[characters]{Character} joined to the unit each have their own Health Pool.

\subsection{Footprint}
\label{footprint}

A unit's Footprint is the area occupied by the combined bases of all the models in a unit.

\subsection{Boundary Rectangle}
\label{boundary_rectangle}

A unit's Boundary Rectangle is the imaginary rectangle around the outer edges of the unit's Footprint. A unit usually cannot be inside another unit's Boundary Rectangle, unless they are overlapping (see \totalref{interactions_between_objects}).

\subsection{Centre of Unit}
\label{centre_of_unit}

A unit's Centre is the centre of its Boundary Rectangle.

\subsection{Unit Facings}
\label{unit_facings}

A unit has 4 Facings (Front, Rear, and two Flanks). The Facings are the edges of the unit's Boundary Rectangle. Units on round bases don't have Facings.

\subsection{Unit Arcs}
\label{unit_arcs}

A unit has 4 Arcs: Front, Rear, Left Flank, and Right Flank. Each Arc is determined by extending a straight line from the corners of the unit's Boundary Rectangle, in a \SI{135}{\degree} angle from the unit's Facings (see figure \ref{figure/arcs}). Any object at least tangent to the lines delimiting an Arc is considered to be inside that Arc. Units on round bases don't have Arcs.

\newcommand{\ARCSa}{a)}
\newcommand{\ARCSb}{b)}
\newcommand{\ARCSc}{c)}

\newcommand{\frontarc}{Front Arc}
\newcommand{\leftflankarc}{%
\begin{minipage}{0.10\unitlength}\begin{center}%
Left Flank Arc%
\end{center}\end{minipage}}
\newcommand{\rightflankarc}{%
\begin{minipage}{0.10\unitlength}\begin{center}%
Right Flank Arc%
\end{center}\end{minipage}}
\newcommand{\reararc}{Rear Arc}
\newcommand{\firstangle}{\SI{90}{\degree}}
\newcommand{\secondangle}{\SI{135}{\degree}}

\newcommand{\frontfacing}{Front Facing}
\newcommand{\leftflankfacing}{%
\begin{minipage}{0.10\unitlength}\begin{center}%
Left Flank Facing%
\end{center}\end{minipage}}
\newcommand{\rightflankfacing}{%
\begin{minipage}{0.10\unitlength}\begin{center}%
Right Flank Facing%
\end{center}\end{minipage}}
\newcommand{\rearfacing}{Rear Facing}
\newcommand{\centreofunit}{Centre of Unit}

\begin{figure}[!htbp]
\centering
\def\svgwidth{\textwidth}
\input{pics/arcs.pdf_tex}
\caption{
This unit has 3 ranks and 6 files. The base on the side is a Character with a Mismatching Base that has joined the unit. The last rank is incomplete and only contains 3 models.\captionpar
a) The Front, Flank, and Rear Arcs are defined by using a \SI{135}{\degree} angle from the unit's edges.\captionpar
b) The Footprint is the area covered by the bases of all models in the unit (green area).\captionpar
c) The Boundary Rectangle is the area drawn around the outer edges of the unit's Footprint (grey area). The Centre of the unit is the centre of the Boundary Rectangle (red x).
}
\label{figure/arcs}
\end{figure}

\section{Interactions between Objects}
\label{interactions_between_objects}

There are many ways models, units, and other objects in the game interact with one another (see figure \ref{figure/contact}).

\newcommand{\CONTACTa}{a)}
\newcommand{\CONTACTb}{b)}
\newcommand{\CONTACTc}{c)}
\newcommand{\CONTACTd}{d)}

\begin{figure}[!htbp]
\def\svgwidth{\textwidth}
\input{pics/contact.pdf_tex}
\caption{\textcolor{white}{nothing}}
\greytextcolor{%
\begin{minipage}{0.205\textwidth}
a) Contact in a line.
\end{minipage}\begin{minipage}{0.265\textwidth}
b) Contact in a single point.
\end{minipage}\begin{minipage}{0.27\textwidth}
c) Partially inside.
\end{minipage}\begin{minipage}{0.25\textwidth}
d) Fully inside.
\end{minipage}
}
\label{figure/contact}
\end{figure}

\subsection{Base Contact between Units and Models}
\label{base_contact_between_units_and_models}

Two \textbf{units} are considered in base contact with each other if their Boundary Rectangles are touching one another (including corner to corner contact).

Two \textbf{models} on rectangular bases are considered to be in base contact with each other if their bases are touching one another (including corner to corner contact). If incomplete ranks or Characters with \hyperref[mismatching_bases]{Mismatching Base} cause empty gaps between two opposing models whose units are in base contact, consider these models to be in base contact with each other over such empty gaps (see figure \ref{figure/empty_gaps} page \pageref{figure/empty_gaps}).

\subsection{Contact between Objects}
\label{contact_between_objects}

Two objects, like bases, Boundary Rectangles, Terrain Features, and so on, are considered to be in \textbf{contact}:

\begin{itemize}[label={-}]
\item if they are tangent (i.e. they touch one another)
\begin{itemize}[label={}]
\item[(a)] along a line (e.g. two \num{20x20} \si{\milli\meter} bases contacting each other along their front).
\item[(b)]at a single point (e.g. corner to corner contact between units).
\end{itemize}
\item if one object is inside another. An object is considered to be \textbf{inside} another if it is
\begin{itemize}[label={}]
\item[(c)] partially inside the other object.
\item [(d)] fully inside the other object.
\end{itemize}
\end{itemize}

\subsection{Overlapping Objects}
\label{overlapping_of_objects}

Two objects are considered to be overlapping if they (or in case of units, their Boundary Rectangles) are at least partially on top of one another (including the edge of both objects), without the two objects being in contact (e.g. a unit with Flying Movement and a Terrain Feature).

\subsection{Interactions with Round Bases}
\label{interactions_with_round_bases}

The Boundary Rectangle of units composed of models on round bases is identical to their Footprint. Units are considered in base contact with a model on a round base if their Boundary Rectangles are in contact. Models are considered to be in base contact with a model on a round base if their units are in base contact and if the Boundary Rectangle of the model on a round base is:
\begin{itemize}[label={-}]
\item directly in front of them (in case the model on round base is located in their Front Facing)
\item directly to their side (in case it is located in their Flank Facing)
\item or directly behind them (in case it is in their Rear Facing)
\end{itemize}
without any models in between them. See figure \ref{figure/empty_gaps} page \pageref{figure/empty_gaps} for an example.
