
\part{The Pre-Game Sequence}
\label{the_pre_game_sequence}

For setting up a game of \nameofthegame{}, the players need to go through the following steps, referred to as the Pre-Game Sequence.

\hspace*{0.3cm}
\begin{tabular}{c|l}
1 & Decide on the size of the game. \tabularnewline
2 & Share your Army List with your opponent. \tabularnewline
3 & Build the Battlefield. \tabularnewline
4 & Determine the Deployment Type. \tabularnewline
5 & Determine the Secondary Objective. \tabularnewline
6 & Determine the Deployment Zones. \tabularnewline
7 & Select Spells. \tabularnewline
8 & Declare Special Deployment. \tabularnewline
9 & Deployment Phase. \tabularnewline
\end{tabular}

\section{The Size of the Game}
\label{the_size_of_the_game}

In \nameofthegame{}, two armies opposing each other on the Battlefield must have roughly the same Point Cost. This ensures that the battle will be decided through the clever strategies and tactics of the players rather than by an unfair difference in army size. The first step to setting up a game is to agree on the Army Points. This pre-determined Point Cost will henceforth correspond to the size of the game. Armies are typically worth between 1500 and 3000 points for small engagements, between 3000 and 8000 points for serious battles, and beyond 8000 points for mighty clashes between epic armies. For an optimal gaming experience we advise playing at 4500 points. This combines sufficient time to play the game with optimal balance. This does not mean that the game is unbalanced at 4000 or 5000 points, it is just not balanced in the same level. As the game size moves further in either direction from what we consider optimal the balance level will decrease, which will however not make the game unplayable. Therefore we encourage players to explore various point sizes and report their experiences to the \theninthage{} team so we can make a better game together.

\section{Sharing Army Lists}
\label{sharing_army_lists}

After deciding on the size of the game, the next step is for both players to swap Army Lists and share all relevant information about the upcoming game. Alternatively, the players may agree to keep certain aspects about their armies secret, which they will progressively reveal during the course of the game. For more information please see \totalref{optional_rules_for_hidden_lists}.

\newpage
\section{Building the Battlefield}
\label{building_the_battlefield}

A game of \nameofthegame{} is played on a board that is \distance{72} wide and \distance{48} deep. For smaller battles involving Warbands, it is recommended to use a board that is \distance{36} wide and \distance{48} deep (half the standard board), while for bigger games involving Grand Armies it is recommended to adjust the size depending on the size of the armies. While some battles may take place on a completely open board, a Battlefield typically has Terrain Features placed upon it. These pieces of terrain could be viewed to represent exactly what they are, but they could also be visual representations of far greater things for the purpose of the game. So a copse of trees could represent a forest, a stream could actually be a wide river, a single house could denote a hamlet, and a tower could represent a keep. The players can freely agree on the size, type, and number of Terrain Features to be placed, as well as their positions. If an agreement cannot be reached, the game provides the following default rules for setting up a randomly generated Battlefield.

\begin{itemize}[label={-}]
\item First, divide the gaming board into six \distance{24} x \distance{24} sections (four \distance{18} x \distance{24} if the board is \distance{36} x \distance{48}).

\item Place the following Terrain Features in the centre of three different randomly selected sections: \begin{itemize}[label={-}]
\item one \building{} or \cliffs{} (randomly decide which of the two);
\item one \hill{};
\item one \forest{}.
\end{itemize}
Then move each Terrain Feature \distance{2D6} in a random direction.

\item Next, add 2D3 (1D3 if the board is \distance{36} x \distance{48}) additional Terrain Features in the centre of different randomly selected sections, and then move each Terrain Feature \distance{2D6} in a random direction. Roll 2D6 and consult the table below to determine the type of each additional Terrain Feature.
\begin{center}
\setlength{\tabcolsep}{10pt}
\begin{tabular}{cccccccc}
\hline
\specialcell{2 - 4 \\ \hill} & \specialcell{5 \\ \water} & \specialcell{6 \\ \field} &
\specialcell{7 \\ \forest} & \specialcell{8 \\ \cliffs} & \specialcell{9 \\ \ruins} &
\specialcell{10 \\ \wall} & \specialcell{11-12 \\ \building} \tabularnewline
\hline
\end{tabular}
\end{center}

\item Terrain Features cannot be moved to be closer than \distance{6} from each other. You may move the pieces as little as possible from their rolled position in order to meet this criterion. If it is not possible to place the Terrain Feature more than \distance{6} away from any other Terrain then discard the problematic piece.

\item Recommended Terrain Feature sizes are between \distance{6} x \distance{8} and \distance{6} x \distance{10}, except for Walls, which are \distance{1} x \distance{10}.
\end{itemize}

\subsubsection{Board Edge}
\label{board_edge}

The Board Edge represents the boundaries of the game. 

\section{Deployment Type}
\label{deployment_type}

Determine the Deployment Type. If no outside source tells you what Deployment Type to use (e.g. tournament organiser, campaign rules, etc.), players may agree on which one to play. Otherwise randomise by rolling a D6 and consulting the list below.

Certain Deployment Types refer to the Centre Line. This is the line drawn through the centre of the board and parallel to the table's long edges, dividing the table into halves.

\begin{minipage}{0.65\textwidth}
\hypertarget{frontline_clash}{\paragraph{1\spacebeforecolon{}: \frontlineclash}}
\label{figure/deployment}
\vspace*{0.5ex}

Deployment Zones are areas more than \distance{12} away from the Centre Line.
\end{minipage}\hfill\begin{minipage}{0.32\textwidth}
\def\svgwidth{\textwidth}
\input{pics/deployment_1_frontline.pdf_tex}
\end{minipage}

\begin{minipage}{0.65\textwidth}
\hypertarget{dawn_assault}{\paragraph{2\spacebeforecolon{}: \dawnassault}}
\vspace*{0.5ex}

The player who chooses the Deployment Zone also chooses a short table edge (the other player gets the opposite short table edge). Deployment Zones are areas more than \distance{12} away from the Centre Line and more than 1/4 of the table's length from the opponent's short table edge (\distance{18} on a \distance{72} table).

When declaring Special Deployment, players may choose to keep up to two of their units as reinforcement. These units follow the rules for \hyperref[ambush]{Ambush}ing units, except that they must be placed touching the controlling player's short table edge when they arrive.
\end{minipage}\hfill\begin{minipage}{0.32\textwidth}
\def\svgwidth{\textwidth}
\input{pics/deployment_2_dawnassault.pdf_tex}
\end{minipage}

\begin{minipage}{0.65\textwidth}
\hypertarget{counterthrust}{\paragraph{3\spacebeforecolon{}: \counterthrust}}
\vspace*{0.5ex}

Deployment Zones are areas more than \distance{8} away from the Centre Line. Units must be deployed more than \distance{20} away from enemy units. Units using Special Deployment, such as \hyperref[scout]{\scout}, ignore these rules. During their first 3 deployment turns, each player must deploy a single unit, and cannot deploy any Characters.
\end{minipage}\hfill\begin{minipage}{0.32\textwidth}
\def\svgwidth{\textwidth}
\input{pics/deployment_3_counterthrust.pdf_tex}
\end{minipage}

\begin{minipage}{0.65\textwidth}
\hypertarget{encircle}{\paragraph{4\spacebeforecolon{}: \encircle}}
\vspace*{0.5ex}

The player who chooses the Deployment Zone decides if they want to be the attacker or the defender. The attacker must deploy more than \distance{9} from the Centre Line if entirely within a quarter of the table's length from either short table edge (\distance{18} on a \distance{72} table), and more than \distance{15} from the Centre Line elsewhere. The defender does the opposite: more than \distance{15} away from the Centre Line if within a quarter of the table's length from the short table edge, and more than \distance{9} away from the Centre Line elsewhere.
\end{minipage}\hfill\begin{minipage}{0.32\textwidth}
\def\svgwidth{\textwidth}
\def\deploymentfigAttacker{Attacker}
\def\deploymentfigDefender{Defender}
\input{pics/deployment_4_encircle.pdf_tex}
\end{minipage}

\begin{minipage}{0.65\textwidth}
\hypertarget{refused_flank}{\paragraph{5\spacebeforecolon{}: \refusedflank}}
\vspace*{0.5ex}

The table is divided into halves by a diagonal line across the table. Whoever gets to choose the Deployment Zone decides which diagonal to use. Deployment Zones are areas more than \distance{9} away from this line.
\end{minipage}\hfill\begin{minipage}{0.32\textwidth}
\def\svgwidth{\textwidth}
\input{pics/deployment_5_refusedflank.pdf_tex}
\end{minipage}

\begin{minipage}{0.65\textwidth}
\hypertarget{marching_columns}{\paragraph{6\spacebeforecolon{}: \marchingcolumns}}
\vspace*{0.5ex}

Deployment Zones are areas more than \distance{12} away from the Centre Line.

Each player must choose a short table edge when deploying their first unit. Each unit this player deploys afterwards must be deployed with its Centre further away from the chosen short table edge than the Centre of the last unit this player deployed (measure from the closest point on the short table edge). \hyperref[war_machine]{War Machines}, \hyperref[war_platform]{War Platforms}, \hyperref[characters]{Characters}, and \hyperref[scout]{Scouting} units ignore these rules.

During their first 3 deployment turns, each player must deploy only a single unit, and cannot deploy any Characters. Instead of deploying a unit, a player may choose to make all undeployed units Delayed that are not using Special Deployment. Delayed units follow the rules for Ambushing units with the following exceptions:
\begin{itemize}[label={-}]
\item In each Player Turn, after rolling for all \hyperref[ambush]{Ambushing} units, the opponent chooses the order in which all Delayed units that passed the 3+ roll enter the Battlefield.
\item In the chosen order, each unit (one after another) must be placed with the centre of its rear rank as close as possible to the centre of the long table edge in their owner's Deployment Zone before any non-Delayed Ambushers are placed on the Battlefield.
\item After all arriving units have been placed, they can be moved as described in the rules for Ambush p. \pageref{ambush}.
\end{itemize}
\end{minipage}\hfill\begin{minipage}{0.32\textwidth}
\def\svgwidth{\textwidth}
\input{pics/deployment_6_marchingcolumns.pdf_tex}
\end{minipage}

\newpage
\section{Secondary Objective}
\label{secondary_objectives}

Before choosing Deployment Zones, determine the Secondary Objective. If no outside source tells you which one to use (e.g. tournament organiser, campaign rule, etc.), players may agree on a Secondary Objective. Otherwise, randomise by rolling a D6 and consulting the following list.

\paragraph{1: Hold the Ground}

\flufffont{Secure and hold the Battlefield centre.}\newline
Mark the centre of the board.

At the end of each Game Turn after the first, the player with the most \hyperref[scoring]{Scoring Units} within \distance{6} of the centre of the board gains a counter. At the end of the game, the player with the most such counters wins this Secondary Objective.

\paragraph{2: Breakthrough}

\flufffont{Invade the enemy territory.}\newline
The player with the most \hyperref[scoring]{Scoring Units} inside their opponent's Deployment Zone at the end of the game, up to a maximum of 3, wins this Secondary Objective.

\paragraph{3: Spoils of War}

\flufffont{Gather precious loot.}\newline
Place 3 markers along the line dividing the board into halves (the dashed line in the figures describing Deployment Types). One marker is placed on a point on this line that is as close as possible to the centre of the board while still being more than \distance{1} away from Impassable Terrain. The other two markers are placed on points on this line that are on either side of the central marker, as close to the centre as possible but at least a third of the long table edge length (\distance{24} on a standard board) away from it, and more than \distance{1} away from Impassable Terrain.

At the start of each of your Player Turns, each of your Scoring units that is not carrying a marker may pick up a single marker they are in contact with. Remove the marker from the Battlefield: the unit is now carrying the marker. Units carrying a marker with less than 3 \hyperref[full_ranks]{Full Ranks} have their March Rate set to their Advance Rate. If a unit carrying a marker is destroyed or loses \hyperref[scoring]{Scoring} (ignore \hyperref[post_combat_reform]{Post-Combat Reform} for this purpose), the opponent must immediately place the marker carried by this unit on a point within \distance{3} of it. This point cannot be within \distance{1} of Impassable Terrain, but it can be inside a unit.

At the end of the game, the player with the most units carrying markers wins this Secondary Objective.

\paragraph{4: King of the Hill}

\flufffont{Desecrate your opponent's holy ground while protecting yours.}\newline
After Spell Selection (at the end of step 7 of the Pre-Game Sequence), both players choose a Terrain Feature that isn't Impassable Terrain and that is not entirely within their Deployment Zone, starting with the player that chose their Deployment Zone (note that both players may choose the same Terrain Feature).

A player captures the opponent's chosen Terrain Feature if any of his \hyperref[scoring]{Scoring Units} are inside that Terrain Feature at the end of the game. A player wins this Secondary Objective if he captures the opponent's chosen Terrain Feature while his chosen Terrain Feature is not captured by his opponent.

\paragraph{5: Capture the Flags}

\flufffont{Valuable targets must be annihilated.}\newline
After Spell Selection (at the end of step 7 of the Pre-Game Sequence), mark all \hyperref[scoring]{Scoring Units} on both players' Army Lists. If either player has less than 3 marked units, their opponent must mark enough units from this player's Army List so that there are exactly 3 marked units in the army, starting with the player that chose their Deployment Zone.

The player that has the lowest number of their marked units removed as casualties at the end of the game wins this Secondary Objective.

\paragraph{6: Secure Target}

\flufffont{Critical resources must not fall into enemy hands.}\newline
Directly after determining Deployment Zones, both players place one marker on the Battlefield, starting with the player that chose their Deployment Zone. Each player must place the marker on a point that is more than \distance{12} away from their Deployment Zone and a third of the long table edge length (\distance{24} on a standard board) from the point marked by the other marker.

At the end of the game, the player controlling the most markers wins this Secondary Objective. A marker is controlled by the player with the most Scoring Units within \distance{6} of the marker. If a unit is within \distance{6} of both markers, it only counts as within \distance{6} of the marker which is closest to its Centre (randomise if both markers are equally close).

See \totalref{victory_conditions} for more details on how capturing an objective affects who is the winner.

\section{Deployment Zones}
\label{deployment_zones}

Both players roll a D6. The player that rolls higher chooses their Deployment Zone and follows the Deployment Type-specific instructions if applicable (if a tie is rolled, roll again).

\section{Spell Selection}
\label{spell_selection}

Starting with the player that picked their Deployment Zone, each player must now select spells for their Wizards, one at a time. All Magic Paths can be found in \nameofthegame{} - Paths of Magic. \hereditaryspells{} can be found in the corresponding Army Books.

\subsubsection{\wizardapprentice}

\begin{itemize}[label={-}]
\item Knows \textbf{1 spell}.
\item Can select between the \learnedspell{} \textbf{1} of its chosen Path and the \textbf{Hereditary} Spell of its army.
\end{itemize}

\subsubsection{\wizardadept}

\begin{itemize}[label={-}]
\item Knows \textbf{2 spells}.
\item Can select from the \learnedspells{} \textbf{1, 2, 3, and 4} of its chosen Path and the \textbf{Hereditary} Spell of its army.
\end{itemize}

\subsubsection{\wizardmaster}

\begin{itemize}[label={-}]
\item Knows \textbf{4 spells}.
\item Can select from the \learnedspells{} \textbf{1, 2, 3, 4, 5, and 6} of its chosen Path and the \textbf{Hereditary} Spell of its army.
\end{itemize}

\section{Declare Special Deployment}
\label{declare_special_deployment}

Starting with the player that picked their Deployment Zone, each player must nominate which units with Special Deployment options (such as \hyperref[scout]{\scout{}} or \hyperref[ambush]{\ambush{}}) will use their Special Deployment, or if they will deploy using the normal rules.