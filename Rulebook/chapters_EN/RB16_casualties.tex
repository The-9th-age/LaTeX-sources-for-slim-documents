
\part{Casualties}
\idx[main=y]{Casualties}\label{casualties}

Models suffering unsaved wounds lose Health Points.

\section{Losing Health Points}
\idx[main=y]{Losing Health Points}\label{losing_health_points}

\subsection{\rnf{} Models}
\idx{Health Pools}

\rnf{} models except Champions in the same unit share a common Health Pool. If the attack was allocated towards or distributed onto a \rnf{} model, the combined \rnf{} Health Pool loses 1 Health Point for each unsaved wound. If the \rnf{} models have 1 Health Point each, remove one \rnf{} model for each Health Point lost.

If the \rnf{} models have more than 1 Health Point each, remove whole \rnf{} models whenever possible. Keep track of Health Points lost from the Health Pool that are not enough to remove an entire model. These lost Health Points are taken into account for future attacks. For example, a unit of 10 Trolls (3 Health Points each) loses 7 Health Points. Remove two whole models (6 Health Points), leaving 1 lost Health Point which is kept track of. Later, this unit loses 2 Health Points, which is enough to remove a single Troll since 1 Health Point was lost from the previous attack.

If all non-Champion \rnf{} models in a unit are wiped out, any excess lost Health Points are allotted to the Champion (even if it is fighting a Duel). If there is no Champion, the excess Health Point losses are ignored.

If a unit consists of \rnf{} models with different Types and/or Heights, all \rnf{} models with the same Type and Height have their own separate Health Pool.

\subsection{Champions}
\idx{Champions}

Even though Champions are \rnf{} models, each Champion has its own Health Pool, and follows the rules for Characters below. If enough Health Points are lost by \rnf{} models in order to wipe out the entire unit, any remaining lost Health Points are allotted to the Champion (even if it is fighting a Duel).

\subsection{Characters}

If the attack was allocated towards or distributed onto a Character, the attacked model loses 1 Health Point for each unsaved wound. If the model reaches 0 Health Points, it is removed as a casualty. Keep track of models that have lost Health Points, but not enough to reach 0 Health Points (placing \enquote{Health Point markers} next to such models works fine). These lost Health Points will be taken into account for future attacks. If the model is removed as a casualty, any excess Health Point losses are ignored.

\subsection{Excess Health Point Losses}
\idx[main=y]{Excess HP Losses}

Whenever more Health Point losses are inflicted than there are Health Points in a Health Pool, these excess Health Point losses are ignored.

When caused by simultaneous attacks from models from two or more Health Pools and/or units, it may be necessary to determine which models caused the excess Health Point losses. In this case, the owner of the models that inflicted the Health Point losses gets to decide.

\subsection{Losing the Last Health Point}
\idx[main=y]{Losing the Last Health Point}\idx[main=y]{Removing as a casualty}

Certain effects are triggered by models being removed as a casualty, while others are set off by models losing their last Health Point or reaching 0 Health Points. Note that losing the last Health Point does not apply to situations in which a model is directly removed as a casualty, without actually losing any Health Points, like Fleeing off the board or being destroyed after Breaking from Combat.


\section{Removing Casualties}
\idx[main=y]{Removing Casualties}\label{removing_casualties}

Whenever the rules tell you to remove models as casualties, remove the models from the Battlefield following the rules below. Models that have been removed as casualties no longer affect the game in any way, but they may award Victory Points to the opponent (see \totalref{victory_conditions}).

\subsection{Removing \rnf{} Models}

If the unit is in multiple ranks, \rnf{} casualties are removed from the rear rank by the owner, in any order they choose.

If the unit is in a single rank, remove models as equally as possible from both sides of the unit. Note that this only applies to each batch of simultaneous attacks.

If a Champion or Character is in a position that would normally be removed as a casualty, remove the next eligible \rnf{} model and slide the Champion and/or Character model(s) into the now empty spot.

\subsection{Removing \rnf{} Models from Units Engaged in Combat}

The removal of casualties from Engaged units follows the general rules for Removing \rnf{} models above. In addition, if the unit is in a single rank, remove casualties from either side of the unit, so that the following conditions are satisfied as best as possible for each batch of simultaneous casualties, in decreasing priority order:
\begin{itemize}
\item \nth{1} priority: As few units as possible Drop out of Combat (see \totalref{losing_base_contact})
\item \nth{2} priority: As few units as possible lose base contact without Dropping out of Combat
\item \nth{3} priority: The number of models in base contact is maximised after nudging all units
\item \nth{4} priority: Casualties are removed as equally as possible from both sides of the unit
\end{itemize}

If it is unavoidable to break one or more of the above conditions, you must avoid breaking the higher priority order conditions, even if this means the total number of conditions you break is higher. As long as all above conditions are satisfied as best as possible, the owner is free to remove casualties as they please. See figure \ref{figure/removing_rnf_models} for examples.


\newcommand{\figRRMChar}{\fontsize{7}{8}\selectfont C}

\begin{figure}[!htbp]
	\renewcommand{\figbiglettersize}{16}
	\centering
	\begin{minipage}{0.55\textwidth}
	\def\svgwidth{\textwidth}
	\input{pics/removing_rnf_models.pdf_tex}
	\end{minipage}\hfill\begin{minipage}{0.43\textwidth}
	\caption{Removing \rnf{} models from units Engaged in Combat.\captionpar
	This figure shows how models are removed as casualties from a unit that is Engaged with one or more enemy units according to Removing \rnf{} Models from Units Engaged in Combat. In all examples, 3 models are removed as casualties from the green Combined Unit containing a Character in its first rank.\captionpar
	a) One of these casualties is the model in the second rank, and the other two have to be removed from both sides of the first rank according to the \nth{4} priority. Since there is a Character on the right side of the unit, the \rnf{} model to its left is removed instead, and the Character is slid into the removed model's spot.\captionpar
	b) In order to maximise the number of models in base contact (\nth{3} priority), the model in the second rank and the two rightmost \rnf{} models from the first rank are removed as casualties. The Character is slid into the spot of a removed \rnf{} model.\captionpar
	c) In order not to have unit B lose base contact by removing casualties (\nth{2} priority), the model in the second rank and the two rightmost \rnf{} models from the first rank are removed as casualties. The Character is slid into the spot of a removed \rnf{} model.\captionpar
	d) In order not to cause any units to Drop out of Combat (\nth{1} priority), the model in the second rank and the two leftmost \rnf{} models from the first rank are removed as casualties. Unit B loses contact but is nudged back into combat (see \totalref{losing_base_contact}).\captionpar
	e) In order to cause as few units as possible to Drop out of Combat (1st priority), the model in the second rank and the two rightmost \rnf{} models from the first rank are removed as casualties. The Character is slid into the spot of a removed \rnf{} model. Unit D loses contact and cannot be nudged back into combat, so that it Drops out of Combat (see \totalref{losing_base_contact}).\captionpar
	}
	\label{figure/removing_rnf_models}
	\end{minipage}
\end{figure}

 \subsection{Removing Champions and Characters}
 \label{removing_non_RnF_models}
 
When Champions and Characters are removed as casualties they are removed from their positions within the unit. Other models are then moved to fill empty spots, following the same guidelines as for casualty removal above. When removing casualties from unengaged units with a single rank, Champions and Characters follow the rules for Matching Bases (see \totalref{front_rank}).
