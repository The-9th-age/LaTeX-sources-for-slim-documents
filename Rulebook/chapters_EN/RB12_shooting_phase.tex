\part{Shooting Phase}
\label{shooting_phase}

In the Shooting Phase, models with Shooting Attacks get a chance to use them.

\section{Shooting Phase Sequence}
\label{shooting_phase_sequence}

The Shooting Phase is divided into the following steps.

\hspace*{0.3cm}
\begin{tabular}{c | m{14cm}}
1 & Start of the Shooting Phase.\\
2 & Select one of your units and perform a Shooting Attack.\\
3 & Repeat step 2 with a different unit that has not fired during this phase yet.\\
4 & When all units that can (and want to) shoot have done so, the Shooting Phase ends.\\
\end{tabular}

\subsection{Shooting With a Unit}
\label{shooting_with_a_unit}

Each unit (with a Shooting Weapon or Shooting Attack) can shoot once per Shooting Phase. Units that are Engaged in Combat, Fleeing, \hyperref[shaken]{Shaken}, or that have Marched or Reformed cannot shoot.
\par
When a unit shoots, first nominate as target an enemy unit that is within the shooting unit's Line of Sight. Friendly units and units that are Engaged in Combat cannot be chosen as targets. All models in the same unit must shoot at the same target and \textbf{only models in the first and second rank may fire}. If the models in the unit have more than one type of Shooting Weapon or Attack, declare which is used. All \rnf{} models except Champions must use the same type of weapon (or attack). Champions and Characters are free to use other types of Shooting Weapons or Attacks if they wish (still maximum one attack per model, and directed at the same target as the unit). In case of multipart models, each model part can make a Shooting Attack in the same phase. Any model in the unit is free to choose not to shoot.
\par
Check the Line of Sight for each shooting model. Remember that Line of Sight is always drawn from the front of the base. Models that do not have a Line of Sight to at least one model in the target unit cannot shoot. Measure the range to the target unit for each individual shooting model. This is measured from the actual position of each shooting model to the closest point of the target unit (even if this particular point is not within Line of Sight). Models that are further away from their target than the range of their weapon cannot shoot (unless this is a Stand and Shoot Charge Reaction). If the weapon has a minimum range, the model can only shoot if the target is at least partly outside the minimum range.
\par
Once it has been established which models can shoot, these models shoot as many times as indicated in their weapon's profile. For each shot, roll to hit with each model, as described below.

\section{Aim}
\label{aim}

All Shooting Weapons have an Aim written in brackets after the weapon's name. The Aim tells you what the model needs to roll on a D6 to successfully hit its target. This roll is called a \textbf{to-hit roll}. Note that the Aim is not bound to the weapon, instead each unit has its own Aim for a given Shooting Weapon available to it. For example, an elven archer might have a Longbow (3+) while a human peasant only has a Longbow (4+). The elf would hit its target if it rolls 3 or higher on a D6, while the human would need to roll 4 or higher.

\section{To-Hit Modifiers}
\label{to_hit_modifiers}

Shooting Attacks may suffer one or more to-hit modifiers to their to-hit rolls. If so, simply modify the dice roll for the shot with the given modifiers. A natural roll of \result{1} is always a miss. If one or more hits are scored, follow the procedure described under \totalref{attacks}. The most common to-hit modifiers are summarised in table \ref{table/to_hit_modifiers}.

\vspace*{10pt}
\begin{table}[!hbtp]
\centering
\begin{tabular}{p{4cm} c p{0.5cm} p{4cm} c}
\toprule
Long Range & -1 & & Stand and Shoot &-1 \\
\hspace*{0.3cm}(if \hyperref[accurate]{Accurate}) & 0 & & Soft Cover & -1 \\
Moving and Shooting & -1 & & Hard Cover & -2 \\
\hspace*{0.3cm}(if \hyperref[quick_to_fire]{Quick to Fire}) & 0 & & \hyperref[hard_target]{Hard Target} & -1 \\
\hspace*{0.3cm}(if \hyperref[unwieldy]{Unwieldy}) & -2 & &  &  \\
\hspace*{0.3cm}(if both) & -1 & &  &  \\
\bottomrule
\end{tabular}
\caption{To-Hit Modifiers Summary.}
\label{table/to_hit_modifiers}
\end{table}

\subsection{Long Range (-1 to hit)}
\label{long_range}

If the target is further away than half of the weapon's range, the shooting model receives a -1 to-hit modifier. Remember that you measure range for each shooting model individually.\newline
Any model not shooting at Long Range is considered to be at Short Range.

\subsection{Moving and Shooting (-1 to hit)}
\label{moving_and_shooting}

A model that has moved during this Player Turn receives a -1 to-hit modifier.

\subsection{Stand and Shoot Charge Reaction (-1 to hit)}
\label{stand_and_shoot_charge_reaction}

Shooting Attacks made as part of a Stand and Shoot Charge Reaction receive a -1 to-hit modifier.

\subsection{Cover}
\label{cover}

Cover is determined individually for each shooting model. There are two types of Cover: Soft Cover and Hard Cover. The most common reason for applying a Cover type is the target being obscured by Terrain or other models, or the target being inside a Terrain Feature.

To determine if the target is obscured, draw a line from the shooting model's Front Facing to the target's Footprint, tangent to the possible obstruction. The line may be drawn outside of the model's Front Arc. Any part of the target unit's Footprint on the same side of the line as the possible obstruction is considered obscured. The line can be drawn from any point of the Front Facing of the shooting model, but only consider the point that yields the least possible obscurity. Models on round bases can use any point on their base for this purpose (since they don't have a Front Facing).

Models always ignore their own unit and the target unit for Cover purposes, and also ignore whatever Terrain Feature they are inside (e.g. a unit shooting from a Forest doesn't suffer a Soft Cover modifier for shooting through that Forest). If the target contains models of different Sizes, consider the Footprint of each model individually.

\paragraph{Target Benefiting from Soft Cover (-1 to hit)}

A model shooting at a target benefiting from Soft Cover receives a -1 to-hit modifier. Soft Cover applies if more than half of the target unit's Footprint is: 
\begin{itemize}[label={-}]
\item obscured by one or more of the following Terrain Features: Forests and Hills (only if the target is partially on the Hill),
\item inside of any of the following Terrain Features: Forests and Fields,
\item or obscured by models that \textbf{do not} block Line of Sight, except if either the target and/or the shooting model is of Gigantic Size (see \totalref{model_classification}), and the obscuring model is of Standard Size (in which case no cover is applied) (remember that \hyperref[skirmisher]{Skirmisher} and \hyperref[tall]{Tall} affect what blocks Line of Sight).
\end{itemize}

See figure \ref{figure/soft_cover} for an example with Terrain and figure \ref{figure/soft_cover_and_intervening_models} for examples with intervening models.

\paragraph{Target Benefiting from Hard Cover (-2 to hit)}

A model shooting at a target benefiting from Hard Cover receives a -2 to-hit modifier. Hard Cover applies if more than half of the target unit's Footprint is:
\begin{itemize}[label={-}]
\item obscured by one or more of the following Terrain Features: Buildings, Cliffs, Defended Walls, and Hills (only if the target is entirely off the Hill),
\item inside of any of the following Terrain Features: Ruins and Building,
\item or obscured by models that \textbf{do} block Line of Sight (remember that \hyperref[skirmisher]{Skirmisher} and \hyperref[tall]{Tall} affect what blocks Line of Sight).
\end{itemize}

See figure \ref{figure/hard_cover} for an example with Terrain.
 
\paragraph{Target Benefiting from Hard and Soft Cover}

If a target is benefiting from both Hard Cover and Soft Cover, only apply the Hard Cover modifier. If the target unit's Footprint is obscured by obstacles that contribute to Hard and Soft Cover, but not enough to grant either Hard Cover or Soft Cover, apply only the Soft Cover modifier if more than half of the target unit's total Footprint is obscured. For example, if \SI{30}{\percent} of the Footprint is obscured by Terrain contributing to Soft Cover, and another \SI{30}{\percent} by Terrain contributing to Hard Cover, then apply the Soft Cover modifier as it has \SI{60}{\percent} of its Footprint obscured in total (see figure \ref{figure/soft_and_hard_cover}).

\section{Hopeless Shots}
\label{hopeless_shots}

When to-hit modifiers make the needed roll to successfully hit with a Shooting Attack 7+, apply the following procedure: rolls of \result{6} are considered successful. For each successful roll, roll to hit again. On a roll of 4+, this second to-hit roll is successful, and the shot hits - proceed as described under \totalref{attacks}. If there are enough modifiers to make the needed roll to successfully hit 8 or more, the shot cannot hit.

For example, a model with Bow (4+) shoots at a target benefiting from Hard Cover (-2 to hit), and is Moving and Shooting (-1). This would require the shooter to roll 7+ on a D6, which means that this shot follows the Hopeless Shots rule. If a \result{6} is rolled, roll to-hit again. If the shooter manages to roll 4+ on the second attempt, the shot hits.

\newcommand{\figureSCNotwithinlightofsight}{\smallfontsize{Not within Line of Sight}}
\newcommand{\figureSCA}{a)}
\newcommand{\figureSCB}{b)}
\newcommand{\figureSCC}{c)}
\newcommand{\figureSCWithinlightofsight}{\smallfontsize{Within Line of Sight}}
\newcommand{\figureSCLessthanhalfoffootprintobscured}{%
\begin{minipage}{0.105\unitlength}\begin{center}%
\smallfontsize{Less than half of Footprint obscured}%
\end{center}\end{minipage}}
\newcommand{\figureSCMorethanhalfoffootprintobscured}{%
\begin{minipage}{0.105\unitlength}\begin{center}%
\smallfontsize{More than half of Footprint obscured}%
\end{center}\end{minipage}}

\vspace*{10pt}

\begin{figure}[!htbp]
\centering
\def\svgwidth{\textwidth}
\input{pics/soft_cover.pdf_tex}
\caption{Example of Soft Cover.\captionpar
a) The model cannot shoot, enemy is not within Line of Sight.\captionpar
b) The model can shoot (enemy within Line of Sight). No Cover since half or less of the enemy unit's Footprint is obscured.\captionpar
c) The model can shoot (enemy within Line of Sight). Soft Cover since more than half of the enemy unit's Footprint is obscured by the Forest.}
\label{figure/soft_cover}
\end{figure}

\newcommand{\figureHCSizeGreen}{\normalfontsize{Size: Standard}}
\newcommand{\figureHCSizeBlue}{\normalfontsize{Size: Standard}}
\newcommand{\figureHCSizePurple}{\normalfontsize{Size: Standard}}
\newcommand{\figureHCNotwithinlightofsight}{%
\begin{minipage}{0.08\unitlength}\begin{center}%
\smallfontsize{Not within Line of Sight}%
\end{center}\end{minipage}%
}
\newcommand{\figureHCWithinlightofsight}{\smallfontsize{Within Line of Sight}}
\newcommand{\figureHCLessthanhalfoffootprintobscured}{%
\begin{minipage}{0.08\unitlength}\begin{center}%
\smallfontsize{Less than half of Footprint obscured}%
\end{center}\end{minipage}%
}
\newcommand{\figureHCMorethanhalfoffootprintobscured}{%
\begin{minipage}{0.08\unitlength}\begin{center}%
\smallfontsize{More than half of Footprint obscured}%
\end{center}\end{minipage}%
}

\begin{figure}[!htbp]
\centering
\def\svgwidth{\textwidth}
\input{pics/hard_cover.pdf_tex}
\caption{Example of Hard Cover.\captionpar
a) The model cannot shoot (blocked Line of Sight).\captionpar
b) The model can shoot (enemy within Line of Sight). Hard Cover since more than half of the enemy unit's Footprint is obscured by a model that blocks Line of Sight.\captionpar
c) The model can shoot (enemy within Line of Sight). No Cover since half or less of the enemy unit's Footprint is obscured by a model that blocks Line of Sight.}
\label{figure/hard_cover}
\end{figure}

\newcommand{\figureSHCSizeGreen}{\normalfontsize{Size: Large}}
\newcommand{\figureSHCSizeBlue}{\normalfontsize{Size: Standard}}
\newcommand{\figureSHCSizePurple}{\normalfontsize{Size: Standard}}
\newcommand{\figureSHCLessthanhalffromhardcover}{\normalfontsize{Less than half of Footprint in Hard Cover}}
\newcommand{\figureSHCLessthanhalffromsoftcover}{%
\begin{minipage}{0.3\unitlength}\begin{center}%
\normalfontsize{Less than half of Footprint in Soft Cover}%
\end{center}\end{minipage}%
}
\newcommand{\figureSHCMorethanhalftotal}{\normalfontsize{{More than half total}}}

\begin{figure}[!htbp]
\begin{minipage}[t]{0.55\textwidth}
\def\svgwidth{\textwidth}
\input{pics/soft_and_hard_cover.pdf_tex}
\end{minipage}\hfill\begin{minipage}[b]{0.42\textwidth}
\caption{Example with Soft and Hard Cover.\captionpar
In this example, less than half of the target unit's Footprint is obscured by obstacles contributing either to Hard or Soft Cover. However, more than half is obscured by the combination of both. In this case, the target counts as benefiting from Soft Cover.}
\label{figure/soft_and_hard_cover}
\end{minipage}
\end{figure}

\newcommand{\figureLoSCSoftcover}{\Largefontsize{Soft Cover (-1 to hit)}}
\newcommand{\figureLoSCNocover}{\Largefontsize{No Cover}}
\newcommand{\figureLoSCStandard}{Standard}
\newcommand{\figureLoSCLarge}{Large}
\newcommand{\figureLoSCGigantic}{Gigantic}

\begin{figure}[!htbp]
\centering
\def\svgwidth{\textwidth}
\input{pics/soft_cover_and_intervening_models.pdf_tex}
\caption{Soft Cover and intervening models.\captionpar
This diagram shows all possible Size combinations between shooting, target, and intervening models that yield a Soft Cover or No Cover result. The intervening model is assumed to be placed in such a way that it is sufficiently obscuring the target from the shooter. All other Size combinations would yield either Hard Cover or no Line of Sight, depending on whether the target is completely obscured by the intervening model or not.}
\label{figure/soft_cover_and_intervening_models}
\end{figure}
