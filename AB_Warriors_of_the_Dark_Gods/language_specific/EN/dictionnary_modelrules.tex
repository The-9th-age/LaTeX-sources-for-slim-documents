% Army Model Rules Names

\newcommand{\favoursofthedarkgods}{Favours of the Dark Gods}

\newcommand{\envy}{Envy}
\newcommand{\envyfulltitle}{Favour of Kuulima, Goddess of \envy{}}
\newcommand{\gluttony}{Gluttony}
\newcommand{\gluttonyfulltitle}{Favour of Akaan, God of \gluttony{}}
\newcommand{\greed}{Greed}
\newcommand{\greedfulltitle}{Favour of Sugulag, God of \greed{}}
\newcommand{\lust}{Lust}
\newcommand{\lustfulltitle}{Favour of Cibaresh, God of \lust{}}
\newcommand{\pride}{Pride}
\newcommand{\pridefulltitle}{Favour of Savar, God of \pride{}}
\newcommand{\sloth}{Sloth}
\newcommand{\slothfulltitle}{Favour of Nukuja, Goddess of \sloth{}}
\newcommand{\wrath}{Wrath}
\newcommand{\wrathfulltitle}{Favour of Vanadra, Goddess of \wrath{}}

\newcommand{\pathfavoured}{Path of the Favoured}
\newcommand{\pathexiled}{Path of the Exiled}
\newcommand{\irredeemable}{Irredeemable}
\newcommand{\veilwalker}{Veil Walker}
\newcommand{\battlefever}{Battle Fever}
\newcommand{\unburnt}{Unburnt}
\newcommand{\warbandstandard}{Warband Standard}% deprecated
\newcommand{\trophyrack}{Trophy Rack}
\newcommand{\hellforgedarmor}{Hell-Forged Armour}
\newcommand{\spikedshield}{Spiked Shield}

% Army Model Rules Texts

\newcommand{\favoursofthedarkgodsintro}{A Character with a Favour cannot join a unit that contains any models with a different Favour than the Character.}

\newcommand{\envydef}{%
The model gains \textbf{\swiftstride}. In addition, a Charging model part with this Attack Attribute must reroll any natural to-hit rolls of \result{1}. Units with all of their models with this Favour must reroll any natural rolls of \result{1} when rolling for Charge Range.%
}

\newcommand{\gluttonydef}{%
The first time a model with this Favour successfully Charges a Fleeing unit, or is on the winning side of a combat and does not Pursue or Overrun, its model parts with this Favour gain +1 Strength on all their Close Combat Attacks (the effect lasts for the duration of the game).%
}

\newcommand{\greeddef}{%
The bearer gains \gw{}, \halberd{}, \pw{}, and \textbf{\weaponmaster}. A Character with this Favour has its Special Item allowance increased by 50 pts.
\columnbreak% You may place this where you want the column break to balance the 2 columns. If you do not put it, it will be done automatically.
}

\newcommand{\lustdef}{%
The model gains \textbf{\strider{}}. In addition, units with more than half of their models with this Favour are subject to the following rules:%
\begin{itemize}
\item They gain \textbf{\feignedflight{}}.
\item They may declare Flee as a Charge Reaction even if they have Fearless.
\item Their Rally Test after voluntarily declaring Flee as a Charge Reaction is subject to Minimised Roll.
\end{itemize}%
}

\newcommand{\pridedef}{%
Discipline Tests taken by units with at least one model with this Favour are subject to Minimised Roll.%
}

\newcommand{\slothdef}{%
Models with this Favour gain +1 Resilience. If a model with this Favour declares a Charge against an enemy unit that is more than \distance{10}\refsymbol{} away or performs an Advance or March Move of more than \distance{10}\refsymbol{}, this effect is lost until the start of the Melee Phase in the next Player Turn.\newline%
\refsymbolbis{}These distances are decreased to \distance{6} if the model is Gigantic.%
}

\newcommand{\wrathdef}{%
The model part gains \textbf{\lightningreflexes} and +1 Agility. Close Combat Attacks allocated towards the model gain +1 to hit. These effects are only applied in the First Round of Combat.%
}

\newcommand{\pathfavoureddef}{%
Units with more than half of their models with Path of the Favoured must reroll failed Break Tests. In addition, model parts with Path of the Favoured upgraded to a Champion additionally gain +1 Health Point to a maximum of 3, and their Discipline is \textbf{set} to 9.%
}
\newcommand{\pathexileddef}{%
Units with more than half of their models with Path of the Exiled must reroll failed Break Tests. At the end of step 7 of a Round of Combat (after taking Break Tests), models with Path of the Exiled in a unit that failed a Break Test simultaneously perform Close Combat Attacks (ignoring the rules for Initiative Order, but otherwise following the normal rules such as Supporting Attacks and Allocating Attacks). Afterwards, they are removed as casualties. Models with Path of the Exiled cannot join or be joined by models with \pathfavoured{}.%
}
\newcommand{\irredeemabledef}{%
The model cannot make Stomp Attacks and, when in the second rank and not in base contact with any enemy models, can make \grindattacks{} across models in the first rank directly in front of it. When a model with \irredeemable{} is killed by a Melee Attack, remove it as a casualty only at the end of Initiative Step 0. A unit with at least one model with \irredeemable{} may never have more ranks than files.%
}
\newcommand{\veilwalkerdef}{%
When a model with \veilwalker{} casts a non-Bound Spell, you may discard a single Veil Token when declaring the target(s) of the spell and activate a single one of the following effects:
\begin{itemize}
\item \spellformat{Secret of Flesh}{}: Failed to-wound rolls from this spell that occur during a Magic Phase must be rerolled.
\item \spellformat{Secret of Separation}{}: The spell's range is increased by \distance{6}. Aura spells only gain +\distance{3} range. Spells with type Caster are unaffected.
\item \spellformat{Secret of Substance}{}: Successful Armour Saves against wounds caused by this spell's effect must be rerolled.
\end{itemize}%
}
\newcommand{\battlefeverdef}{%
Units with more than half of their models with \battlefever{} must reroll failed Panic and Break Tests.%
}
\newcommand{\unburntdef}{%
Flaming Attacks made against the model must reroll successful to-wound rolls. In addition, the model considers all units consisting entirely of models without Unburnt as Insignificant.%
}
\newcommand{\trophyrackdef}{%
The bearer's unit may reroll failed Discipline Tests, unless Fleeing. Each time attacks made the bearer's model kill an enemy model in a Duel, the bearer's model gains a +1 Combat Score modifier for the rest of the game (this also applies to attacks made outside the Melee Phase). In addition, the bearer's model may take a single Banner Enchantment (using the bearer's Special Item allowance as normal).%
}
\newcommand{\hellforgedarmordef}{%
Follows the rules for \pa{} (can be enchanted as if it was \pa{}). The wearer's model gains \textbf{\aegis{} (5+, against Toxic Attacks)}.%
}
\newcommand{\spikedshielddef}{%
	\itemrestriction{Models on foot only.}
	Follows the rules for Shields (can be enchanted as if it was a Shield). For each \textbf{successful} Armour Save roll of \textbf{4+} made by the bearer against a Melee Attack while using a Spiked Shield, the model that caused the wound immediately suffers 1 hit with the bearer's \Strength{} and \ArmourPenetration{}, before any casualties are removed, distributed onto the model's Health Pool. This is considered a Special Attack.%
}