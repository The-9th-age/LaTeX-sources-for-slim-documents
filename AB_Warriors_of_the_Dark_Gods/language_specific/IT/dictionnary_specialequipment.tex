% Special Equipment Names

\newcommand{\burningportent}{Portento in fiamme} 
\newcommand{\symbolofslaughter}{Simbolo del massacro} 

\newcommand{\thriceforged}{Triforgiata}
\newcommand{\gladiatorsspirit}{Spirito del gladiatore} 

\newcommand{\iconoftheinfinite}{Icona dell'infinito} 
\newcommand{\bannerofthevoid}{Stendardo del nulla} 
\newcommand{\wastelandtorch}{Torcia della Desolazione} 

\newcommand{\ledgerofsouls}{Grimorio delle anime} 
\newcommand{\orbofforeboding}{Orbe profetico} 
\newcommand{\immortalgauntlets}{Guanti immortali} 
\newcommand{\lordofthedamned}{Signore della dannazione} 
\newcommand{\wyrdstone}{Pietra dell'idolatria} 

% Special Equipment Texts

\newcommand{\burningportentdef}{%
\enchantmentrestriction{Incanto di \hw{}.}
Gli Attacchi con questa arma hanno la propria Penetrazione armatura \textbf{fissata} a 10 e ottengono \textbf{\flamingattacks{}}, \textbf{\magicalattacks{}}, e \textbf{\multiplewounds{D3}{}}.
}
\newcommand{\symbolofslaughterdef}{%
\enchantmentrestriction{Incanto di \hw{} e \pw{}.}
Mentre usa questa arma, il portatore ottiene +2 Numero di attacchi e +2 Agilità, e subisce -2 Abilità offensiva. Gli attacchi effettuati con questa arma ottengono \textbf{\magicalattacks{}}.
}

\newcommand{\thriceforgeddef}{%
\itemrestriction{Solo modelli di Taglia Normale.}
\enchantmentrestriction{Incanto di Corazza.}
Il portatore ottiene +3 Armatura.
}
\newcommand{\gladiatorsspiritdef}{%
\enchantmentrestriction{Incanto di \spikedshield{}.}
I colpi causati dallo \spikedshield{} ottengono +1 Forza e +1 Penetrazione armatura.
}

\newcommand{\iconoftheinfinitedef}{%
\itemrestriction{0-2 per Esercito.}
Il portatore può lanciare \spellformat{\hellfire}{\hereditaryspell} come Incantesimo infuso con Livello di potere (4/8).
}
\newcommand{\bannerofthevoiddef}{%
\itemrestriction{Solo Prescelti, Prescelti su karkadan, e Signore prescelto.}
L'unità del portatore ottiene \textbf{\aegis{} (5+, contro Attacchi a distanza).}
}
\newcommand{\wastelandtorchdef}{%
L'unità del portatore ottiene \textbf{Viaggiatore (Rovine)}. Dopo aver determinato le Zone di schieramento (alla fine del passo 6 della Sequenza prepartita) puoi scegliere un singolo Elemento di terreno Edificio, Campo, o Foresta. I Campi e le Foreste scelti diventano Rovine, e gli Edifici scelti diventano Rupi. 
}

\newcommand{\ledgerofsoulsdef}{%
\itemrestriction{Dominante.}
Il portatore ottiene \textbf{\thereckoning{}} (vedi Altare da battaglia).
}
\newcommand{\lordofthedamneddef}{%
\itemrestriction{Solo \sorcerer{}.}
Qualsiasi unità con \irredeemable{} entro \distance{18} dal portatore può ripetere i tiri per la distanza di cui si sposta nella Fase di movimento tramite \randommovement{}.
}
\newcommand{\immortalgauntletsdef}{%
All'inizio di un qualunque Round di combattimento puoi scartare un Segnalino velo dalla tua riserva. In tal caso, scegli tra \divineattacks{}, \flamingattacks{}, o \magicalattacks{}. Gli Attacchi da corpo a corpo del portatore ottengono l'Attributo d'attacco scelto. Gli effetti durano fino alla fine della fase.
}
\newcommand{\orbofforebodingdef}{%
All'inizio della Fase di schieramento, subito prima del passo 1 (Determina chi schiera per primo) puoi scegliere una delle unità nemiche non Personaggio. Questa unità perde \vanguard{} ed \scout{}. L'avversario deve schierare immediatamente tale unità. Ciò avviene al di fuori della normale procedura di Schieramento e viene ignorato nel determinare l'unità schierata per prima e il numero di unità schierate.
}
\newcommand{\wyrdstonedef}{%
Un solo uso. Deve essere attivata quando il modello portatore è colpito da un attacco con almeno Forza 5, o con almeno Penetrazione armatura 5. Il modello del portatore ottiene \textbf{\aegis{4+}}.\newline 
Se il modello del portatore è Normale o Grande, l'effetto termina alla fine del Turno del giocatore. \newline 
Se il modello del portatore è Gigantesco, l'effetto termina dopo aver tirato il primo Tiro ègida generato dalla Pietra dell'idolatria (se il portatore è colpito contemporaneamente da più attacchi, il proprietario sceglie contro quale attacco usare il tiro \aegis{}).
}