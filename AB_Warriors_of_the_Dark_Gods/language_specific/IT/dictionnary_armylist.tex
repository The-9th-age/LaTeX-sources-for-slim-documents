
%%%% Unit entries names

% CHARACTERS

\newcommand{\exaltedherald}{Araldo esaltato} 

\newcommand{\chosenlord}{Signore prescelto} 

\newcommand{\doomlord}{Signore dei dannati} 

\newcommand{\sorcerer}{Stregone} 

\newcommand{\barbarianchief}{Capo barbaro} 

\newcommand{\feldrakancestor}{Feldrak ancestrale} 

% MOUNTS

\newcommand{\blacksteed}{Destriero nero} 

\newcommand{\scythedskywheel}{Disco falcato} 

\newcommand{\darkchariot}{Carro oscuro} 

\newcommand{\wardais}{Portantina da guerra} 

\newcommand{\karkadan}{Karkadan} 

\newcommand{\wastelanddragon}{Drago della Desolazione} 

\newcommand{\shadowchaser}{Cacciatore d'ombra} 

\newcommand{\chimera}{Chimera} 

\newcommand{\wastelandbehemoth}{Behemot della Desolazione} 


% CORE

\newcommand{\warrior}{Guerriero} 
\newcommand{\warriors}{Guerrieri} 

\newcommand{\fallen}{Caduti} 
\newcommand{\fallenSINGULAR}{Caduto} 

\newcommand{\barbarian}{Barbaro} 
\newcommand{\barbarians}{Barbari} 

% SPECIAL


\newcommand{\warriorknights}{Guerrieri a cavallo} 
\newcommand{\warriorknight}{Guerriero a cavallo} 
\newcommand{\warriorrider}{Guerriero}
\newcommand{\warriorriders}{Guerrieri}

\newcommand{\warriorchariot}{Carro dei guerrieri} 
\newcommand{\chassis}{Telaio} 
\newcommand{\warriorcrew}{Guerrieri} 

\newcommand{\chosen}{Prescelti} 
\newcommand{\chosenSINGULAR}{Prescelto} 

\newcommand{\chosenknights}{Prescelti su karkadan} 
\newcommand{\chosenknight}{Prescelto su karkadan} 
\newcommand{\chosenrider}{Prescelto}
\newcommand{\chosenriders}{Prescelti}

\newcommand{\chosenchariot}{Carro dei prescelti} 
\newcommand{\chosencrew}{Prescelti} 

\newcommand{\forsworn}{Rinnegati} 
\newcommand{\forswornSINGULAR}{Rinnegato} 

\newcommand{\wretchedones}{Abietti} 
\newcommand{\wretchedone}{Abietto} 

\newcommand{\battleshrine}{Altare da battaglia} 
\newcommand{\shrinepriest}{Sacerdote dell'altare} 

\newcommand{\barbarianhorsemen}{Barbari a cavallo} 
\newcommand{\barbarianhorseman}{Barbaro a cavallo} 

\newcommand{\flayers}{Scorticatori} 
\newcommand{\flayer}{Scorticatore} 

\newcommand{\warhounds}{Segugi da guerra} 
\newcommand{\warhound}{Segugio da guerra} 

\newcommand{\feldraks}{Feldrak} 
\newcommand{\feldrak}{Feldrak} 


% LEGENDARY BEASTS

\newcommand{\forsakenone}{L'obliato} 

\newcommand{\maraudinggiant}{Gigante saccheggiatore} 

\newcommand{\feldrakelder}{Feldrak antico} 

\newcommand{\hellmaw}{Fauci d'inferno} 


%%%% Unit entries rules

%% CHARACTERS

% Exalted Herald

\newcommand{\manifestation}{Manifestazione} 
\newcommand{\manifestationdef}{Durante la Selezione degli incantesimi, ciascun Araldo esaltato deve scegliere due Manifestazioni dalla lista in basso, e applicare gli effetti per tutta la partita. Il modello conosce gli incantesimi indicati nella Manifestazione scelta. Ciò sostituisce le normali regole per la Selezione degli incantesimi dei Maghi adepti o maestri.} 

\newcommand{\theexaltedheraldgains}{L'Araldo esaltato ottiene} 
\newcommand{\theexaltedheraldknows}{L'Araldo esaltato conosce} 

% The manifestation will be automatically sorted in alphabetical order.
\newcommand{\abidingspirit}{Spirito granitico} 
\newcommand{\abidingspiritrule}{\textbf{\hardtarget}. Alla fine di ogni Fase di mischia amica, se l'Araldo esaltato ha vinto un combattimento in tale fase, Recupera 1 Punto vita.} 
\newcommand{\abidingspiritspells}{\spellformat{\shamanismspellsix}{\shamanism}.} 

\newcommand{\brandofthedragon}{Marchio del drago} 
\newcommand{\brandofthedragonrule}{\textbf{\fly{8}{16}}.} 
\newcommand{\brandofthedragonspells}{\spellformat{\thaumaturgyspellone}{\thaumaturgy}.} 

\newcommand{\emissaryofchaos}{Emissario del caos} 
\newcommand{\emissaryofchaosrule}{+1 Disciplina, \textbf{\terror}, e \textbf{\toweringpresence}.} 
\newcommand{\emissaryofchaosspells}{\spellformat{\alchemyspellsix}{\alchemy}.} 

\newcommand{\sorcererimmortal}{Stregone immortale} 
\newcommand{\sorcererimmortalrule}{\textbf{\veilwalker}.} 
\newcommand{\sorcererimmortalspells}{\spellformat{\alchemyspellthree}{\alchemy} e \spellformat{\thaumaturgyspellfive{}}{\thaumaturgy{}}. Può sostituire uno dei suoi Incantesimi appresi con l'Incantesimo ereditario \spellformat{\hellfire}{\hereditaryspell} durante la Selezione degli incantesimi.} 

\newcommand{\unholyavatar}{Empia incarnazione}
\newcommand{\unholyavatarrule}{+1 Forza, +1 Penetrazione armatura e \textbf{\divineattacks}.} 
\newcommand{\unholyavatarspells}{\spellformat{\shamanismspellone}{\shamanism}.} 

% Chosen Lord

\newcommand{\allowanceifgeneral}{\suboptionindent{}Se è il Generale} 
\newcommand{\maytakegifts}{Un singolo \textbf{Dono degli dei oscuri}} 
\newcommand{\mustchoosefavour}[1]{\optionschoiceTWOCOL{Deve scegliere un Favore:}{\AlOrder{#1}}} 
\newcommand{\replaceshieldwithspikedshield}{Sostituisci Scudo con Scudo chiodato}
\newcommand{\withsloth}{Con Favore dell'Accidia}
\newcommand{\withoutsloth}{Senza Favore dell'Accidia}
\newcommand{\giftsofdarkgodsnote}{\textbf{Dono degli dei oscuri}. Ogni Dono è Unico.} 

\newcommand{\daemonicwings}{Ali demoniache} 
\newcommand{\daemonicwingsrule}{%
\itemrestriction{Solo modelli a piedi.}%
Il portatore ottiene \textbf{\fly{8}{16}}.% 
} 
\newcommand{\entropicaura}{Aura d'entropia} 
\newcommand{\entropicaurarule}{% 
\itemrestriction{Solo Taglia Normale e Grande.}% 
Gli Incanti d'arma e Incanti d'armatura del modello del portatore, dei modelli nell’unità del portatore e delle unità a contatto di base col portatore non possono essere utilizzati. \columnbreak
} 
\newcommand{\luckofthedarkgods}{Benedizione degli dei oscuri} 
\newcommand{\luckofthedarkgodsrule}{%
Il modello del portatore ottiene \textbf{\aegis{+1, max. 4+}}.%
} 
\newcommand{\idolofspite}{Idolo del rancore} 
\newcommand{\idolofspiterule}{% 
Un solo uso. Può essere attivato all'inizio di un Round di combattimento. Per la durata di questo Round di combattimento, il portatore ottiene +1 Numero di attacchi, +1 Forza, e +1 Penetrazione armatura.% 
} 
\newcommand{\masterofdestruction}{Signore della distruzione} 
\newcommand{\masterofdestructionrule}{% 
Il portatore può usare contemporaneamente uno \shield{} (o uno \spikedshield{}) con una \gw{} o un'\halberd{} (tuttavia non ottiene l’Attacco speciale dello \spikedshield{}).% 
} 

% Barbarian Chief

\newcommand{\deedsnotwords}{Fatti non parole} 
\newcommand{\deedsnotwordsdef}{Se si trova all’interno di un'unità di modelli \rnf{} con \battlefever{}, la parte del modello ottiene \textbf{\battlefocus}.} 

% Feldrak Ancestor

\newcommand{\primallegend}{Leggenda primeva} 
\newcommand{\primallegenddef}{Se il modello è il Generale, considera le unità di Taglia Normale come \insignificant{}, e il limite per Bestie leggendarie è incrementato a \SI{45}{\percent}. La Lista dell'esercito non può includere modelli con \fly{}, e nessun modello amico può scegliere \brandofthedragon{}. I modelli con \hellforgedarmor{} non possono ricevere \commandingpresence{} da modelli con \primallegend{}.} 

\newcommand{\dyingembers}{Tizzoni morenti} 
\newcommand{\dyingembersdef}{Dopo aver usato l'\breathattack{}, il modello perde un Punto vita (non sono concessi tiri protezione di alcun tipo).} 

\newcommand{\againstfly}{contro \fly{}} 

\newcommand{\specialequipmentfeldrak}{Un Incanto d'arma,\newline pagando il doppio dei punti} 

%% MOUNTS

% Black Steed

\newcommand{\barbarianchiefmusttakeprizedstallion}{Il \barbarianchief{} \textbf{deve} prendere \textbf{\prizedstallion}}
\newcommand{\prizedstallion}{Stallone}
\newcommand{\prizedstalliondef}{Il modello ha la propria Marcia \textbf{fissata} a \distance{16}.}

% Chimera

\newcommand{\wings}{Ali} 
\newcommand{\wingsdef}{Il modello ha la propria Marcia \textbf{fissata} a \distance{16} e ottiene \fly{8}{16}.} 

% Unbroken Chimera

\newcommand{\additionallimbs}{Arti supplementari} 
\newcommand{\additionallimbsdef}{Il modello ha la propria Marcia \textbf{fissata} a \distance{20} e la propria Armatura \textbf{fissata} a 3.} 

%% CORE

% Warriors

\newcommand{\favouredchampionmustchoosefavour}[2]{\optionschoiceTWOCOL{Se l'unità include un \favouredchampion{} \textbf{devi} scegliere un singolo Favore:}{\AlOrder{#2}}}
\newcommand{\unitsizereduction}{Dimensione max dell'unità ridotta a \textbf{20} modelli}
\newcommand{\upgradeto}{Promuovi a}

% Fallen

\newcommand{\fallennote}{\refsymbol{} \zerotoXchoiceperarmy{6} se un \doomlord{} è Generale.} 

%% SPECIAL

% Chosen

\newcommand{\mastersofbattle}{Signori della battaglia}
\newcommand{\mastersofbattledef}{%
Il modello può effettuare fino a 3 Attacchi di supporto.%
}

% Chosen Knights

\newcommand{\XXXmustchoosefavour}[2]{\optionschoiceTWOCOL{#1: \textbf{devi} scegliere un singolo Favore}{\AlOrder{#2}}}
\newcommand{\replacespikedshieldwithhalberd}{Sostituisci \spikedshield{} con \halberd{}}

% Battleshrine

\newcommand{\thereckoning}{Il giorno del giudizio}
\newcommand{\thereckoningdef}{%
Ogniqualvolta un modello amico con \hellforgedarmor{} o \irredeemable{} è rimosso come perdita, se la sua unità si trovava entro \distance{12} da un modello amico con \thereckoning, ottieni un Segnalino velo per ciascun Punto vita con cui il modello rimosso come perdita ha iniziato la partita. In aggiunta, alla fine di Incanalare il velo, l’esercito può conservare fino a 6 Segnalini velo (invece dei consueti 3).%
}

\newcommand{\piercetheveil}{Lacerare il velo} 
\newcommand{\piercetheveildef}{%
Alla fine della tua Fase di movimento, puoi scegliere un'unità nemica che sia o Ingaggiata con questo modello o non ingaggiata ed entro \distance{12} dal modello. Scarta fino a 3 Segnalini velo dalla tua riserva. Per ogni segnalino scartato, l'unità scelta subisce D3 colpi con Forza 4, Penetrazione armatura 0 e Attacchi magici.%
}

\newcommand{\seespecialequipment}{vedi Equipaggiamento speciale}

% Flayers

\newcommand{\skinninglash}{Fruste scuoiatrici} 
\newcommand{\skinninglashdef}{% 
Un'unità con almeno un modello con Fruste scuoiatrici può effettuare un Attacco di passaggio contro una singola unità nemica non ingaggiata se passa entro \distance{1} (non può né deve passare attraverso o sopra l'unità). Il nemico subisce 1 colpo con Forza 4 e Penetrazione armatura 0 per ciascun modello con Fruste scuoiatrici nell'unità. Un'unità che perde uno o più Punti vita a causa di un Attacco di passaggio con Fruste scuoiatrici subisce -1 Disciplina fino alla fine del suo successivo Turno del giocatore.% 
} 

% Warhounds

\newcommand{\releasethehounds}{Liberare i segugi} 
\newcommand{\releasethehoundsdef}{% 
Durante il primo Turno del giocatore proprietario, il modello ottiene \distance{+8} Marcia e \textbf{Carica devastante (+1 Att, +1 Off, +1 \St{}, +1 \AP{})}.% 
} 

% Chimera

\newcommand{\chimeracategorynote}{Il modello conta anche come \legendarybeasts{} nel caso prenda \textbf{\wings{}}.}

%% LEGENDARY BEASTS

% Marauding Giant

\newcommand{\giantseegiantdo}{Gigante vede, gigante fa} 
\newcommand{\giantseegiantdodef}{Il modello ottiene \textbf{\battlefever{}}.} 

\newcommand{\rage}{Rabbia} 
\newcommand{\ragedef}{Ogniqualvolta il modello perde un Punto vita, ottiene +1 Numero di attacchi. Ogniqualvolta ottiene un Punto vita, subisce -1 Numero di attacchi.} 

\newcommand{\monstrousfamiliar}{Famiglio mostruoso}
\newcommand{\monstrousfamiliardef}{%
Il modello ottiene \textbf{\wizardapprentice}. Durante la Selezione degli incantesimi, invece di procedere come di consueto seleziona un singolo incantesimo tra questi: \spellformat{\occultismspellfour}{\occultism}, \spellformat{\occultismspellone}{\occultism}, o \spellformat{\hellfire}{\hereditaryspell}.%
}

\newcommand{\giantclub}{Randello gigante} 
\newcommand{\giantclubdef}{Gli Attacchi con un \giantclub{} ottengono +1 Forza e +1 Penetrazione armatura.} 

\newcommand{\tribalwarspear}{Lancia tribale da guerra} 
\newcommand{\tribalwarspeardef}{Gli Attacchi con una \tribalwarspear{} ottengono +1 Forza e \textbf{Ferite multiple (D3, contro Presenza torreggiante)}. Le unità nemiche in carica a contatto di base  con il portatore subiscono -1 Agilità. Il portatore segue le regole per Piattaforma di guerra con la seguente eccezione: può aggregarsi solo a unità di Fanteria che includono almeno un modello \rnf{} di Fanteria di Barbari.} 

% Hellmaw

\newcommand{\ominousgateway}{Varco sinistro}
\newcommand{\oneominousgateway}{Un \textbf{Varco sinistro}}
\newcommand{\twoominousgateways}{Due \textbf{Varchi sinistri}}
\newcommand{\ominousgatewaydef}{%
0-2 per Esercito.\newline%
Al passo 7 della Sequenza prepartita (durante la Selezione degli incantesimi), segna un punto sul Campo di battaglia con un Segnalino varco. Tale punto deve trovarsi fuori dalla Zona di schieramento dell'avversario. Se entrambi i giocatori possiedono Fauci d'inferno, inizia dal giocatore che ha scelto la propria Zona di schieramento.%
}

\newcommand{\gateway}{Varco} 
\newcommand{\gatewaydef}{% 
Alla fine di ogni Fase magica amica, ciascun Fauci d'inferno può applicare uno dei seguenti effetti:
\begin{itemize}[label={-}]%
\item \textbf{Aprire un varco:} Segna un punto del Campo di battaglia con un Segnalino varco. Questo punto deve essere entro la Linea di vista e \distance{24} dal Fauci d'inferno, e ad almeno \distance{6} da unità nemiche. Non possono mai essere presenti più di 4 Segnalini varco amici sul Campo da battaglia (Varchi sinistri inclusi).
\item \textbf{Chiudere un varco:} Scegli un Segnalino varco amico con il suo centro entro la Linea di vista e \distance{24} dal Fauci d'inferno. Tutte le unità entro \distance{2D6} subiscono D6 colpi con Forza 4, Penetrazione armatura 0 e Attacchi magici. Rimuovi il segnalino. Se tutti i Fauci d'inferno amici sono stati rimossi come perdita, chiudi immediatamente tutti i altri varchi, come descritto precedentemente.
\end{itemize}%

Un'unità amica composta interamente da modelli non Giganteschi con \hellforgedarmor{}, Sovrannaturale e/o Errante del velo che termina un movimento di Avanzata o Marcia a contatto col centro di un Segnalino varco amico può scegliere di attraversarlo: rimuovi l'unità dal Campo di battaglia. L'unità:
\begin{itemize}[label={-}]%
\item È riposizionata sul Campo di battaglia entro \distance{3} dal centro di un qualunque altro Segnalino varco amico. In tal caso la distanza dal Segnalino varco scelto alla posizione finale, misurata dal centro di ciascun modello, non può essere superiore alla propria Marcia.
\item Deve mantenere la stessa formazione, ma può essere rivolta in una qualsiasi direzione.
\item Deve rispettare la regola Distanza tra le unità.
\item Perde Presidiante fino al successivo Turno del giocatore amico. 
\end{itemize}%

Durante ciascun Turno del giocatore solo una singola unità può uscire dal medesimo Varco.%
} 
