\RequirePackage{amsmath}																									% debugging a warning
\documentclass[a4paper,10pt]{article}

\usepackage[a4paper, top=2cm, bottom=2cm, left=2cm, right=2cm]{geometry}			% marge reduction


%% Fonts

\usepackage{fontspec}
\setmainfont[
	Ligatures=TeX,
	SmallCapsFont={Caladea}
	]{Caladea} % default is Latin Modern
%\setmainfont[Ligatures=TeX,SmallCapsFont={OpenDyslexic}]{OpenDyslexic}		% default is Latin Modern
\newfontfamily\titlefont[Ligatures=TeX]{Georgia} % font for front page title
\newfontfamily\fulltitlefont[Ligatures=TeX]{CaslonOS} % font for front page title


%% Typing and text

\usepackage{microtype}																											% greatly improves general appearance of the text
\usepackage{siunitx}																												% streamlining numbers and units, use \SI{number}{unit}
\sisetup{detect-all}																													% avoid using the math font for numbers in a non-math text
\usepackage{ulem}																													% to cross words out, use \sout{}
\usepackage{anyfontsize}  																										% disable the warnings when a font size isn't available
\usepackage{unicode-math} 																									% so math characters are copy pasted correctly
\usepackage{csquotes}																											% automatic quotation marks adapted of the current language, can be summoned with \enquote{X}
\usepackage{parskip}%[=v1] 																									% no paragraph indentation and spaces between paragraphs
\usepackage{accsupp}																												% allows to automatically replace some characters when copy-pasted
\usepackage{pdfrender}																											% Allows to print outlined characters
\usepackage{verse}																													% For poems


%% Language specific packages

\ifdefined\languageisfrench
	\usepackage[french]{babel}
	\frenchbsetup{StandardLists=true} 																					% necessary to use enumitem with babel/french
\fi
\ifdefined\languageisenglish
	\usepackage[super]{nth}																										% to write 9^th correctly
\fi
\ifdefined\languageisitalian
	\usepackage[italian]{babel}
\fi
\ifdefined\languageisspanish
	\usepackage[spanish,es-noquoting]{babel}
\fi
\ifdefined\languageispolish
	\usepackage[polish]{babel}
\fi
\ifdefined\languageisrussian
	\usepackage[russian]{babel}
	\usepackage[super]{nth}																										% to write 9^th correctly
	\setmainfont[Ligatures=TeX]{Cambria} % default is Latin Modern
	%\newfontfamily\titlefont[Ligatures=TeX]{Georgia} % font for front page title
	\renewfontfamily\fulltitlefont[Ligatures=TeX]{Georgia} % font for front page title
\fi
\ifdefined\languageisgerman
	\usepackage[german]{babel}
	\usepackage[super]{nth}																										% to write 9^th correctly
\fi


%% Array utilities

\usepackage{booktabs} 																											% for rules (lines) in arrays
\usepackage{pbox} 																													% for linebreaks in arrays
\usepackage{array}																													% additionnal options for arrays
\usepackage{colortbl}																												% additionnal options for coloring arrays
\usepackage[table]{xcolor}																										% auto alternate grey-white rows, xcolor must be loaded before mdframed
\usepackage[export]{adjustbox}																							% centered pics in tables
\usepackage{diagbox}																												% diagonal slash in a cell
\usepackage{multirow}																											% make a row from multiple rows


%% List utilities and code

\usepackage[inline]{enumitem}   																							% display inline lists
\usepackage{etoolbox}           																									% general utility, good for lists for instance
\usepackage{xparse}             																									% list utilities
\usepackage{calc}																														% allow for calculation, in length for example
\usepackage{xstring}            																									% string parsing, cutting, etc.
\usepackage{keyval}																													% macro arguments as keywords
\usepackage{datatool}																												% database management


%% Frames

\usepackage{framed}																												% boxes
\usepackage[framemethod=TikZ]{mdframed}																	% fancy frames
\usepackage{tikz}																														% fancy frames and drawing
\usetikzlibrary{shapes, positioning,tikzmark,arrows,shadows,shadows.blur,decorations.pathreplacing}%
\usepackage{wrapfig}																												% fancy insertion of pics in text


%% Graphics and other floats

\usepackage{graphicx}        																										% for the \includegraphics command
\usepackage{caption}																												% caption customization
\captionsetup{singlelinecheck=off, labelfont=bf, textfont={color=black!60}}
\usepackage{float}																													% forces float in a specific position with [H]
\usepackage{epstopdf}      																									    % needed to use the .eps format in LuaTeX
\usepackage{transparent}																										% for Inkscape
\graphicspath{{./pics/}{./../Layout/pics/}{./../Arcane_Compendium/pics/}}


%% Page utilities

\usepackage{multicol}																												% allows to divide a part of the page in multiple columns
\usepackage{titlesec} 																												% titles customization
\usepackage{fancyhdr}																											% for custom headers and foot texts
\pagestyle{fancy}
\usepackage[strict]{changepage}																							% allows to check if the page number is odd or even
\usepackage{ragged2e}																											% Allows for hyphenation in center and such environments, if using the caps version (Center, ...)


%% TOC

\ifdefined\thisistherulebook
	\usepackage{tocloft} 																											% http://ctan.org/pkg/tocloft
%	\usepackage[toc]{multitoc}																									% ToC on multiple columns
\fi

%% Index

\ifdefined\thisistherulebook
	\usepackage{imakeidx}																										% index management, no need of a second run, customization
	\makeindex[columns=1,options=-s ../Layout/indexstyle.ist]
	\indexsetup{level=\void, noclearpage=true, othercode={\pagestyle{fancy}},firstpagestyle=fancy}
\fi

	
%% Links and references

\ifdefined\thisistherulebook
	\usepackage{refcount}																											% extract number from page number
\fi
\definecolor{linkcolour}{RGB}{131,25,139}
\usepackage[unicode, colorlinks=true, linkcolor=linkcolour, urlcolor=linkcolour, bookmarks=false, pdfdisplaydoctitle=true, pdfstartview=FitH, pdfpagelabels=true]{hyperref} % Links in PDF.
\usepackage[ocgcolorlinks]{ocgx2}																						% colours and/or links disappear when printed



\makeatletter																															% @ is used as a letter in this code
