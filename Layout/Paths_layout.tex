\RequirePackage{amsmath}
\documentclass[a4paper,10pt]{extarticle} % extarticle allows to use font size of 8pt.

\usepackage[a4paper, top=2cm, bottom=2cm, left=2cm, right=2cm]{geometry} % Marge reduction.

%% Font and typing packages
\usepackage{fontspec}
\setmainfont[
	Ligatures=TeX,
	]{Caladea} % default is Latin Modern
\newfontfamily\titlefont[Ligatures=TeX]{Georgia} % font for front page title
\usepackage{microtype}			% Greatly improves general appearance of the text.
\usepackage{siunitx}			% Unit appearance.
\sisetup{detect-all}				% Avoid using math font in a normal text.
\usepackage{ulem}				% To cross words out. Use \sout{}.
\usepackage{anyfontsize}  % Disable the warnings when a font size isn't available.
\usepackage{unicode-math} % Math characters are copy pasted correctly
\usepackage{accsupp}				% Allows to automatically replace some characters when copy-pasted

\ifdefined\languageisfrench
	%% Language specific package
	\usepackage[french]{babel}
	\frenchbsetup{StandardLists=true} % Necessary to use enumitem with babel/french.
\fi
\ifdefined\languageisenglish
	%% Language specific package
	\usepackage[super]{nth}
\fi
\ifdefined\languageisitalian
	%% Language specific package
	\usepackage[italian]{babel}
\fi
\ifdefined\languageisspanish
	%% Language specific package
	\usepackage[spanish]{babel}
\fi
\ifdefined\languageispolish
	%% Language specific package
	\usepackage[polish]{babel}
\fi
\ifdefined\languageisrussian
	%% Language specific package
	\usepackage[russian]{babel}
\fi
\ifdefined\languageisgerman
	%% Language specific package
	\usepackage[german]{babel}
	\usepackage[super]{nth}
\fi

%% Array utilities
\usepackage{array}				% Additionnal options for arrays.
\usepackage{colortbl}			% Additionnal options for coloring arrays.
\usepackage[table]{xcolor}		% Auto alternate grey-white rows.
\usepackage{multirow}		% make a row from multiple rows

%% List utilities
\usepackage[inline]{enumitem}   % Display inline lists.
\usepackage{etoolbox}           % General utility. Good for lists for instance.
\usepackage{xparse}             % List utilities.

%% Page utilities
\usepackage{graphicx}        % for the \includegraphics command
\usepackage{parskip} 			% no paragraph indentation and spaces between paragraphs.
\usepackage{multicol}			% Allows to divide a part of the page in multiple columns.
\usepackage{fancyhdr}		% For custom headers and foot texts
\usepackage{titlesec}			% title customization
\pagestyle{fancy}
\usepackage{framed}
\usepackage{csquotes}		% automatic quotation marks adapted of the current language. can be summoned with \enquote{X}
	
%% Others
\usepackage{calc}
\usepackage{epstopdf}         % needed to use the .eps format in LuaTeX
\usepackage{xstring}            % String parsing, cutting, etc.
\definecolor{linkcolour}{RGB}{131,25,139}
\usepackage[unicode, colorlinks=true, linkcolor=linkcolour, urlcolor=linkcolour, bookmarks=false, pdfdisplaydoctitle=true, pdfstartview=FitH, pdfpagelabels=true]{hyperref} % Links in PDF.
\usepackage[ocgcolorlinks]{ocgx2}

\makeatletter

%%% Language specific stuff

\input{../Layout/language_specific_files.tex}

%%% Technical commands %%%

% Necessary for language_specific
\newcommand{\ifsubstring}[4]{%
\protected@edef\split@temp{#1}%
\protected@edef\split@tempbis{#2}%
\saveexpandmode%
\expandarg\IfSubStr{\split@temp}{\split@tempbis}{#3}{#4}%
\restoreexpandmode%
}

\newcommand{\isitoneornot}[3]{%
% First step is to remove spaces if there are some
\def\numberwithoutspaces{\expandafter\removespaces\expandafter{#1}}%
% Next step is getting rid of formatting if there are any (bold, color, ...)
\pdfstringdef\cleannumber{\numberwithoutspaces}%
%Defining 1 in \pdfstringdef terms (it will add \376\377\000 before usually - unicode identifier)
\pdfstringdef\numberone{1}%
% Now we can try if it is 1 or not
\ifsubstring{\numberone}{\cleannumber}{#2}{#3}%
}

\newcommand{\isthereaplusornot}[3]{%
\ifsubstring{#1}{+}{#2}{#3}%
}

\def\removespaces#1{\zap@space#1 \@empty}

% Other
\newcommand{\leftspecialbracket}{% angle bracket that get replaced by a < when copy pasted
\BeginAccSupp{method=hex,unicode,ActualText=003C}%
$\langle$%
\EndAccSupp{}%
}

\newcommand{\rightspecialbracket}{% angle bracket that get replaced by a > when copy pasted
\BeginAccSupp{method=hex,unicode,ActualText=003E}%
$\rangle$%
\EndAccSupp{}%
}

\newenvironment{hidewhenprinted}{\begin{ocg}[printocg=never]{HideWhenPrinted}{id1}{1}}{\end{ocg}}

\newcommand{\boosted}[1]{\textbf{\textcolor{blue}{$\{$#1$\}$}}}
\newcommand{\base}[1]{\textcolor{red}{\hspace*{-0.25ex}\leftspecialbracket{}#1\rightspecialbracket{}}}
\newcommand{\specialboosted}[1]{\textbf{\textcolor{olive}{\hspace*{-0.25ex}\leftspecialbracket{}\hspace*{-0.4ex}\leftspecialbracket{}#1\rightspecialbracket{}\hspace*{-0.4ex}\rightspecialbracket{}}}}

\newcommand{\newrule}{\textcolor{green!50!black}}
\newcommand{\removedrule}[1]{\textcolor{green!50!black}{\sout{#1}}}
\newcommand{\rewordedrule}{\textcolor{green!60!black}}
\newcommand{\removedreworded}[1]{\textcolor{green!60!black}{\sout{#1}}}
\newcommand{\starsymbol}{$\star$}
\newcommand{\refsymbol}{$^\star$}

\DeclareSIUnit{\degree}{\SIUnitSymbolDegree}
\DeclareSIUnit{\inch}{″}
\DeclareSIUnit{\foot}{′}
\newcommand{\range}[1] {\labels@range~#1\si{\inch}}
\newcommand{\distance}[1] {#1\si{\inch}}
\newcommand{\result}[1] {‘#1’}

\newcommand{\verysmallfontsize}{\fontsize{4}{4.8}\selectfont}
\newcommand{\smallfontsize}{\fontsize{6}{7.2}\selectfont}
\newcommand{\normalfontsize}{\fontsize{8}{9.6}\selectfont}
\newcommand{\largefontsize}{\fontsize{10}{12}\selectfont}
\newcommand{\largerfontsize}{\fontsize{12}{14.4}\selectfont}
\newcommand{\Largefontsize}{\fontsize{14}{16.8}\selectfont}
\newcommand{\Largerfontsize}{\fontsize{15}{18}\selectfont}
\newcommand{\hugefontsize}{\fontsize{18}{21.6}\selectfont}
\newcommand{\Hugefontsize}{\fontsize{25}{30}\selectfont}

%%% Table of Contents %%%

\newcommand{\toctarget}[1]{%
\phantomsection\label{#1}%
\hypertarget{#1}%
}

\newcommand{\tocentry}[2]{%
\noindent\hyperlink{#1}{#2}\hfill\pageref{#1}%
}


%%% Icons %%%

\newcommand{\alchemyicon}{pics/Icon_alchemy.pdf}
\newcommand{\shamanismicon}{pics/Icon_shamanism.pdf}
\newcommand{\cosmologyicon}{pics/Icon_cosmology.pdf}
\newcommand{\divinationicon}{pics/Icon_divination.pdf}
\newcommand{\druidismicon}{pics/Icon_druidism.pdf}
\newcommand{\evocationicon}{pics/Icon_evocation.pdf}
\newcommand{\occultismicon}{pics/Icon_occultism.pdf}
\newcommand{\pyromancyicon}{pics/Icon_pyromancy.pdf}
\newcommand{\witchcrafticon}{pics/Icon_witchcraft.pdf}
\newcommand{\thaumaturgyicon}{pics/Icon_thaumaturgy.pdf}

\newcommand{\cosmologychaosicon}{pics/Icon_cosmology_chaos.pdf}
\newcommand{\cosmologycosmosicon}{pics/Icon_cosmology_cosmos.pdf}

%%% Headers %%%

\renewcommand{\headrulewidth}{0pt}
% Footers defined in the main .tex file to customize order

\newcommand{\footerlogosize}{0.5cm}

\setlength{\columnsep}{1cm}

%%% Table parameters %%%

\newcolumntype{M}[1]{>{\centering\let\newline\\\arraybackslash\hspace{0pt}}m{#1}}
\newcolumntype{P}[1]{>{\centering\let\newline\\\arraybackslash\hspace{0pt}}p{#1}}

\setlength{\tabcolsep}{0pt}
\renewcommand{\arraystretch}{1.55}

\arrayrulecolor{black!50}
\setlength{\arrayrulewidth}{0.5pt}

\newcommand{\newpathlogo}[2]{\clearpage\begin{center}\toctarget{#2}{\includegraphics[height=1.8cm]{#1}}\end{center}}

\newcommand{\starttable}[2]{%
\begin{center}
\begin{tabular}{m{0.53\linewidth}m{0.47\linewidth}}
\rowcolor[RGB]{#1}\multicolumn{2}{p{\linewidth}}{%
\rule{0pt}{0.7cm}% setting a minimum height
\centering\hugefontsize\textbf{#2}}\tabularnewline%
}

\newcommand{\tablepassive}[2]{%
\multicolumn{2}{p{\linewidth}}{%
\textbf{#1}\spacebeforecolon{}: #2}\tabularnewline
}

\newcommand{\tablelabels}{%
\hspace*{0.08\linewidth}\begin{tabular}{M{0.14\linewidth}M{0.24\linewidth}m{0.25\linewidth}m{0.29\linewidth}}
\textit{\spellsCastingValue}&
\textit{\spellsRange}&
\textit{\spellsType}&
\textit{\spellsDuration}\tabularnewline
\end{tabular}&
\textit{\spellsEffect}\tabularnewline
\hline
}

\newcommand{\spelllabelandtitle}[2]{%
\begin{tabular}{@{}M{0.08\linewidth}@{}m{0.92\linewidth}@{}}%
{\Largefontsize\textbf{#1}}&%
%\makebox[0.08\linewidth]{\Largefontsize\textbf{#1}}
{\Largefontsize\textbf{#2}}\tabularnewline%
\end{tabular}%
}

\newcommand{\replabel}{\newline\textit{\largefontsize\replicablespellnumber}}

\newcommand{\tablespellcastingvalue}[1]{%
\vspace*{-5pt}
\hspace*{0.08\linewidth}\begin{tabular}{M{0.14\linewidth}M{0.24\linewidth}m{0.25\linewidth}m{0.29\linewidth}}
\textbf{#1}&
}

\newcommand{\tablespellrange}[1]{%
\textbf{#1}&
}

\newcommand{\tablespelltype}[1]{%
#1 &
}

\newcommand{\tablespellduration}[1]{%
#1\tabularnewline
\end{tabular}
&
}

\newcommand{\tablespelleffect}[1]{%
#1\tabularnewline
}

\newcommand{\closetable}{%
\end{tabular}
\end{center}
}

\newcommand{\newcosmologylogo}{\clearpage\begin{center}\toctarget{cosmology}{\hspace*{0.3cm}\includegraphics[width=0.5cm]{\cosmologycosmosicon}\hspace*{4.35cm}\includegraphics[height=1.8cm]{\cosmologyicon}\hspace*{3.95cm}\includegraphics[width=0.8cm]{\cosmologychaosicon}}\end{center}}

\newcommand{\starttablecosmology}{%
\begin{center}%
\begin{tabular}{M{0.05\linewidth}m{0.15\linewidth}M{0.18\linewidth}m{0.12\linewidth}M{0.15\linewidth}M{0.1\linewidth}m{0.25\linewidth}}%
\rowcolor[RGB]{\colors@cosmology}\multicolumn{7}{p{\linewidth}}{%
\rule{0pt}{0.7cm}% setting a minimum height
\centering{%
\begin{minipage}{\widthof{\textbf{\cosmos}}}\vspace*{-0.35cm}\textbf{\textcolor{white}{\cosmos}}\end{minipage}%
\hspace*{3.5cm}%
\begin{minipage}{\widthof{\hugefontsize\textbf{\cosmology}}}{\vspace*{-0.2cm}\hugefontsize\textbf{\textcolor{white}{\cosmology}}}\end{minipage}%
\hspace*{3.5cm}%
\begin{minipage}{\widthof{\textbf{\chaos}}}\vspace*{-0.35cm}\textbf{\textcolor{white}{\chaos}}\end{minipage}%
}}\tabularnewline%
}

\newcommand{\tablepassivecosmology}[2]{%
\multicolumn{7}{p{\linewidth}}{%
\textbf{#1}\spacebeforecolon{}: #2}\tabularnewline
}

\newcommand{\tablelabelscosmology}{%
\hline
&\textit{\spellsType}&
\textit{\spellsDuration}&&
\textit{\spellsCastingValue}&
\textit{\spellsRange}&
\tabularnewline
\hline
}

\newcommand{\spelllabelandtitlecosmology}[2]{%
\Largefontsize\textbf{#1}&
\multicolumn{3}{m{0.45\linewidth}}{\Largefontsize\textbf{#2}}&
}

\newcommand{\tablespellcastingvaluecosmology}[1]{%
\textbf{#1}&
}

\newcommand{\tablespellrangecosmology}[1]{%
\textbf{#1}&\tabularnewline
}

\newcommand{\tablecosmosicon}{\vspace*{-0.15cm}\includegraphics[width=0.4cm]{\cosmologycosmosicon}&}
\newcommand{\tablechaosicon}{\vspace*{0.05cm}\includegraphics[width=0.5cm]{\cosmologychaosicon}&}

\newcommand{\tablespelltextcosmology}[1]{\multicolumn{4}{m{0.62\linewidth}}{#1}\tabularnewline}




\def\colors@alchemy{255,215,0}
\def\colors@shamanism{114,56,16}
\def\colors@cosmology{0,57,116}
\def\colors@divination{47,191,226}
\def\colors@druidism{127,223,90}
\def\colors@evocation{94,29,101}
\def\colors@occultism{222,0,100}
\def\colors@pyromancy{204,0,0}
\def\colors@witchcraft{135,139,166}
\def\colors@thaumaturgy{0,84,34}


% SPELL FORMAT FOR CONSISTENCY

% #1 : Spell name
% #2 : Spell path
\newcommand{\spellformat}[2]{\textit{#1}\ifblank{#2}{}{ (#2)}}


% Titles

\titleformat{\paragraph}{\normalfont\normalsize\bfseries}{\theparagraph}{1em}{}
\titlespacing*{\paragraph}{0pt}{2.5ex plus 1ex minus .2ex}{0.5ex plus .2ex}

\newcommand{\basictitle}[2]{%
\section*{}\vspace*{-20pt}\noindent\begin{center}\toctarget{#1}{\Largerfontsize\textbf{#2}}\end{center}\vspace*{10pt}
}

\newcommand{\basicsubtitle}[1]{%
\noindent{\largerfontsize\textbf{#1}}
}

\newcommand{\basicsubsubtitle}[1]{%
\noindent{\largefontsize\textbf{#1}}
}
