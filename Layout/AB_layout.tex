\RequirePackage{amsmath}
\documentclass[a4paper,10pt]{article}

\usepackage[a4paper, top=2cm, bottom=2cm, left=2cm, right=2cm]{geometry} % Marge reduction.

%% Font and typing packages
\usepackage{fontspec}
\setmainfont[
	Ligatures=TeX,
	SmallCapsFont={Caladea}
	]{Caladea}																		% default is Latin Modern

\newfontfamily\titlefont[Ligatures=TeX]{Georgia}	% font for front page title
\usepackage{microtype}													% Greatly improves general appearance of the text.
\usepackage{siunitx}														% Unit appearance.
\sisetup{detect-all}															% Avoid using the math font in a normal text.
\usepackage{ulem}															% To cross words out. Use \sout{}.
\usepackage{anyfontsize}												% Disable the warnings when a font size isn't available.
\usepackage{unicode-math}											% Math characters are copy pasted correctly
\usepackage{accsupp}														% Allows to automatically replace some characters when copy-pasted

\def\isitanAB{Yesofcourse} 											% Is it still of any use?

\ifdefined\languageisfrench
	%% Language specific package
	\usepackage[french]{babel}
	\frenchbsetup{StandardLists=true}							 % Necessary to use enumitem with babel/french.
\fi
\ifdefined\languageisenglish
	%% Language specific package
	\usepackage[super]{nth}
\fi
\ifdefined\languageisitalian
	%% Language specific package
	\usepackage[italian]{babel}
\fi
\ifdefined\languageisspanish
	%% Language specific package
	\usepackage[spanish]{babel}
\fi
\ifdefined\languageispolish
	%% Language specific package
	\usepackage[polish]{babel}
\fi
\ifdefined\languageisrussian
	%% Language specific package
	\usepackage[russian]{babel}
\fi
\ifdefined\languageisgerman
	%% Language specific package
	\usepackage[german]{babel}
	\usepackage[super]{nth}
\fi

%% Array utilities
\usepackage{booktabs} 													% for rules in arrays
\usepackage{pbox} 															% for linebreaks in arrays
\usepackage{array}															% Additionnal options for arrays.
\usepackage{colortbl}														% Additionnal options for coloring arrays.
\usepackage[table]{xcolor}												% Auto alternate grey-white rows. Xcolor must be loaded before mdframed.
\usepackage{multirow}													% make a row from multiple rows
\usepackage[export]{adjustbox}									% Centered pics in tables
\usepackage{diagbox}														% diagonal slash in a cell

%% List utilities
\usepackage[inline]{enumitem}									% Display inline lists.
\usepackage{etoolbox}													% General utility. Good for lists for instance.
\usepackage{xparse}														% List utilities.
\usepackage{datatool}														% Handling alphabetical order.

%% Frames
\usepackage{framed}														% Boxes.
\usepackage[framemethod=TikZ]{mdframed}			% For fancy frames.
\usepackage{tikz}																% For fancy frames.
\usepackage{wrapfig}														% Fancy insertion of pics in text.

%% Page utilities
\usepackage{refcount}														% extract number from page number
\usepackage{graphicx}														% for the \includegraphics command
\usepackage{parskip}														% no paragraph indentation and spaces between paragraphs.
\usepackage{multicol}														% Allows to divide a part of the page in multiple columns.
\usepackage{titlesec}														% titles customization
\usepackage{caption}														% captions customization
\captionsetup{singlelinecheck=off, labelfont=bf, textfont={color=black!60}}
\usepackage{float}															% Forces float in a specific position with H
\usepackage{fancyhdr}													% For custom headers and foot texts
\pagestyle{fancy}
\usepackage{csquotes}													% automatic quotation marks adapted of the current language. can be summoned with \enquote{X}
	
%% Others
\usepackage{calc}																% arithmetic operation and length utilities
\usepackage{epstopdf}													% needed to use the .eps format in LuaTeX
\usepackage{keyval}															% Used to create maps of commands/labels/objects.
\makeatletter																	% Mandatory for the usage of keyval.
\usepackage{xstring}														% String parsing, cutting, etc.
\definecolor{linkcolour}{RGB}{131,25,139}
\usepackage[unicode, colorlinks=true, linkcolor=linkcolour, urlcolor=linkcolour, bookmarks=false, pdfdisplaydoctitle=true, pdfstartview=FitH, pdfpagelabels=false]{hyperref} % Links in PDF.
\usepackage[ocgcolorlinks]{ocgx2}

\graphicspath{{./pics/}{./../Layout/pics/}{./../Paths/pics/}}

%%% Language specific stuff

\input{../Layout/language_specific_files.tex}

%%% Commands to handle strings, better than xstring to handle commands inside the strings %%%

\newcommand{\substitute}[3]{%
  \protected@edef\sub@temp{#1}%
  \saveexpandmode%
  \expandarg\StrSubstitute{\sub@temp}{#2}{#3}[#1]%
  \restoreexpandmode%
}

\newcommand{\pdfsubstitute}[3]{%
	% #1 is the string in which we need to substitute something
	% #2 need to be replaced by #3
	\pdfstringdef\specialcharcode{a#2}%
	\pdfstringdef\tempcharcode{a}%
	\substitute\specialcharcode{\tempcharcode}{}%
	\pdfstringdef\replacementcharcode{a#3}%
	\substitute\replacementcharcode{\tempcharcode}{}%
	\substitute{#1}{\specialcharcode}{\replacementcharcode}%
}

\newcommand{\splitatstar}[3]{%
  \protected@edef\split@temp{#1}%
  \saveexpandmode%
  \expandarg\StrCut{\split@temp}{*}#2#3%
  \restoreexpandmode%
}

\newcommand{\splitatinf}[3]{%
  \protected@edef\split@temp{#1}%
  \saveexpandmode%
  \expandarg\StrCut{\split@temp}{<}#2#3%
  \restoreexpandmode%
}

\newcommand{\splitatstickto}[3]{%
  \protected@edef\split@temp{#1}%
  \saveexpandmode%
  \expandarg\StrCut{\split@temp}{<STICKTO<}#2#3%
  \restoreexpandmode%
}

\newcommand{\splitatequal}[3]{%
  \protected@edef\split@temp{#1}%
  \saveexpandmode%
  \expandarg\StrCut{\split@temp}{=}#2#3%
  \restoreexpandmode%
}

\newcommand{\ifsubstring}[4]{%
\protected@edef\split@temp{#1}%
\protected@edef\split@tempbis{#2}%
\saveexpandmode%
\expandarg\IfSubStr{\split@temp}{\split@tempbis}{#3}{#4}%
\restoreexpandmode%
}

\def\removespaces#1{\zap@space#1 \@empty}

\newcommand{\isitoneornot}[3]{%
% First step is to remove spaces if there are some
\def\numberwithoutspaces{\expandafter\removespaces\expandafter{#1}}%
% Next step is getting rid of formatting if there are any (bold, color, ...)
\pdfstringdef\cleannumber{\numberwithoutspaces}%
%Defining 1 in \pdfstringdef terms (it will add \376\377\000 before usually - unicode identifier)
\pdfstringdef\numberone{1}%
% Now we can try if it is 1 or not
\ifsubstring{\numberone}{\cleannumber}{#2}{#3}%
}

\newcommand{\isthereaplusornot}[3]{%
\ifsubstring{#1}{+}{#2}{#3}%
}


%%% Commands for alphabetical ordering %%%

\newcommand{\addtosortedlist}[1]{%
	\ifsubstring{#1}{<STICKTO<}{%
		\def\prefixforsorting{}%
		\def\actualname{}%
		\splitatstickto{#1}{\prefixforsorting}{\actualname}%
		\pdfstringdef\textwithoutformatting{#1}%
		\pdfsubstitute\textwithoutformatting{<STICKTO<}{}%
		\pdfsubstitute\textwithoutformatting{'}{}%
		\pdfsubstitute\textwithoutformatting{ }{}%
		\dolanguagespecificsubstitute{}%
		\sortitem{\textwithoutformatting}{\actualname}%	
	}{%
	\pdfstringdef\textwithoutformatting{#1}%
	\pdfsubstitute\textwithoutformatting{'}{}%
	\pdfsubstitute\textwithoutformatting{ }{}%
	\dolanguagespecificsubstitute{}%
	\sortitem{\textwithoutformatting}{#1}%
	}%
}%

\newcommand{\addtosortedlistwithcost}[1]{%
	\ifsubstring{#1}{<STICKTO<}{%
		\def\prefixforsorting{}%
		\def\actualname{}%
		\splitatstickto{#1}{\prefixforsorting}{\actualname}%
		\pdfstringdef\textwithoutformatting{#1}%
		\pdfsubstitute\textwithoutformatting{<STICKTO<}{}%
		\pdfsubstitute\textwithoutformatting{'}{}%
		\pdfsubstitute\textwithoutformatting{ }{}%
		\dolanguagespecificsubstitute{}%
		\def\textwithoutcost{}%
		\def\costwewanttoextract{}%
		\StrCut{#1}{=}{\textwithoutcost}{\costwewanttoextract}%
		\pdfstringdef\costwithoutformatting{\costwewanttoextract}%
		\pdfsubstitute\costwithoutformatting{ }{}%
		\pdfsubstitute\costwithoutformatting{\free}{0}%
		\pdfsubstitute\costwithoutformatting{\permodel}{}%
		\pdfsubstitute\costwithoutformatting{\nolimit}{999999}%
		\substitute\costwithoutformatting{\string\376}{}%
		\substitute\costwithoutformatting{\string\377}{}%
		\substitute\costwithoutformatting{\string\000}{}%
		\sortitemwithcost{\textwithoutformatting}{\actualname}{\costwithoutformatting}%
	}{%
	\pdfstringdef\textwithoutformatting{#1}%
	\pdfsubstitute\textwithoutformatting{'}{}%
	\pdfsubstitute\textwithoutformatting{ }{}%
	\dolanguagespecificsubstitute{}%
	\def\textwithoutcost{}%
	\def\costwewanttoextract{}%
	\StrCut{#1}{=}{\textwithoutcost}{\costwewanttoextract}%
	\pdfstringdef\costwithoutformatting{\costwewanttoextract}%
	\pdfsubstitute\costwithoutformatting{ }{}%
	\pdfsubstitute\costwithoutformatting{\free}{0}%
	\pdfsubstitute\costwithoutformatting{\permodel}{}%
	\pdfsubstitute\costwithoutformatting{\nolimit}{999999}%
	\substitute\costwithoutformatting{\string\376}{}%
	\substitute\costwithoutformatting{\string\377}{}%
	\substitute\costwithoutformatting{\string\000}{}%
	\sortitemwithcost{\textwithoutformatting}{#1}{\costwithoutformatting}%
	}%
}%

\newcommand{\addtopathlist}[1]{%
	\DTLnewrow{pathlist}% Create a new entry
	\def\textwithouticon{}%
	\def\pathiconname{}%
	\noexpandarg\StrCut{#1}{=}{\textwithouticon}{\pathiconname}%
	\def\DTLentrycommand{\DTLnewdbentry{pathlist}{name}}%
	\expandafter\DTLentrycommand\expandafter{\textwithouticon}% Add name
	\def\DTLentrycommand{\DTLnewdbentry{pathlist}{icon}}%
	\expandafter\DTLentrycommand\expandafter{\pathiconname}% Add icon name
	\pdfstringdef\textwithoutformatting{\textwithouticon}%
	\pdfsubstitute\textwithoutformatting{'}{}%
	\pdfsubstitute\textwithoutformatting{ }{}%
	\dolanguagespecificsubstitute{}%
	\substitute\textwithoutformatting{\string\376}{}%
	\substitute\textwithoutformatting{\string\377}{}%
	\substitute\textwithoutformatting{\string\000}{}%
	\def\DTLentrycommand{\DTLnewdbentry{pathlist}{sortlabel}}%
	\expandafter\DTLentrycommand\expandafter{\textwithoutformatting}% Add entry sortlabel	
}%

\newcommand{\sortitem}[2]{%
	\DTLnewrow{alphaorderlist}% Create a new entry
	\def\DTLentrycommand{\DTLnewdbentry{alphaorderlist}{sortlabel}}%
	\expandafter\DTLentrycommand\expandafter{#1}% Add entry sortlabel
	\def\DTLentrycommand{\DTLnewdbentry{alphaorderlist}{description}}%
	\expandafter\DTLentrycommand\expandafter{#2}% Add entry description
}

\newcommand{\sortitemwithcost}[3]{%
	\DTLnewrow{alphaandcostorderlist}% Create a new entry
	\def\DTLentrycommand{\DTLnewdbentry{alphaandcostorderlist}{sortlabel}}%
	\expandafter\DTLentrycommand\expandafter{#1}% Add entry sortlabel
	\def\DTLentrycommand{\DTLnewdbentry{alphaandcostorderlist}{description}}%
	\expandafter\DTLentrycommand\expandafter{#2}% Add entry description
	\def\DTLentrycommand{\DTLnewdbentry{alphaandcostorderlist}{cost}}%
	\expandafter\DTLentrycommand\expandafter{#3}% Add entry cost
}

\newenvironment{sortedlist}{%
	\DTLifdbexists{alphaorderlist}{\DTLcleardb{alphaorderlist}}{\DTLnewdb{alphaorderlist}}% Create new/discard old list
}{%
	\DTLsort*{sortlabel}{alphaorderlist}% Sort list
	\setlength{\parskip}{0pt}%
	\DTLforeach*{alphaorderlist}{\theDesc=description}{% Print each item
		\theDesc{}\par%
	}%
	\DTLcleardb{alphaorderlist}%
}

\newcommand{\alphaorderstickto}[1]{%
#1<STICKTO<%
}

\newcommand{\alphaorderlist}[1]{%
\DTLifdbexists{alphaorderlist}{\DTLcleardb{alphaorderlist}}{\DTLnewdb{alphaorderlist}}% Create new/discard old list
\expandafter\parcommalist\expandafter{#1}{\locallists@alphaorderlist}%
\forlistloop{\addtosortedlist}{\locallists@alphaorderlist}%
\DTLsort*{sortlabel}{alphaorderlist}% Sort list
\DTLforeach*{alphaorderlist}{\theDesc=description}{% Print back the ordered list
\theDesc{}\DTLiflastrow{}{, }%
}%
\DTLcleardb{alphaorderlist}%
}

\pdfstringdefDisableCommands{\def\textcolor#1{}}
\pdfstringdefDisableCommands{\def\newline{}}
\pdfstringdefDisableCommands{\def\removedrule{}}
\pdfstringdefDisableCommands{\def\removedreworded{}}
\pdfstringdefDisableCommands{\def\void{}}

% See language specific file for \addtosortedlist

%%% Database for automatic Quick Ref Sheet %%%

\DTLnewdb{profiles} % Database containing profiles info for QRS
\newcounter{categorynumber}
\DTLnewdb{categories} % Database containing categories names

\newcommand{\dtbfillunitname}[1]{\DTLnewdbentry{profiles}{unitname}{#1}}
\newcommand{\dtbfillhypertag}[1]{\DTLnewdbentry{profiles}{hypertag}{#1}}
\newcommand{\dtbfillcategorytag}[1]{\DTLnewdbentry{profiles}{categorytag}{#1}}
\newcommand{\dtbfillsize}[1]{\DTLnewdbentry{profiles}{size}{#1}}
\newcommand{\dtbfilltype}[1]{\DTLnewdbentry{profiles}{type}{#1}}

\newcommand{\dtbfillglobal@Ad}[1]{\DTLnewdbentry{profiles}{globalAd}{#1}}
\newcommand{\dtbfillglobal@Ma}[1]{\DTLnewdbentry{profiles}{globalMa}{#1}}
\newcommand{\dtbfillglobal@Di}[1]{\DTLnewdbentry{profiles}{globalDi}{#1}}
\newcommand{\dtbfillglobal@Rsr}[1]{\DTLnewdbentry{profiles}{globalRsr}{#1}}
\newcommand{\dtbfillglobal@RsrSortLabel}[1]{\DTLnewdbentry{profiles}{globalRsrSortLabel}{#1}}

\newcommand{\dtbfilldefense@HP}[1]{\DTLnewdbentry{profiles}{defenseHP}{#1}}
\newcommand{\dtbfilldefense@Df}[1]{\DTLnewdbentry{profiles}{defenseDf}{#1}}
\newcommand{\dtbfilldefense@Re}[1]{\DTLnewdbentry{profiles}{defenseRe}{#1}}
\newcommand{\dtbfilldefense@Arm}[1]{\DTLnewdbentry{profiles}{defenseArm}{#1}}

\newcommand{\dtbfilloffensename}[1]{\DTLnewdbentry{profiles}{offensename}{#1}}
\newcommand{\dtbfilloffense@Ag}[1]{\DTLnewdbentry{profiles}{offenseAg}{#1}}
\newcommand{\dtbfilloffense@At}[1]{\DTLnewdbentry{profiles}{offenseAt}{#1}}
\newcommand{\dtbfilloffense@Of}[1]{\DTLnewdbentry{profiles}{offenseOf}{#1}}
\newcommand{\dtbfilloffense@St}[1]{\DTLnewdbentry{profiles}{offenseSt}{#1}}
\newcommand{\dtbfilloffense@AP}[1]{\DTLnewdbentry{profiles}{offenseAP}{#1}}

\newcommand{\dtbfilloffensenameB}[1]{\DTLnewdbentry{profiles}{offensenameB}{#1}}
\newcommand{\dtbfilloffenseB@Ag}[1]{\DTLnewdbentry{profiles}{offenseBAg}{#1}}
\newcommand{\dtbfilloffenseB@At}[1]{\DTLnewdbentry{profiles}{offenseBAt}{#1}}
\newcommand{\dtbfilloffenseB@Of}[1]{\DTLnewdbentry{profiles}{offenseBOf}{#1}}
\newcommand{\dtbfilloffenseB@St}[1]{\DTLnewdbentry{profiles}{offenseBSt}{#1}}
\newcommand{\dtbfilloffenseB@AP}[1]{\DTLnewdbentry{profiles}{offenseBAP}{#1}}

\newcommand{\dtbfilloffensenameC}[1]{\DTLnewdbentry{profiles}{offensenameC}{#1}}
\newcommand{\dtbfilloffenseC@Ag}[1]{\DTLnewdbentry{profiles}{offenseCAg}{#1}}
\newcommand{\dtbfilloffenseC@At}[1]{\DTLnewdbentry{profiles}{offenseCAt}{#1}}
\newcommand{\dtbfilloffenseC@Of}[1]{\DTLnewdbentry{profiles}{offenseCOf}{#1}}
\newcommand{\dtbfilloffenseC@St}[1]{\DTLnewdbentry{profiles}{offenseCSt}{#1}}
\newcommand{\dtbfilloffenseC@AP}[1]{\DTLnewdbentry{profiles}{offenseCAP}{#1}}

\newcommand{\dtbfilloffensenameI}[1]{\DTLnewdbentry{profiles}{offensenameI}{#1}}
\newcommand{\dtbfilloffenseI@Ag}[1]{\DTLnewdbentry{profiles}{offenseIAg}{#1}}
\newcommand{\dtbfilloffenseI@St}[1]{\DTLnewdbentry{profiles}{offenseISt}{#1}}
\newcommand{\dtbfilloffenseI@AP}[1]{\DTLnewdbentry{profiles}{offenseIAP}{#1}}

\newcommand{\void}[1]{}


%%% Technical commands %%%

\newenvironment{hidewhenprinted}{\begin{ocg}[printocg=never]{HideWhenPrinted}{id1}{1}}{\end{ocg}}

\newcommand{\debugfooter}{\hfill\textcolor{white}{debug}}

\newcommand{\greycolor}{black!50}
\newcommand{\greytextcolor}{\textcolor{\greycolor}}
\newcommand{\newrule}{\textcolor{blue!80!black}}
\newcommand{\protectednewrule}{\textcolor{blue!80!black}}
\newcommand{\removedrule}[1]{\textcolor{blue!80!black}{\sout{#1}}}
\newcommand{\rewordedrule}{\textcolor{green!60!black}}
\newcommand{\removedreworded}[1]{\textcolor{green!60!black}{\sout{#1}}}
\newcommand{\protectedrewordedrule}{\textcolor{green!80!black}}
\newcommand{\starsymbol}{$\star$}
\newcommand{\refsymbol}{*}
\newcommand{\flufffont}[1]{\textit{#1}}

\DeclareSIUnit[number-unit-product = {}]{\inch}{″}
\DeclareSIUnit[number-unit-product = {}]{\foot}{′}
\newcommand{\range}[1] {\labels@range~#1\si{\inch}}
\newcommand{\distance}[1] {#1\si{\inch}}
\newcommand{\result}[1] {‘#1’}
\newcommand{\plusone}{+1}

\newcommand{\captionpar}{\vspace*{10pt}\newline}
\newcommand{\captionlist}{\vspace*{3pt}\newline}
\newcommand{\captionitem}{\hspace*{0.3cm}}

\newcommand{\pts}[1]{\isitoneornot{#1}{#1~\labels@point}{#1~\labels@points}}

%%% Fonts and sizes %%%

\newcommand{\verysmallfontsize}{\fontsize{4}{4.8}\selectfont}
\newcommand{\smallfontsize}{\fontsize{6}{7.2}\selectfont}
\newcommand{\normalfontsize}{\fontsize{8}{9.6}\selectfont}
\newcommand{\largefontsize}{\fontsize{10}{12}\selectfont}
\newcommand{\largerfontsize}{\fontsize{12}{14.4}\selectfont}
\newcommand{\Largefontsize}{\fontsize{14}{16.8}\selectfont}
\newcommand{\Largerfontsize}{\fontsize{15}{18}\selectfont}
\newcommand{\hugefontsize}{\fontsize{18}{21.6}\selectfont}
\newcommand{\Hugefontsize}{\fontsize{25}{30}\selectfont}


%%% Table of Contents %%%

\newcommand{\toctarget}[1]{%
\phantomsection\label{#1}%
\hypertarget{#1}%
}

\newcommand{\tocentry}[2]{%
\noindent\hyperlink{#1}{#2}\hfill\pageref{#1}%
}


%%% Headers and footers %%%

\newcommand{\standardfooterspace}{\hspace*{0.4cm}}
\renewcommand{\headrulewidth}{0pt}
\fancyhead[R]{}
\fancyhead[L]{}
\fancyfoot[R]{%
\strut\begin{hidewhenprinted}%
\normalfontsize\footerright%
\end{hidewhenprinted}%
}
\fancyfoot[L]{%
\strut\begin{hidewhenprinted}%
\normalfontsize\footerleft%
\end{hidewhenprinted}%
}

%%% Titles %%%

\newlength{\mycurrentparskip}
\setlength{\mycurrentparskip}{\parskip}
\def\@minipagerestore{\setlength{\parskip}{\mycurrentparskip}} % Else paragraph and title spaces behave differently when inside or outside a minipage

\titleformat{\section}{\Hugefontsize\bfseries\filcenter}{}{0pt}{}
\titleformat{\subsection}{\Largerfontsize\bfseries}{}{0pt}{}
\titleformat{\subsubsection}{\largerfontsize\bfseries}{}{0pt}{}
\titleformat{\paragraph}{\normalfont\normalsize\bfseries}{\theparagraph}{1em}{}

\titlespacing*{\paragraph}{0pt}{\parskip}{-0.5\parskip}
\titlespacing*{\subsubsection}{0pt}{\parskip}{-0.5\parskip}
\titlespacing*{\subsection}{0pt}{1.5\parskip}{0pt plus .2ex}

\newcommand{\maintitle}[2]{\hypertarget{#1}{\section{#2}}\label{#1}}

\newcommand{\additionalspacebeforemaintitle}{\vspace*{15pt}}

\newcommand{\subtitle}[1]{\subsection{#1}}
\newcommand{\centeredsubtitle}[1]{\subsection*{\hfill#1\hfill}}

\newcommand{\titleruletype}[1]{\textnormal{\largefontsize{} - #1}}

\newcommand{\subsubtitle}[1]{\subsubsection{#1}}
\newcommand{\centeredsubsubtitle}[1]{\subsubsection*{\hfill#1\hfill}}

\newcommand{\armylisttitle}[3]{%
\hypertarget{#1}{%
\section*{#2\textnormal{ #3}}% \expandafter\uppercase\expandafter{#2}
}\label{#1}%
\stepcounter{categorynumber}%
\DTLnewrow{categories}%
\DTLnewdbentry{categories}{name}{#2}%
}

\newcommand{\armylistsubtitle}[1]{%
\subsection*{#1}%
}

\newcommand{\spaceaftersection}{\vspace{0.8cm}}

\newcommand{\separator}{\noindent\begin{center}\textcolor{\greycolor}{\rule{0.7\columnwidth}{2pt}}\end{center}}

\newcommand{\RTSlogo}{\includegraphics[width=1cm]{{logo_\bookprefix_rts}.png}}


%%% Custom lists and description for first sections of the army books

\setlength{\columnsep}{1cm}

\newcommand{\startpricelist}{\begin{samepage}\begin{description}[leftmargin=0.3cm, labelindent=0cm, labelsep=0.1cm, itemsep=8pt]}
\def\endpricelist{\end{description}\end{samepage}}
\newcommand{\pricelistitem}[2]{\item \textbf{#1}\hfill\pts{#2}\newline}
\newcommand{\nopricelistitem}[1]{\item \textbf{#1}\newline}

\newcommand{\itemrestriction}[1]{\greytextcolor{#1}\par\vspace*{-3pt}}%
\newcommand{\itemrestrictionnopar}[1]{\greytextcolor{#1}\vspace*{7pt}\newline}%
\newcommand{\itemrestrictionblack}[1]{#1\par\vspace*{-3pt}}%
\newcommand{\enchantmentrestriction}[1]{#1\newline}

\newcommand{\startpricelistNSP}{\begin{description}[leftmargin=0.3cm, labelindent=0cm, labelsep=0.1cm, itemsep=8pt]}
\def\endpricelistNSP{\end{description}}

\newcommand{\startsortedpricelist}{%
\begin{samepage}\begin{description}[leftmargin=0.3cm, labelindent=0cm, labelsep=0.1cm, itemsep=8pt]%
\DTLifdbexists{sortedpricelist}{\DTLcleardb{sortedpricelist}}{\DTLnewdb{sortedpricelist}}% Create new/discard old list
}
\def\endsortedpricelist{%
\DTLsort*{costsortlabel=descending,sortlabel}{sortedpricelist}% Sort list
\DTLforeach*{sortedpricelist}{\ruletext=ruletext,\nameoftheitem=name,\cost=cost}{% Print back the ordered table
\begin{samepage}\pricelistitem{\nameoftheitem}{\cost}%
\ruletext{}\end{samepage}%
}%
\end{description}\end{samepage}%
}

\newcommand{\sortedpricelistitem}[3]{%
\DTLnewrow{sortedpricelist}%
\DTLnewdbentry{sortedpricelist}{name}{#1}%
\DTLnewdbentry{sortedpricelist}{cost}{#2}%
\DTLnewdbentry{sortedpricelist}{ruletext}{#3}%
\pdfstringdef\textwithoutformatting{#1}%
\pdfsubstitute\textwithoutformatting{'}{}%
\pdfsubstitute\textwithoutformatting{ }{}%
\dolanguagespecificsubstitute{}%
\def\DTLentrycommand{\DTLnewdbentry{sortedpricelist}{sortlabel}}%
\expandafter\DTLentrycommand\expandafter{\textwithoutformatting}%
\pdfstringdef\costwithoutformatting{#2}%
\pdfsubstitute\costwithoutformatting{ }{}%
\pdfsubstitute\costwithoutformatting{\free}{0}%
\pdfsubstitute\costwithoutformatting{\nolimit}{999999}%
\substitute\costwithoutformatting{\string\376}{}%
\substitute\costwithoutformatting{\string\377}{}%
\substitute\costwithoutformatting{\string\000}{}%
\def\DTLentrycommand{\DTLnewdbentry{sortedpricelist}{costsortlabel}}%
\expandafter\DTLentrycommand\expandafter{\costwithoutformatting}%
}

\newcommand{\startsortedpricelistNSP}{%
\begin{description}[leftmargin=0.3cm, labelindent=0cm, labelsep=0.1cm, itemsep=8pt]%
\DTLifdbexists{sortedpricelist}{\DTLcleardb{sortedpricelist}}{\DTLnewdb{sortedpricelist}}% Create new/discard old list
}
\def\endsortedpricelistNSP{%
\DTLsort*{costsortlabel=descending,sortlabel}{sortedpricelist}% Sort list
\DTLforeach*{sortedpricelist}{\ruletext=ruletext,\nameoftheitem=name,\cost=cost}{% Print back the ordered table
\pricelistitem{\nameoftheitem}{\cost}%
\ruletext{}%
}%
\end{description}%
}

\newcommand{\startitemlist}{\raggedcolumns\begin{multicols}{2}\begin{description}[leftmargin=0.3cm, labelindent=0cm, labelsep=0.1cm, itemsep=8pt]}
\def\enditemlist{\end{description}\end{multicols}}
\newcommand{\listitem}[1]{\item[#1\spacebeforecolon{}:]}

\newcommand{\startitemlistonecol}{\begin{description}[leftmargin=0.3cm, labelindent=0cm, labelsep=0.1cm, itemsep=8pt]}
\def\enditemlistonecol{\end{description}}
\newcommand{\listitemonecol}[1]{\item \textbf{#1\spacebeforecolon{}:}\newline}

\newenvironment{smallitemize}{\begin{itemize}[label={-},itemsep=3pt,topsep=3pt]}{\end{itemize}}


%%% Army Model Rules sorting

\newcommand{\startAMRsortedlist}{%
	\DTLifdbexists{AMRlist}{\DTLcleardb{AMRlist}}{\DTLnewdb{AMRlist}}% Create new/discard old list
}

\newcommand{\closeAMRsortedlist}{%
	\DTLsort*{sortlabel}{AMRlist}% Sort list
	\DTLforeach*{AMRlist}{\name=name,\ruletext=ruletext}{%
		\subsubtitle{\name}
		\ruletext{}\par%
	}
	\DTLcleardb{AMRlist}%
}

\newcommand{\AMRsortedlistentry}[2]{%
	\DTLnewrow{AMRlist}%
	\pdfstringdef\textwithoutformatting{#1}%
	\pdfsubstitute\textwithoutformatting{'}{}%
	\pdfsubstitute\textwithoutformatting{ }{}%
	\dolanguagespecificsubstitute{}%
	\def\DTLentrycommand{\DTLnewdbentry{AMRlist}{sortlabel}}%
	\expandafter\DTLentrycommand\expandafter{\textwithoutformatting}%
	\DTLnewdbentry{AMRlist}{name}{#1}%
	\DTLnewdbentry{AMRlist}{ruletext}{#2}%
}


%%% Table parameters %%%

\newcommand{\alternaterowcolors}{\rowcolors{1}{white}{black!10}}

\newcolumntype{M}[1]{>{\centering\let\newline\\\arraybackslash\hspace{0pt}}m{#1}}
\newcolumntype{P}[1]{>{\centering\let\newline\\\arraybackslash\hspace{0pt}}p{#1}}
\arrayrulecolor{\greycolor}
\setlength{\arrayrulewidth}{0.5pt}
\newcommand{\resetarraystretch}{%
\renewcommand{\arraystretch}{1.2}
}
\resetarraystretch

%%%  Lists handling %%%

\newcommand{\addlocallist}{\listadd\locallists@dummy}%
\NewDocumentCommand{\parsespacelist}{>{\SplitList{ }} m }{%
	\ProcessList{#1}{\addlocallist}%
}%
\NewDocumentCommand{\parsecommalist}{>{\SplitList{,}} m }{%
	\ProcessList{#1}{\addlocallist}%
}%
\newcommand{\parselist}[3][,]{%
	\renewcommand\addlocallist{\listadd#3}%
  	\undef#3%
  	\ifstrequal{#1}{ }{\parsespacelist{#2}}{\parsecommalist{#2}}%
}

\def\parcommalist{\parselist[,]}%


%%% Hereditary Spell part

\newcommand{\leftspecialbracket}{% angle bracket that get replaced by a < when copy pasted
\BeginAccSupp{method=hex,unicode,ActualText=003C}%
$\langle$%
\EndAccSupp{}%
}

\newcommand{\rightspecialbracket}{% angle bracket that get replaced by a > when copy pasted
\BeginAccSupp{method=hex,unicode,ActualText=003E}%
$\rangle$%
\EndAccSupp{}%
}

\newcommand{\boosted}[1]{\textbf{\textcolor{blue}{$\{$#1$\}$}}}
\newcommand{\base}[1]{\textcolor{red}{\hspace*{-0.25ex}\leftspecialbracket{}#1\rightspecialbracket{}}}
\newcommand{\specialboosted}[1]{\textbf{\textcolor{olive}{\hspace*{-0.25ex}\leftspecialbracket{}\hspace*{-0.4ex}\leftspecialbracket{}#1\rightspecialbracket{}\hspace*{-0.4ex}\rightspecialbracket{}}}}

\newcommand{\replabel}{\newline\textit{\largefontsize\replicablespellnumber}}

\newcommand{\starthereditaryspell}[2][\hereditaryspellnumber{}]{%
\begin{center}%
\setlength{\tabcolsep}{0pt}%
\renewcommand{\arraystretch}{1.55}%
\arrayrulecolor{\greycolor}%
\setlength{\arrayrulewidth}{0.5pt}%
\begin{tabular}{m{0.53\linewidth}m{0.47\linewidth}}%
\hspace*{0.08\linewidth}\begin{tabular}{M{0.14\linewidth}M{0.24\linewidth}m{0.25\linewidth}m{0.29\linewidth}}
\textit{\spellsCastingValue}&
\textit{\spellsRange}&
\textit{\spellsType}&
\textit{\spellsDuration}\tabularnewline
\end{tabular}&
\textit{\spellsEffect}\tabularnewline
\hline
\begin{tabular}{@{}M{0.08\linewidth}@{}m{0.92\linewidth}@{}}%
{\Largefontsize\textbf{#1}}&%
{\Largefontsize\textbf{#2}}\tabularnewline%
\end{tabular}%
\par%
}

\newcommand{\HScastingvalue}[1]{%
\vspace*{-5pt}
\hspace*{0.08\linewidth}\begin{tabular}{M{0.14\linewidth}M{0.24\linewidth}m{0.25\linewidth}m{0.29\linewidth}}
\textbf{#1}&
}

\newcommand{\HSrange}[1]{%
\textbf{#1}&
}

\newcommand{\HStype}[1]{%
#1 &
}

\newcommand{\HSduration}[1]{%
#1\tabularnewline
\end{tabular}
&
}

\newcommand{\HSeffect}[1]{%
#1\tabularnewline
}

\newcommand{\closehereditaryspell}{%
\end{tabular}
\end{center}
}


%%% Army Organization part %%%

\newcommand{\openarmyorganizationtable}[1]{%
\begin{center}%
\renewcommand{\arraystretch}{1}%
\begin{tabular}{@{}*{#1}{M{\textwidth/#1}@{}}}%
}

\newcommand{\closearmyorganizationtable}{%
\end{tabular}\end{center}%
}

\newcommand{\armyorganizationlogo}[1]{%
\includegraphics[width=0.142\textwidth]{{logo_#1}.png}
}

\newcommand{\armyorganizationcategoryandlimit}[2]{%
\vspace*{-0.65cm}%

\textbf{#1}%

#2%
}

% SPELL FORMAT FOR CONSISTENCY

% #1 : Spell name
% #2 : Spell path
\newcommand{\spellformat}[2]{\textit{#1}\ifblank{#2}{}{ (#2)}}


%%%%%%%%%%%%%%%%%%
%%% Unit rules %%%
%%%%%%%%%%%%%%%%%%

%%% Rules %%%

% Rules listing for a unit, with alphabetical order.
\newcommand{\ruleslist}[1]{%
	\expandafter\parcommalist\expandafter{#1}{\locallists@ruleslist}%
	\begin{sortedlist}%
		\forlistloop{\addtosortedlist}{\locallists@ruleslist}%
	\end{sortedlist}%
}


%%% Options %%%

% Options listing.
\newcommand{\optionslist}[1]{%
	\expandafter\parcommalist\expandafter{#1}{\locallists@optionslist}%
	\begin{description}[leftmargin=0.3cm, labelindent=0cm, labelsep=0cm, itemsep=0cm, parsep=0cm]%
		\forlistloop{\item\setoption}{\locallists@optionslist}%
	\end{description}%
}

\newcounter{numberofitemsinlist}

\newcommand{\countitemsinoptionslist}[1]{%
	\setcounter{numberofitemsinlist}{0}%
	\expandafter\ifblank\expandafter{#1}{}{
		\expandafter\parcommalist\expandafter{#1}{\locallists@optionslist}%
		\forlistloop{\stepcounter{numberofitemsinlist}\void}{\locallists@optionslist}%
	}
}
% This function has no output. You can check the result with \arabic{numberofitemsinlist}, or use it in numeral expression with \value{numberofitemsinlist}

\newcommand{\optionsalphaorderlist}[1]{%
	\DTLifdbexists{alphaorderlist}{\DTLcleardb{alphaorderlist}}{\DTLnewdb{alphaorderlist}}% Create new/discard old list
	\expandafter\parcommalist\expandafter{#1}{\locallists@alphaorderlist}%
	\forlistloop{\addtosortedlist}{\locallists@alphaorderlist}%
	\DTLsort*{sortlabel}{alphaorderlist}% Sort list
	\begin{description}[leftmargin=0.3cm, labelindent=0cm, labelsep=0cm, itemsep=0cm, parsep=0cm]%
		\DTLforeach*{alphaorderlist}{\theDesc=description}{% Print back the ordered list
		\item\expandafter\setoption\expandafter{\theDesc}%
	}%
	\end{description}%
	\DTLcleardb{alphaorderlist}%
}

\newcommand{\optionsalphaandcostorderlist}[1]{%
	\DTLifdbexists{alphaandcostorderlist}{\DTLcleardb{alphaandcostorderlist}}{\DTLnewdb{alphaandcostorderlist}}% Create new/discard old list
	\expandafter\parcommalist\expandafter{#1}{\locallists@alphaorderlist}%
	\forlistloop{\addtosortedlistwithcost}{\locallists@alphaorderlist}%
	\DTLsort*{cost,sortlabel}{alphaandcostorderlist}% Sort list
	\begin{description}[leftmargin=0.3cm, labelindent=0cm, labelsep=0cm, itemsep=0cm, parsep=0cm]%
		\DTLforeach*{alphaandcostorderlist}{\theDesc=description}{% Print back the ordered list
		\item\expandafter\setoption\expandafter{\theDesc}%
	}%
	\end{description}%
	\DTLcleardb{alphaandcostorderlist}%
}

% Options entry.
\newcommand{\options}[1]{\begin{innerframe}[\labels@options]\optionslist{#1}\end{innerframe}}
\newcommand{\addoptions}[1]{\begin{innerframe}[\labels@addoptions]\optionslist{#1}\end{innerframe}}
\newcommand{\optionstwocols}[1]{\begin{innerframetwocols}[\labels@options]%
\setlength{\columnsep}{10pt+0.3cm}%
\setlength{\multicolsep}{0pt}%
\vspace*{-3pt}\begin{multicols}{2}\optionslist{#1}\end{multicols}\end{innerframetwocols}}

% Option specific commands.
\newcommand{\setoption}[1]{%
	\noexpandarg\StrCut{#1}{=}\optiontext\optionvalue%
	\expandafter\ifstrequal\expandafter{\optionvalue}{}{%
		\optiontext%
	}{%
		\option{\optiontext}{\optionvalue}%
	}%
}

\newcommand{\option}[2]{#1\nobreak\hspace*{3pt}\hfill\nobreak#2}%\penalty0\hbox{}

\newcommand\optionschoice[2]{%
	\parselist[,]{#2}{\locallists@optionschoice}%
	#1%
	\begin{itemize}[label={}, parsep=0cm, labelindent=0cm, labelwidth=0cm, noitemsep, topsep=0em, leftmargin=0.3cm]%
	\forlistloop{\item\setoption}{\locallists@optionschoice}%
	\end{itemize}%
}

\newcommand\optionschoiceTWOCOL[2]{%
	\parselist[,]{#2}{\locallists@optionschoice}%
	#1%
	\begin{itemize}[label={}, parsep=0cm, labelindent=0cm, labelwidth=0cm, noitemsep, topsep=0em, leftmargin=0.3cm]%
	\setlength{\columnseprule}{0.5pt}
	\renewcommand{\columnseprulecolor}{\color{\greycolor}}
	\raggedcolumns\begin{multicols}{2}
	\forlistloop{\item\setoption}{\locallists@optionschoice}%
	\end{multicols}\setlength{\columnseprule}{0pt}
	\end{itemize}%
}

\newcommand{\suboptionindent}{\hspace*{0.6cm}}

%%% alpha and cost ordering specific to options

\newcounter{AlCoOrdercounter}

\newcommand{\AlCoOrder}[1]{%
	\setcounter{AlCoOrdercounter}{0}% initialization
	\DTLifdbexists{alphaandcostorderlist}{\DTLcleardb{alphaandcostorderlist}}{\DTLnewdb{alphaandcostorderlist}}% Create new/discard old list
	\parcommalist{#1}{\locallists@alphaorderlist}%
	\forlistloop{\addtosortedlistwithcost}{\locallists@alphaorderlist}%
	\DTLsort*{cost,sortlabel}{alphaandcostorderlist}% Sort list
		\DTLforeach*{alphaandcostorderlist}{\theDesc=description}{% Print back the ordered list
		\ifnumcomp{\value{AlCoOrdercounter}}{=}{0}{\stepcounter{AlCoOrdercounter}}{\item}%
		\expandafter\setoption\expandafter{\theDesc}%
	}%
	\DTLcleardb{alphaandcostorderlist}%	
}

\newcommand{\AlOrder}[1]{%
	\setcounter{AlCoOrdercounter}{0}% initialization
	\DTLifdbexists{alphaorderlist}{\DTLcleardb{alphaorderlist}}{\DTLnewdb{alphaorderlist}}% Create new/discard old list
	\parcommalist{#1}{\locallists@alphaorderlist}%
	\forlistloop{\addtosortedlist}{\locallists@alphaorderlist}%
	\DTLsort*{sortlabel}{alphaorderlist}% Sort list
		\DTLforeach*{alphaorderlist}{\theDesc=description}{% Print back the ordered list
		\ifnumcomp{\value{AlCoOrdercounter}}{=}{0}{\stepcounter{AlCoOrdercounter}}{\item}%
		\expandafter\setoption\expandafter{\theDesc}%
	}%
	\DTLcleardb{alphaorderlist}%	
}

%%% Magic options %%%

% Magic entry.
\newcommand{\magic}[2]{\begin{innerframe}[\magicoptions]%
\expandafter\ifblank\expandafter{#1}{}{\optionslist{#1}\par}%
\pathslist{#2}%
\end{innerframe}}

\newcommand{\pathslist}[1]{%
	\expandafter\ifblank\expandafter{#1}{}{%
		\DTLifdbexists{pathlist}{\DTLcleardb{pathlist}}{\DTLnewdb{pathlist}}% Create new/discard old list
		\setcounter{numberofitemsinlist}{0}%
		\expandafter\parcommalist\expandafter{#1}{\locallists@pathlist}%
		\forlistloop{\stepcounter{numberofitemsinlist}\addtopathlist}{\locallists@pathlist}%
		\DTLsort*{sortlabel}{pathlist}% Sort list
		\begin{center}%
		\begin{tabular}{@{}*{\value{numberofitemsinlist}}{M{\textwidth/\value{numberofitemsinlist}}@{}}}%
			\DTLforeach*{pathlist}{%
										\theName=name,%
										\theIcon=icon%
										}{% Print the path table	
				\includegraphics[height=0.8cm]{Icon_\theIcon.pdf}\par%
				\theName{}%
				\DTLiflastrow{\tabularnewline}{&}%
			}%
		\end{tabular}\end{center}\vspace*{-20pt}% debugging a weird additional space?
	}%
}

%%% Wizard Conclave %%%

% Wizard Conclave entry.

\newcommand{\printwizardconclave}[1]{\begin{innerframenoptsnonegspace}[\wizardconclave]%
#1%
\end{innerframenoptsnonegspace}}

\newcommand{\wizardconclavespellslist}[2]{%
\parselist[,]{#2}{\locallists@optionschoice}%
#1%
\vspace*{-5pt}\begin{itemize}[label={-},noitemsep,topsep=0pt,parsep=0pt]%
\forlistloop{\item}{\locallists@optionschoice}%
\end{itemize}%
}

%%% Mount options %%%

% Mount entry.
\newcommand{\mounts}[1]{\begin{innerframe}[\labels@mountsoptions]\optionsalphaandcostorderlist{#1}\end{innerframe}}
\newcommand{\mountsnopts}[1]{\begin{innerframenopts}[\labels@mountsoptions]\optionsalphaandcostorderlist{#1}\end{innerframenopts}}

\newcommand{\mountstwocols}[1]{\begin{innerframetwocols}[\labels@mountsoptions]%
\setlength{\columnsep}{10pt+0.3cm}%
\setlength{\multicolsep}{0pt}%
\vspace*{-3pt}\begin{multicols}{2}\optionsalphaandcostorderlist{#1}\end{multicols}\end{innerframetwocols}}

%%% Command group %%%

% Command group entry.
\newcommand{\commandgroup}[1]{\begin{innerframe}[\labels@commandgroupoptions]%
\optionsalphaorderlist{#1}%
\end{innerframe}}
\newcommand{\commandgrouptwocols}[1]{\begin{innerframetwocols}[\labels@commandgroupoptions]%
\setlength{\columnsep}{10pt+0.3cm}%
\setlength{\multicolsep}{0pt}%
\vspace*{-3pt}\begin{multicols}{2}\optionsalphaorderlist{#1}\end{multicols}\end{innerframetwocols}}

%%% Model rules %%%

% Rule type
\newcommand{\ruletype}[1]{#1.\newline}

% Model rule entry.
\newcommand{\modelruledef}[2]{%
\DTLnewrow{unitruleslist}%
\DTLnewdbentry{unitruleslist}{name}{#1}%
\DTLnewdbentry{unitruleslist}{ruletext}{#2}%
\pdfstringdef\textwithoutformatting{#1}%
\pdfsubstitute\textwithoutformatting{'}{}%
\pdfsubstitute\textwithoutformatting{ }{}%
\dolanguagespecificsubstitute{}%
\def\DTLentrycommand{\DTLnewdbentry{unitruleslist}{sortlabel}}%
\expandafter\DTLentrycommand\expandafter{\textwithoutformatting}%
}

% Model rules list and frame.
\newcommand{\modelrulesdef}[1]{%
\DTLifdbexists{unitruleslist}{\DTLcleardb{unitruleslist}}{\DTLnewdb{unitruleslist}}% Create new/discard old list
\begin{innerframenopts}[\labels@modelrulesdef]%
#1%
\DTLifdbempty{unitruleslist}{}{%
	\DTLsort*{sortlabel}{unitruleslist}% Sort list
	\begin{description}[leftmargin=0.3cm, labelindent=0cm, labelsep=0.1cm, itemsep=0.15cm, parsep=0cm]%
		\DTLforeach*{unitruleslist}{\ruletext=ruletext,\nameoftherule=name}{% Print back the ordered list
			\item[\nameoftherule{}\spacebeforecolon{}:]\ruletext%
		}%
	\end{description}%
	\DTLcleardb{unitruleslist}%
}
\end{innerframenopts}%
}%

% Optional model rules list and frame.
\newcommand{\optionalmodelrulesdef}[1]{%
\DTLifdbexists{unitruleslist}{\DTLcleardb{unitruleslist}}{\DTLnewdb{unitruleslist}}% Create new/discard old list
\begin{innerframenopts}[\labels@optionalmodelrulesdef]%
#1%
\DTLifdbempty{unitruleslist}{}{%
	\DTLsort*{sortlabel}{unitruleslist}% Sort list
	\begin{description}[leftmargin=0.3cm, labelindent=0cm, labelsep=0.1cm, itemsep=0.15cm, parsep=0cm]%
		\DTLforeach*{unitruleslist}{\ruletext=ruletext,\nameoftherule=name}{% Print back the ordered list
			\item[\nameoftherule{}\spacebeforecolon{}:]\ruletext%
		}%
	\end{description}%
	\DTLcleardb{unitruleslist}%
}
\end{innerframenopts}%
}%

\newlength{\lengthbeforemodelsize}
\setlength{\lengthbeforemodelsize}{6.5cm}
\newlength{\lengthbeforemodelsizetwologos}

\newlength{\lengthoftheptsblock}



%
%%%%%%%%%%%%%%%%%%%%%%%%%%%%%%%%%
%%%% Profile input and layout %%%
%%%%%%%%%%%%%%%%%%%%%%%%%%%%%%%%%

%%% Input parameters %%%
		
\define@key{unit}{name}{\def\unit@name{#1}}
\define@key{unit}{logo}{\def\unit@logo{#1}}
\define@key{unit}{secondlogo}{\def\unit@secondlogo{#1}}
\define@key{unit}{optionallogo}{\def\unit@optionallogo{#1}}

\define@key{unit}{cost}{\def\unit@cost{#1}}
\define@key{unit}{unitsize}{\def\unit@unitsize{#1}}
\define@key{unit}{costpermodel}{\def\unit@costpermodel{#1}}
\define@key{unit}{maxunitsize}{\def\unit@maxunitsize{#1}}
\define@key{unit}{maxunitsperarmy}{\def\unit@maxunitsperarmy{#1}}
\define@key{unit}{maxmodelsperarmy}{\def\unit@maxmodelsperarmy{#1}}
\define@key{unit}{maxmountsperarmy}{\def\unit@maxmountsperarmy{#1}}
\define@key{unit}{addrestriction}{\def\unit@addrestriction{#1}}
\define@key{unit}{type}{\def\unit@type{#1}}
\define@key{unit}{size}{\def\unit@size{#1}}
\define@key{unit}{basesize}{\def\unit@basesize{#1}}
\define@key{unit}{scoring}{\def\unit@scoring{#1}}
\define@key{unit}{categorynote}{\def\unit@categorynote{#1}}

\define@key{unit}{global@Ad}{\def\unit@global@Ad{#1}}
\define@key{unit}{global@Adfly}{\def\unit@global@Adfly{#1}}
\define@key{unit}{global@Ma}{\def\unit@global@Ma{#1}}
\define@key{unit}{global@Mafly}{\def\unit@global@Mafly{#1}}
\define@key{unit}{global@Di}{\def\unit@global@Di{#1}}
\define@key{unit}{global@Rsr}{\def\unit@global@Rsr{#1}}
\define@key{unit}{globalrules}{\def\unit@globalrules{#1}}

\define@key{unit}{defense@HP}{\def\unit@defense@HP{#1}}
\define@key{unit}{defense@Df}{\def\unit@defense@Df{#1}}
\define@key{unit}{defense@Re}{\def\unit@defense@Re{#1}}
\define@key{unit}{defense@Arm}{\def\unit@defense@Arm{#1}}
\define@key{unit}{defenserules}{\def\unit@defenserules{#1}}
\define@key{unit}{defensearmour}{\def\unit@defensearmour{#1}}

\define@key{unit}{offensename}{\def\unit@offensename{#1}}
\define@key{unit}{forceoffensenameprint}{\def\unit@forceoffensenameprint{#1}}
\define@key{unit}{offense@Ag}{\def\unit@offense@Ag{#1}}
\define@key{unit}{offense@At}{\def\unit@offense@At{#1}}
\define@key{unit}{offense@Of}{\def\unit@offense@Of{#1}}
\define@key{unit}{offense@St}{\def\unit@offense@St{#1}}
\define@key{unit}{offense@AP}{\def\unit@offense@AP{#1}}
\define@key{unit}{offenserules}{\def\unit@offenserules{#1}}
\define@key{unit}{offenseweapons}{\def\unit@offenseweapons{#1}}

\define@key{unit}{offensenameB}{\def\unit@offensenameB{#1}}
\define@key{unit}{offenseB@Ag}{\def\unit@offenseB@Ag{#1}}
\define@key{unit}{offenseB@At}{\def\unit@offenseB@At{#1}}
\define@key{unit}{offenseB@Of}{\def\unit@offenseB@Of{#1}}
\define@key{unit}{offenseB@St}{\def\unit@offenseB@St{#1}}
\define@key{unit}{offenseB@AP}{\def\unit@offenseB@AP{#1}}
\define@key{unit}{offenserulesB}{\def\unit@offenserulesB{#1}}
\define@key{unit}{offenseweaponsB}{\def\unit@offenseweaponsB{#1}}

\define@key{unit}{offensenameC}{\def\unit@offensenameC{#1}}
\define@key{unit}{offenseC@Ag}{\def\unit@offenseC@Ag{#1}}
\define@key{unit}{offenseC@At}{\def\unit@offenseC@At{#1}}
\define@key{unit}{offenseC@Of}{\def\unit@offenseC@Of{#1}}
\define@key{unit}{offenseC@St}{\def\unit@offenseC@St{#1}}
\define@key{unit}{offenseC@AP}{\def\unit@offenseC@AP{#1}}
\define@key{unit}{offenserulesC}{\def\unit@offenserulesC{#1}}
\define@key{unit}{offenseweaponsC}{\def\unit@offenseweaponsC{#1}}

\define@key{unit}{offensenameI}{\def\unit@offensenameI{#1}}
\define@key{unit}{offenseI@Ag}{\def\unit@offenseI@Ag{#1}}
\define@key{unit}{offenseI@St}{\def\unit@offenseI@St{#1}}
\define@key{unit}{offenseI@AP}{\def\unit@offenseI@AP{#1}}
\define@key{unit}{offenserulesI}{\def\unit@offenserulesI{#1}}
\define@key{unit}{offenseweaponsI}{\def\unit@offenseweaponsI{#1}}

\define@key{unit}{magic}{\def\unit@magic{#1}}
\define@key{unit}{paths}{\def\unit@paths{#1}}
\define@key{unit}{wizardconclave}{\def\unit@wizardconclave{#1}}
\define@key{unit}{options}{\def\unit@options{#1}}
\define@key{unit}{mounts}{\def\unit@mounts{#1}}
\define@key{unit}{commandgroup}{\def\unit@commandgroup{#1}}
\define@key{unit}{modelrulesdef}{\def\unit@modelrulesdef{#1}}
\define@key{unit}{optionalmodelrulesdef}{\def\unit@optionalmodelrulesdef{#1}}
\define@key{unit}{endtext}{\def\unit@endtext{#1}}
\define@key{unit}{toggles}{\def\unit@toggles{#1}}




%%%%%%%%%%%%%%%%%%%%%%%%%%%%%%%%%%%%%%%%%%%%%%%%%%%%%%%%%%%%%%%%%%%%%%%%%%%
%%% FRAMES DEFINITION
%%%%%%%%%%%%%%%%%%%%%%%%%%%%%%%%%%%%%%%%%%%%%%%%%%%%%%%%%%%%%%%%%%%%%%%%%%%

\newenvironment{innerframe}[1][]{%
\begin{minipage}[t]{\columnwidth}%
%Title
\strut\textcolor{\greycolor}{\rule[2pt]{0.5cm}{0.5pt}}%
\hspace*{1pt}\ChLab{#1}\hspace*{1pt}%
{\leaders\hbox{\textcolor{\greycolor}{\rule[2pt]{1pt}{0.5pt}}}\hfill}%
\hspace*{1pt}\ChLab{\pts{}}\hspace*{1pt}%
\textcolor{\greycolor}{\rule[2pt]{0.1cm}{0.5pt}}\par%
\vspace*{-23pt}\hspace*{0.15cm}\begin{minipage}[t]{\columnwidth - 0.3cm}%
\strut%
}{%
\end{minipage}\end{minipage}\par%
}

%%%%%%%%%%%%%%%%%%%%%%%
% Simple inner frame, no 'pts' label

\newenvironment{innerframenopts}[1][]{%
\begin{minipage}[t]{\columnwidth-3pt}% -3 pt because else there is a 1.5 pt overfull, who knows why
%Title
\strut\textcolor{\greycolor}{\rule[2pt]{0.5cm}{0.5pt}}%
\hspace*{1pt}\ChLab{#1}\hspace*{1pt}%
\textcolor{\greycolor}{\rule[2pt]{\columnwidth -\widthof{\ChLab{#1}} -2pt -0.5cm}{0.5pt}}\par%
\vspace*{-23pt}\hspace*{0.15cm}\begin{minipage}[t]{\columnwidth - 0.3cm}%
\strut%
}{%
\end{minipage}%
\end{minipage}\par%
}

\newenvironment{innerframenoptsnonegspace}[1][]{%
\begin{minipage}[t]{\columnwidth-3pt}% -3 pt because else there is a 1.5 pt overfull, who knows why
%Title
\strut\textcolor{\greycolor}{\rule[2pt]{0.5cm}{0.5pt}}%
\hspace*{1pt}\ChLab{#1}\hspace*{1pt}%
\textcolor{\greycolor}{\rule[2pt]{\columnwidth -\widthof{\ChLab{#1}} -2pt -0.5cm}{0.5pt}}\par%
\vspace*{-4pt}\hspace*{0.15cm}\begin{minipage}[t]{\columnwidth - 0.3cm}%
\strut%
}{%
\end{minipage}%
\end{minipage}\par%
}

\newenvironment{innerframetwocols}[1][]{%
\begin{minipage}[t]{\textwidth}%
%Title
\strut\textcolor{\greycolor}{\rule[2pt]{0.5cm}{0.5pt}}%
\hspace*{1pt}\ChLab{#1}\hspace*{1pt}%
{\leaders\hbox{\textcolor{\greycolor}{\rule[2pt]{1pt}{0.5pt}}}\hfill}%
\hspace*{1pt}\ChLab{\pts{}}\hspace*{1pt}%
\textcolor{\greycolor}{\rule[2pt]{0.1cm}{0.5pt}}%
\hspace*{10pt}%
\textcolor{\greycolor}{\rule[2pt]{0.5cm}{0.5pt}}%
\hspace*{1pt}\ChLab{#1}\hspace*{1pt}%
{\leaders\hbox{\textcolor{\greycolor}{\rule[2pt]{1pt}{0.5pt}}}\hfill}%
\hspace*{1pt}\ChLab{\pts{}}\hspace*{1pt}%
\textcolor{\greycolor}{\rule[2pt]{0.1cm}{0.5pt}}\par%
\vspace*{-14pt}\hspace*{0.15cm}\begin{minipage}[t]{\textwidth - 0.3cm}%
\strut%
}{%
\end{minipage}\end{minipage}\par%
}

% Characteristics table definition

\newcommand{\startcharacteristicstable}{%
\begin{tabular}{@{}p{0.20\textwidth}@{\hskip 0.01\textwidth}P{0.06\textwidth}@{}P{0.06\textwidth}@{}P{0.06\textwidth}@{}P{0.06\textwidth}@{}P{0.06\textwidth}@{\hskip 0.01\textwidth}p{0.48\textwidth}@{}}%
}
\newcommand{\nameindent}{\hspace*{8pt}}
\newcommand{\characteristicsline}{\noalign{\vskip -3.5pt}\hline}% deprecated?

%%% Command to add a new unit definition %%%

\newcommand{\defunit}{
	\setkeys{unit}{%
		name=, logo=, secondlogo=, optionallogo=, cost=, unitsize=, costpermodel=, maxunitsize=, maxunitsperarmy=, maxmodelsperarmy=, maxmountsperarmy=, addrestriction=, type=, size=, basesize=, scoring=, categorynote=, global@Ad=, global@Adfly=, global@Ma=, global@Mafly=, global@Di=, global@Rsr=, globalrules=, defense@HP=, defense@Df=, defense@Re=, defense@Arm=, defenserules=, defensearmour=, offensename=, forceoffensenameprint=, offense@Ag=, offense@At=, offense@Of=,  offense@St=, offense@AP=, offenserules=, offenseweapons=, offensenameB=, offenseB@Ag=, offenseB@At=, offenseB@Of=,  offenseB@St=, offenseB@AP=, offenserulesB=, offenseweaponsB=, offensenameC=, offenseC@Ag=, offenseC@At=, offenseC@Of=,  offenseC@St=, offenseC@AP=, offenserulesC=, offenseweaponsC=, offensenameI=, offenseI@Ag=, offenseI@St=, offenseI@AP=, offenserulesI=, offenseweaponsI=, magic=, paths=, wizardconclave=, options=, mounts=, commandgroup=, modelrulesdef=, optionalmodelrulesdef=, endtext=, toggles=,
	}%
	\setkeys{unit}%
}

%%% Counters and boolean for two columns balance %%%

\newtoggle{printoffensename}

\newcounter{numberofnonrulesdefframes}
\newcounter{numberofrestrictions}
\newcounter{numberoftwocolframes}

\newtoggle{thereismagic}
\newtoggle{thereiswizardconclave}
\newtoggle{thereisoptions}
\newtoggle{thereismounts}
\newtoggle{thereiscommandgroup}
\newtoggle{thereismodelrulesdef}
\newtoggle{thereisoptionalmodelrulesdef}

\newtoggle{twocolprintmagic}
\newtoggle{twocolprintwizardconclave}
\newtoggle{twocolprintoptions}
\newtoggle{twocolprintmounts}
\newtoggle{twocolprintcommandgroup}
\newtoggle{twocolprintmodelrulesdef}
\newtoggle{twocolprintoptionalmodelrulesdef}

\newtoggle{onecolprintmagic}
\newtoggle{onecolprintwizardconclave}
\newtoggle{onecolprintoptions}
\newtoggle{onecolprintmounts}
\newtoggle{onecolprintcommandgroup}
\newtoggle{onecolprintmodelrulesdef}
\newtoggle{onecolprintoptionalmodelrulesdef}

\newtoggle{thereismorethanXitemsinoptions}
\newtoggle{optionsistheonlynonrulesdefframe}

\newtoggle{thereisatleastoneframe}

%%%%%%%%%%%%%%%%%%%%%%%%%%%%%%%%%%%%%%%%%%%%%%%%%%%%%%%%%%%%%%%%%%%%%%%%%%%
%   CHARACTERISTICS
%%%%%%%%%%%%%%%%%%%%%%%%%%%%%%%%%%%%%%%%%%%%%%%%%%%%%%%%%%%%%%%%%%%%%%%%%%%

\newcommand{\ChLab}[1]{{\normalfontsize\greytextcolor{\textit{#1}}}} % Font for Characteristic's labels
\newcommand{\ChVal}[1]{\textbf{#1}} % Font for Characteristic's values

\newcommand{\unitentry}[1]{%
\defunit{#1}%
%
\begin{minipage}[t]{\textwidth}% To avoid splitted profiles between 2 pages
%
%%%%%%%%%%%%%%%%%%%%%%
% Unit title, logos and general info
%
\def\temphypertag{}
\StrGobbleRight{\expandafter\expandafter\expandafter\@gobble\expandafter\string\unit@name}{1}[\temphypertag] % remove first the \ to the left and then the {} to the right - it counts as 1 character and then bug the hyperlink tag. It will remove the last character of the macro name if there is no {} in the end of the name key. So it should always be input as "name=\fancyunitname{},". "name=\fancyunitname," will make hyperlinks crash.
\xdef\unithypertag{\bookprefix\arabic{categorynumber}\temphypertag}% using the unit name macro to define an hypertag
\hypertarget{\unithypertag}{}%
\rule{\textwidth}{0.7pt}\par\vspace{-5pt}%
{%
% Logos
\hspace*{-3pt}\expandafter\ifblank\expandafter{\unit@secondlogo}{%
\begin{minipage}[b]{0.08\textwidth}%
\includegraphics[width=\textwidth]{{logo_\unit@logo}.png}%
\vspace*{-4pt}%
}{%
\begin{minipage}[b]{0.16\textwidth}%
\includegraphics[width=0.5\textwidth]{{logo_\unit@logo}.png}%
\includegraphics[width=0.5\textwidth]{{logo_\unit@secondlogo}.png}%
\vspace*{-4pt}%
}%
\end{minipage}%
%
% Title, unit cost and size
\setlength{\lengthbeforemodelsizetwologos}{\lengthbeforemodelsize - 0.08\textwidth}
\expandafter\ifblank\expandafter{\unit@secondlogo}{%
\hspace{0.01\textwidth}\begin{minipage}[b]{0.5\textwidth+3pt}%
}{%
\hspace{0.01\textwidth}\begin{minipage}[b]{0.42\textwidth+3pt}%
\renewcommand{\perextramodel}{\perextramodelshortened}%
\setlength{\lengthbeforemodelsize}{\lengthbeforemodelsizetwologos}%
}%
% Title with hyperlink, scoring logo
{\Largerfontsize\textbf{\unit@name}}%
\vspace*{3pt}\newline%
% Unit cost and size
\expandafter\ifblank\expandafter{\unit@cost}{}{{\largerfontsize\pts{\textbf{\unit@cost}}} }%
% Is it a single model or a unit with multiple models?
\expandafter\ifblank\expandafter{\unit@unitsize}{}{% if there is no unitsize, we don't print anything at all
	% Now we want to see if there is 1 or more models
	\isitoneornot{\unit@unitsize}{% 1 model
		% Can you add models?
		\expandafter\ifblank\expandafter{\unit@costpermodel}{% no other models
			\hspace*{\lengthbeforemodelsize}%
			\setlength{\lengthoftheptsblock}{\widthof{{\largerfontsize\pts{\textbf{\unit@cost}}} }}%		
			\hspace*{-\lengthoftheptsblock}%
			\singlemodel{}%
		}{% possibility of more models
			+ \pts{\textbf{\unit@costpermodel}}\perextramodel{}%
			%
			\hspace*{\lengthbeforemodelsize}%
			\setlength{\lengthoftheptsblock}{\widthof{{\largerfontsize\pts{\textbf{\unit@cost}}} + \pts{\textbf{\unit@costpermodel}}\perextramodel{}}}%		
			\hspace*{-\lengthoftheptsblock}%
			\textbf{1-\unit@maxunitsize} \Models{}%
		}%
	}{% more than 1 model
		+ \pts{\textbf{\unit@costpermodel}}\perextramodel{}%
		%
		\hspace*{\lengthbeforemodelsize}%
		\setlength{\lengthoftheptsblock}{\widthof{{\largerfontsize\pts{\textbf{\unit@cost}}} + \pts{\textbf{\unit@costpermodel}}\perextramodel{}}}%		
		\hspace*{-\lengthoftheptsblock}%
		\textbf{\unit@unitsize{}-\unit@maxunitsize} \Models{}%
	}%
}%
\vspace{0pt}%
\end{minipage}%
%
% Restrictions
\hfill\begin{minipage}[b]{0.2\textwidth}%
\setlength{\parskip}{0pt}%
\begin{center}%
% Count the number of restrictions
\setcounter{numberofrestrictions}{0}% reset
\expandafter\ifblank\expandafter{\unit@maxunitsperarmy}{}{\stepcounter{numberofrestrictions}}%
\expandafter\ifblank\expandafter{\unit@maxmountsperarmy}{}{\stepcounter{numberofrestrictions}}%
\expandafter\ifblank\expandafter{\unit@maxmodelsperarmy}{}{\stepcounter{numberofrestrictions}}%
\expandafter\ifblank\expandafter{\unit@addrestriction}{}{\stepcounter{numberofrestrictions}}%
\expandafter\ifstrequal\expandafter{\unit@scoring}{yes}{%
	\ifnumcomp{\value{numberofrestrictions}}{>}{1}{%
		\strut\includegraphics[height=8pt]{logo_scoring.png}\par%
	}{%
		\ifnumcomp{\value{numberofrestrictions}}{=}{0}{%
			\strut\vspace*{6pt}\includegraphics[height=12pt]{logo_scoring.png}\par%
		}{%
			\strut\includegraphics[height=12pt]{logo_scoring.png}\par%
		}%
	}%
}{}%
\expandafter\ifblank\expandafter{\unit@maxunitsperarmy}{}{%
\zerotoXunitsperarmy{\unit@maxunitsperarmy}\par%
}%
\expandafter\ifblank\expandafter{\unit@maxmountsperarmy}{}{%
\zerotoXmountsperarmy{\unit@maxmountsperarmy}\par%
}%
\expandafter\ifblank\expandafter{\unit@maxmodelsperarmy}{}{%
\zerotoXmodelsperarmy{\unit@maxmodelsperarmy}\par%
}%
\expandafter\ifblank\expandafter{\unit@addrestriction}{}{%
\unit@addrestriction%
}%
\end{center}%
\vspace{0pt}%
\end{minipage}%
%
% Size, Type and Base
\begin{minipage}[b]{0.2\textwidth}%
\renewcommand{\arraystretch}{1}%
\begin{tabular}{@{}>{\raggedleft}p{0.25\textwidth}@{\hskip 0.03\textwidth}p{0.72\textwidth}@{}}%
\ChLab{\size}&\unit@size{}\tabularnewline%
\ChLab{\type}&\unit@type{}\tabularnewline%
\ChLab{\basesize}&\unit@basesize{} \si{\milli\meter}%
\ifsubstring{\unit@basesize}{x}{}{~\roundbase}\tabularnewline%
\end{tabular}%
\vspace*{-4pt}%
\end{minipage}\par%
}%
\vspace{-10pt}\rule{\textwidth}{0.7pt}\par%
%
% Category note and optional logo
\expandafter\ifblank\expandafter{\unit@optionallogo}{% no logo
\expandafter\ifblank\expandafter{\unit@categorynote}{}{%
\vspace*{-4pt}\textit{\unit@categorynote}\par%
\vspace*{-10pt}\greytextcolor{\rule{\textwidth}{0.5pt}}\par%
}%
}{% with logo
\vspace*{-7pt}\hspace*{-3pt}\begin{tabular}{@{}m{0.06\textwidth}@{}m{0.9\textwidth}@{}}%
\includegraphics[width=0.06\textwidth]{{logo_\unit@optionallogo}.png}&%
\textit{\unit@categorynote}\tabularnewline%
\end{tabular}\par%
\vspace*{-10pt}\greytextcolor{\rule{\textwidth}{0.5pt}}\par%
}%
\vspace*{-5pt}%

%%%%%%%%%%%%%%%%%%%%%%
% Characteristics and rules
%%%%%%%%%%%%%%%%%%%%%%
% Global Characteristics
\renewcommand{\arraystretch}{1.4}%
\vspace*{-5pt}%
\startcharacteristicstable{}%
{\ChLab{\GlobalCharacteristics}}&%
{\ChLab{\AdvanceRateInitials}}&%
{\ChLab{\MarchRateInitials}}& %
{\ChLab{\DisciplineInitials}}&%
\expandafter\ifblank\expandafter{\unit@global@Rsr}{}{{\ChLab{\resurrectedInitials}}}&%
&%
\ChLab{\ModelRules}\tabularnewline%
\expandafter\ifblank\expandafter{\unit@global@Adfly}{}{%
\raggedleft\setlength{\parskip}{0pt}%
\ChLab{\groundmovementlabel}\par%
\ChLab{\flymovementlabel}%
}%
&%
\expandafter\ifblank\expandafter{\unit@global@Ad}{-}{%
\ifsubstring{\unit@global@Ad}{\ascharacter}{\ChVal{{\unit@global@Ad}}}{\ChVal{\distance{\unit@global@Ad}}}%
}%
\expandafter\ifblank\expandafter{\unit@global@Adfly}{}{\setlength{\parskip}{0pt}\par\ChVal{\distance{\unit@global@Adfly}}}%
&%
\expandafter\ifblank\expandafter{\unit@global@Ma}{-}{%
\ifsubstring{\unit@global@Ma}{\ascharacter}{\ChVal{{\unit@global@Ma}}}{\ChVal{\distance{\unit@global@Ma}}}%
}%
\expandafter\ifblank\expandafter{\unit@global@Mafly}{}{\setlength{\parskip}{0pt}\par\ChVal{\distance{\unit@global@Mafly}}}%
&%
\expandafter\ifblank\expandafter{\unit@global@Di}{-}{\ChVal{\unit@global@Di}}&%
\expandafter\ifblank\expandafter{\unit@global@Rsr}{}{\ChVal{\unit@global@Rsr}}&%
&%
\strut\expandafter\ifblank\expandafter{\unit@globalrules}{}{\alphaorderlist{\unit@globalrules}}%
\tabularnewline%
\hline%
\end{tabular}%

%%%%%%%%%%%%%%%%%%%%%%
% Defensive Characteristics
\vspace*{-10pt}%
\startcharacteristicstable{}%
\ChLab{\DefensiveCharacteristics{}}&%
{\ChLab{\HealthPointsInitials}}&%
{\ChLab{\DefensiveSkillInitials}}&%
{\ChLab{\ResilienceInitials}}& %
{\ChLab{\ArmourInitials}}&%
&%
\tabularnewline%
&% no defensive name
\expandafter\ifblank\expandafter{\unit@defense@HP}{-}{\ChVal{\unit@defense@HP}}&%
\expandafter\ifblank\expandafter{\unit@defense@Df}{-}{\ChVal{\unit@defense@Df}}&%
\expandafter\ifblank\expandafter{\unit@defense@Re}{-}{\ChVal{\unit@defense@Re}}&%
\expandafter\ifblank\expandafter{\unit@defense@Arm}{\ChVal{0}}{\ChVal{\unit@defense@Arm}}&%
&%
\strut\expandafter\ifblank\expandafter{\unit@defenserules}{}{\alphaorderlist{\unit@defenserules}}%
\expandafter\ifblank\expandafter{\unit@defensearmour}{}{%
\expandafter\ifblank\expandafter{\unit@defenserules}{}{, }%
\alphaorderlist{\unit@defensearmour}}%
\tabularnewline%
\hline%
\end{tabular}%

%%%%%%%%%%%%%%%%%%%%%%
% Offensive Characteristics
\vspace*{-10pt}%
\startcharacteristicstable{}%
\ChLab{\OffensiveCharacteristics{}}&%
{\ChLab{\AttackValueInitials}}&%
{\ChLab{\OffensiveSkillInitials}}&%
{\ChLab{\StrengthInitials}}& %
{\ChLab{\ArmourPenetrationInitials}}&%
{\ChLab{\AgilityInitials}}&%
\tabularnewline%
\togglefalse{printoffensename}% reset
\expandafter\ifblank\expandafter{\unit@forceoffensenameprint}{}{\toggletrue{printoffensename}}%
\expandafter\ifblank\expandafter{\unit@offensenameB}{}{\toggletrue{printoffensename}}%
\expandafter\ifblank\expandafter{\unit@offensenameI}{}{\toggletrue{printoffensename}}%
\iftoggle{printoffensename}{%
	\nameindent\unit@offensename{}%
}{}%
&%
\expandafter\ifblank\expandafter{\unit@offense@At}{-}{\ChVal{\unit@offense@At}}&%
\expandafter\ifblank\expandafter{\unit@offense@Of}{-}{\ChVal{\unit@offense@Of}}&%
\expandafter\ifblank\expandafter{\unit@offense@St}{-}{\ChVal{\unit@offense@St}}&%
\expandafter\ifblank\expandafter{\unit@offense@AP}{\ChVal{0}}{\ChVal{\unit@offense@AP}}&%
\expandafter\ifblank\expandafter{\unit@offense@Ag}{-}{\ChVal{\unit@offense@Ag}}&%
\strut\expandafter\ifblank\expandafter{\unit@offenserules}{}{\alphaorderlist{\unit@offenserules}}%
\expandafter\ifblank\expandafter{\unit@offenseweapons}{}{%
\expandafter\ifblank\expandafter{\unit@offenserules}{}{, }%
\alphaorderlist{\unit@offenseweapons}}%
\tabularnewline%
%
\expandafter\ifblank\expandafter{\unit@offensenameB}{}{%
\nameindent\unit@offensenameB{}&%
\expandafter\ifblank\expandafter{\unit@offenseB@At}{-}{\ChVal{\unit@offenseB@At}}&%
\expandafter\ifblank\expandafter{\unit@offenseB@Of}{-}{\ChVal{\unit@offenseB@Of}}&%
\expandafter\ifblank\expandafter{\unit@offenseB@St}{-}{\ChVal{\unit@offenseB@St}}&%
\expandafter\ifblank\expandafter{\unit@offenseB@AP}{\ChVal{0}}{\ChVal{\unit@offenseB@AP}}&%
\expandafter\ifblank\expandafter{\unit@offenseB@Ag}{-}{\ChVal{\unit@offenseB@Ag}}&%
\strut\expandafter\ifblank\expandafter{\unit@offenserulesB}{}{\alphaorderlist{\unit@offenserulesB}}%
\expandafter\ifblank\expandafter{\unit@offenseweaponsB}{}{%
\expandafter\ifblank\expandafter{\unit@offenserulesB}{}{, }%
\alphaorderlist{\unit@offenseweaponsB}}%
\tabularnewline%
}%
%
\expandafter\ifblank\expandafter{\unit@offensenameC}{}{%
\nameindent\unit@offensenameC{}&%
\expandafter\ifblank\expandafter{\unit@offenseC@At}{-}{\ChVal{\unit@offenseC@At}}&%
\expandafter\ifblank\expandafter{\unit@offenseC@Of}{-}{\ChVal{\unit@offenseC@Of}}&%
\expandafter\ifblank\expandafter{\unit@offenseC@St}{-}{\ChVal{\unit@offenseC@St}}&%
\expandafter\ifblank\expandafter{\unit@offenseC@AP}{\ChVal{0}}{\ChVal{\unit@offenseC@AP}}&%
\expandafter\ifblank\expandafter{\unit@offenseC@Ag}{-}{\ChVal{\unit@offenseC@Ag}}&%
\strut\expandafter\ifblank\expandafter{\unit@offenserulesC}{}{\alphaorderlist{\unit@offenserulesC}}%
\expandafter\ifblank\expandafter{\unit@offenseweaponsC}{}{%
\expandafter\ifblank\expandafter{\unit@offenserulesC}{}{, }%
\alphaorderlist{\unit@offenseweaponsC}}%
\tabularnewline%
}%
%
\expandafter\ifblank\expandafter{\unit@offensenameI}{}{%
\nameindent\unit@offensenameI{}&%
&%
&%
\expandafter\ifblank\expandafter{\unit@offenseI@St}{}{\ChVal{\unit@offenseI@St}}&%
\expandafter\ifblank\expandafter{\unit@offenseI@AP}{}{\ChVal{\unit@offenseI@AP}}&%
\expandafter\ifblank\expandafter{\unit@offenseI@Ag}{}{\ChVal{\unit@offenseI@Ag}}&%
\strut\expandafter\ifblank\expandafter{\unit@offenserulesI}{}{\alphaorderlist{\unit@offenserulesI}}%
\expandafter\ifblank\expandafter{\unit@offenseweaponsI}{}{%
\expandafter\ifblank\expandafter{\unit@offenserulesI}{}{, }%
\alphaorderlist{\unit@offenseweaponsI}}%
\tabularnewline%
}%
\end{tabular}\newline%
\vspace*{-3pt}\strut% Fixing the space after the characteristics table (else there is a bug when there is multiple offense lines)

%
%%%%%%%%%%%%%%%%%%%%%%
% Assessing the balance of the columns to see if we should put some frames outside of the two column layout
%%%%%%%%%%%%%%%%%%%%%%
% Which frames are defined?
\expandafter\ifblank\expandafter{\unit@options}{}{\toggletrue{thereisoptions}}%
\expandafter\ifblank\expandafter{\unit@magic}{}{\toggletrue{thereismagic}}%
\expandafter\ifblank\expandafter{\unit@paths}{}{\toggletrue{thereismagic}}%
\expandafter\ifblank\expandafter{\unit@wizardconclave}{}{\toggletrue{thereiswizardconclave}}%
\expandafter\ifblank\expandafter{\unit@mounts}{}{\toggletrue{thereismounts}}%
\expandafter\ifblank\expandafter{\unit@commandgroup}{}{\toggletrue{thereiscommandgroup}}%
\expandafter\ifblank\expandafter{\unit@modelrulesdef}{}{\toggletrue{thereismodelrulesdef}}%
\expandafter\ifblank\expandafter{\unit@optionalmodelrulesdef}{}{\toggletrue{thereisoptionalmodelrulesdef}}%
% Are there any non rules-def frames? If there is only one, is it options?
\setcounter{numberofnonrulesdefframes}{0}% reset
\iftoggle{thereismagic}{\stepcounter{numberofnonrulesdefframes}}{}%
\iftoggle{thereiswizardconclave}{\stepcounter{numberofnonrulesdefframes}}{}%
\iftoggle{thereismounts}{\stepcounter{numberofnonrulesdefframes}}{}%
\iftoggle{thereiscommandgroup}{\stepcounter{numberofnonrulesdefframes}}{}%
\iftoggle{thereisoptions}{\stepcounter{numberofnonrulesdefframes}%
	\ifnumcomp{\value{numberofnonrulesdefframes}}{=}{1}{% then there is only options defined
		\toggletrue{optionsistheonlynonrulesdefframe}%
	}{}% End of \ifnumcomp
}{}% End of \iftoggle{thereisoptions}
% Is options long enough to put it out of the twocols layout?
\iftoggle{thereisoptions}{%
	\countitemsinoptionslist{\unit@options}%
	\ifnumcomp{\value{numberofitemsinlist}}{>}{3}{\toggletrue{thereismorethanXitemsinoptions}}{}%
	\ifboolexpr{ togl {optionsistheonlynonrulesdefframe} and togl {thereismorethanXitemsinoptions}}{%
		\toggletrue{onecolprintoptions}
	}{\toggletrue{twocolprintoptions}}% End ifnumcomp
}{}% End of \iftoggle{thereisoptions}
% How do we handle the modelrulesdef now?
\ifboolexpr{ test {\ifnumcomp{\value{numberofnonrulesdefframes}}{=}{0}}
	or
	(test {\ifnumcomp{\value{numberofnonrulesdefframes}}{=}{1}} and togl {onecolprintoptions})
}{% Then we want rules on a one col layout
	\iftoggle{thereismodelrulesdef}{\toggletrue{onecolprintmodelrulesdef}}{}%
	\iftoggle{thereisoptionalmodelrulesdef}{\toggletrue{onecolprintoptionalmodelrulesdef}}{}%
}{% Else we want them in twocols layout
	\iftoggle{thereismodelrulesdef}{\toggletrue{twocolprintmodelrulesdef}}{}%
	\iftoggle{thereisoptionalmodelrulesdef}{\toggletrue{twocolprintoptionalmodelrulesdef}}{}%
}% End of \ifboolexpr
% How do we handle the last frames?
\setcounter{numberoftwocolframes}{0}%
\iftoggle{thereismagic}{\stepcounter{numberoftwocolframes}\toggletrue{twocolprintmagic}}{}%
\iftoggle{thereiswizardconclave}{\stepcounter{numberoftwocolframes}\toggletrue{twocolprintwizardconclave}}{}%
\iftoggle{thereismounts}{\stepcounter{numberoftwocolframes}\toggletrue{twocolprintmounts}}{}%
\iftoggle{twocolprintoptions}{\stepcounter{numberoftwocolframes}}{}%
\iftoggle{thereiscommandgroup}{%
	\ifnumcomp{\value{numberoftwocolframes}}{>}{0}{%
		\toggletrue{twocolprintcommandgroup}%
	}{%
		\countitemsinoptionslist{\unit@commandgroup}%
		\ifnumcomp{\value{numberofitemsinlist}}{>}{1}{%
				\toggletrue{onecolprintcommandgroup}%
		}{%
				\toggletrue{twocolprintcommandgroup}%		
		}%
	}%
}{}%
% Is there at least one thing to put in the big rules frame?
\ifboolexpr{ test {\ifnumcomp{\value{numberofnonrulesdefframes}}{>}{0}}
	or togl {thereismodelrulesdef}
	or togl {thereisoptionalmodelrulesdef}
}{\toggletrue{thereisatleastoneframe}}{}%

%%%%%%%%%%%%%%%%%%%%%%
% If any toggles have to be manually changed
\unit@toggles{}%
%%%%%%%%%%%%%%%%%%%%%%
% Rules frame
% We print the frame if and only if there is something to put inside
\iftoggle{thereisatleastoneframe}{%
%%%%%%%%%%%%%%%%%%%%%%
% Two columns frames, we print it if there is something to put in the frame
\ifboolexpr{togl{twocolprintmodelrulesdef} or togl{twocolprintoptions} or togl{twocolprintmagic} or togl{twocolprintwizardconclave} or togl{twocolprintmounts} or togl{twocolprintcommandgroup} or togl{twocolprintoptionalmodelrulesdef}}{%
\setlength{\columnsep}{10pt}%
\setlength{\multicolsep}{0pt}%
\raggedcolumns%
\begin{multicols}{2}%
\iftoggle{twocolprintmodelrulesdef}{\strut\modelrulesdef{\unit@modelrulesdef}}{}%
\iftoggle{twocolprintmagic}{\strut\magic{\unit@magic}{\unit@paths}}{}%
\iftoggle{twocolprintoptions}{\strut\options{\unit@options}}{}%
\iftoggle{twocolprintwizardconclave}{\strut\printwizardconclave{\unit@wizardconclave}}{}%
\iftoggle{twocolprintmounts}{\strut\mounts{\unit@mounts}}{}%
\iftoggle{twocolprintcommandgroup}{\strut\commandgroup{\unit@commandgroup}}{}%
\iftoggle{twocolprintoptionalmodelrulesdef}{\strut\optionalmodelrulesdef{\unit@optionalmodelrulesdef}}{}%
\end{multicols}%
% Additional space if there is a single column rule after
\ifboolexpr{ togl {onecolprintmodelrulesdef}
	or togl {onecolprintoptions}
	or togl {onecolprintmounts}
	or togl {onecolprintcommandgroup}
	or togl {onecolprintoptionalmodelrulesdef}
}{\vspace*{8pt}}{}%
}{}%
%%%%%%%%%%%%%%%%%%%%%%
% Single column
\ifboolexpr{ togl {onecolprintmodelrulesdef}
	or togl {onecolprintoptions}
	or togl {onecolprintmounts}
	or togl {onecolprintcommandgroup}
	or togl {onecolprintoptionalmodelrulesdef}
}{%
\vspace*{-5pt}%
\iftoggle{onecolprintmodelrulesdef}{\strut\modelrulesdef{\unit@modelrulesdef}}{}%
\iftoggle{onecolprintoptions}{\strut\optionstwocols{\unit@options}}{}%
\iftoggle{onecolprintmounts}{\strut\mountstwocols{\unit@mounts}}{}%
\iftoggle{onecolprintcommandgroup}{\strut\commandgrouptwocols{\unit@commandgroup}}{}%
\iftoggle{onecolprintoptionalmodelrulesdef}{\strut\optionalmodelrulesdef{\unit@optionalmodelrulesdef}}{}%
}{}%
}{}% END of there is at least one frame

%%%%%%%%%%%%%%%%%%%%%%
% Additional stuff
\expandafter\ifblank\expandafter{\unit@endtext}{}{\unit@endtext\vspace*{-7pt}}%

\vspace*{-3pt}%
\hfill{\verysmallfontsize\textcolor{white}{d}} % Else footer goes wild for some reason with ocgcolorlinks option
\end{minipage}%
\vspace{5pt}%


%%%%%%%%%%%%%%%%%%%%%%%%%%%%%%%%%%%%%%%%%%%%%%%%%%%%
%%%%%%% Filling the profiles database for the QRS
\DTLnewrow{profiles}%
\expandafter\dtbfillunitname\expandafter{\unit@name}%
\expandafter\dtbfillhypertag\expandafter{\unithypertag}%
\edef\categorynumberstring{\arabic{categorynumber}}%
\expandafter\dtbfillcategorytag\expandafter{\categorynumberstring}%
\expandafter\dtbfillsize\expandafter{\unit@size}%
\expandafter\dtbfilltype\expandafter{\unit@type}%
%
\expandafter\dtbfillglobal@Ad\expandafter{\unit@global@Ad}%
\expandafter\dtbfillglobal@Ma\expandafter{\unit@global@Ma}%
\expandafter\dtbfillglobal@Di\expandafter{\unit@global@Di}%
%
\expandafter\ifblank\expandafter{\unit@global@Rsr}{}{%
	\expandafter\dtbfillglobal@Rsr\expandafter{\unit@global@Rsr}%
	\pdfstringdef\textwithoutformatting{\unit@global@Rsr}%
	\pdfsubstitute\textwithoutformatting{ }{}%
	\dolanguagespecificsubstitute{}%
	\substitute\textwithoutformatting{\string\376}{}%
	\substitute\textwithoutformatting{\string\377}{}%
	\substitute\textwithoutformatting{\string\000}{}%
	\expandafter\dtbfillglobal@RsrSortLabel\expandafter{\textwithoutformatting}%
}%
%
\expandafter\dtbfilldefense@HP\expandafter{\unit@defense@HP}%
\expandafter\dtbfilldefense@Df\expandafter{\unit@defense@Df}%
\expandafter\dtbfilldefense@Re\expandafter{\unit@defense@Re}%
\expandafter\dtbfilldefense@Arm\expandafter{\unit@defense@Arm}%
%
\expandafter\dtbfilloffensename\expandafter{\unit@offensename}%
\expandafter\dtbfilloffense@Ag\expandafter{\unit@offense@Ag}%
\expandafter\dtbfilloffense@At\expandafter{\unit@offense@At}%
\expandafter\dtbfilloffense@Of\expandafter{\unit@offense@Of}%
\expandafter\dtbfilloffense@St\expandafter{\unit@offense@St}%
\expandafter\dtbfilloffense@AP\expandafter{\unit@offense@AP}%
%
\expandafter\dtbfilloffensenameB\expandafter{\unit@offensenameB}%
\expandafter\dtbfilloffenseB@Ag\expandafter{\unit@offenseB@Ag}%
\expandafter\dtbfilloffenseB@At\expandafter{\unit@offenseB@At}%
\expandafter\dtbfilloffenseB@Of\expandafter{\unit@offenseB@Of}%
\expandafter\dtbfilloffenseB@St\expandafter{\unit@offenseB@St}%
\expandafter\dtbfilloffenseB@AP\expandafter{\unit@offenseB@AP}%
%
\expandafter\dtbfilloffensenameC\expandafter{\unit@offensenameC}%
\expandafter\dtbfilloffenseC@Ag\expandafter{\unit@offenseC@Ag}%
\expandafter\dtbfilloffenseC@At\expandafter{\unit@offenseC@At}%
\expandafter\dtbfilloffenseC@Of\expandafter{\unit@offenseC@Of}%
\expandafter\dtbfilloffenseC@St\expandafter{\unit@offenseC@St}%
\expandafter\dtbfilloffenseC@AP\expandafter{\unit@offenseC@AP}%
%
\expandafter\dtbfilloffensenameI\expandafter{\unit@offensenameI}%
\expandafter\dtbfilloffenseI@Ag\expandafter{\unit@offenseI@Ag}%
\expandafter\dtbfilloffenseI@St\expandafter{\unit@offenseI@St}%
\expandafter\dtbfilloffenseI@AP\expandafter{\unit@offenseI@AP}%
% End of Unit Entry
}% END newcommand unitentry

%%%%%%%%%%%%%%%%%%%%%%%%%%%%%%%%%%%%%%%%%%%%%%%%%%%%
%%%%%%%%%%%%%%%%%%%%%%%%%%%%%%%%%%%%%%%%%%%%%%%%%%%%
%%%%%%%% QRS
%%%%%%%%%%%%%%%%%%%%%%%%%%%%%%%%%%%%%%%%%%%%%%%%%%%%

\newcommand{\startartillerytable}{%
\begin{center}
\alternaterowcolors\normalfontsize
\noindent\begin{tabular}{@{}p{3.7cm}M{2.1cm}@{}M{1.7cm}@{}M{1.5cm}@{}M{1cm}@{}M{1cm}@{}M{1.2cm}M{3.5cm}@{}}
\textbf{\ATname} & \textbf{\ATartillery} & \textbf{\ATtohit} & \textbf{\ATrange} & \textbf{\St{}} & \textbf{\AP{}} & \textbf{\ATshots} & \textbf{\ATrules} \tabularnewline
}

\newcommand{\closeartillerytable}{%
\end{tabular}
\end{center}
}

\newcommand{\artillerytablenote}[1]{\vspace*{-5pt}{\normalfontsize#1}}

\newcommand{\startshootingtable}{%
\begin{center}
\alternaterowcolors\normalfontsize
\begin{tabular}{lcl}
\textbf{\ATname} & \textbf{\ATtohit} & \textbf{\STshootingmodel} \tabularnewline
}

\newcommand{\closeshootingtable}{%
\end{tabular}
\end{center}
}