
%%% Gaming Scenarios

\newcommand{\startgamingscenarioheadings}{%
	\begin{description}[leftmargin=0.3cm, labelindent=0cm, labelsep=0.1cm, itemsep=0.15cm, parsep=0cm]%
}

\newcommand{\gamingscenarioheading}[1]{\item[\textbf{#1 --}]}

\newcommand{\closegamingscenarioheadings}{%
	\end{description}%
}

\newcommand{\startgamingscenariolist}{%
	\vspace*{10pt}
	\begin{center}
	\begin{tabular}{c c l c}
	\hline
}

\newcommand{\closegamingscenariolist}{%
	\hline
	\end{tabular}
	\end{center}
}

\newcommand{\newgamingscenario}[3]{\section*{#1 -- #2. #3}\phantomsection}

\newcommand{\printmap}[1]{%
\vspace*{10pt}
	\begin{center}
	\def\svgwidth{0.8\textwidth}
	\input{#1}
	\end{center}
\vspace*{10pt}
}

\newcommand{\printmapportals}[1]{%
\vspace*{10pt}
	\begin{center}
	\def\svgwidth{0.85\textwidth}
	\input{#1}
	\end{center}
\vspace*{10pt}
}

\newcommand{\startswiftstrikingtable}{%
	\vspace*{10pt}
	\begin{tabular}{r l}
	\hline
}

\newcommand{\closeswiftstrikingtable}{%
	\end{tabular}
}


\newcommand{\startmultipletaskstable}[3]{%
	\renewcommand{\arraystretch}{1.69}
	\vspace*{10pt}
	\begin{tabular}{P{0.1\textwidth} P{0.1\textwidth} p{0.72\textwidth}}
	\hline
	\textbf{#1}&
	\textbf{#2}&
	\textbf{#3}\tabularnewline
}

\newcommand{\closemultipletaskstable}{%
	\hline	
	\end{tabular}
}

\newcommand{\newtaskfortable}[4]{%
	#1&%
	#2&%
	\textbf{#3}\vspace*{3pt}\newline%
	#4%
	\tabularnewline%
}


%%% Narrative Scenarios

\newcommand{\titleruletype}[1]{\textnormal{\largefontsize{} -- #1}}

\newcommand{\startnarrativescenariolist}{%
	\vspace*{10pt}
	\begin{center}
	\begin{tabular}{c l c}
	\hline
}

\newcommand{\closenarrativescenariolist}{%
	\hline
	\end{tabular}
	\end{center}
}

\newcommand{\newnarrativescenario}[2]{\section*{#1. #2}\phantomsection}

\newcommand{\marchtovictorytable}[4]{%
	\vspace*{10pt}
	\begingroup
	\setlength{\tabcolsep}{3pt}
	\renewcommand{\arraystretch}{1.4}
	\begin{tabular}{M{0.165\textwidth}M{0.098\textwidth}M{0.11\textwidth}M{0.11\textwidth}M{0.11\textwidth}M{0.11\textwidth}M{0.11\textwidth}M{0.08\textwidth}}
	\hline
	\textbf{#1}&
	\rewordedrule{\SI{0}{\percent}--\SI{10}{\percent}}&
	\rewordedrule{>\SI{10}{\percent}--\SI{20}{\percent}}&
	\rewordedrule{>\SI{20}{\percent}--\SI{25}{\percent}}&
	\rewordedrule{>\SI{25}{\percent}--\SI{30}{\percent}}&
	\rewordedrule{>\SI{30}{\percent}--\SI{35}{\percent}}&
	\rewordedrule{>\SI{35}{\percent}--\SI{40}{\percent}}&
	>\rewordedrule{\SI{40}{\percent}}\\
	\textbf{(#2)}&
	\rewordedrule{0--450}&
	\rewordedrule{451--900}&
	\rewordedrule{901--1125}&
	\rewordedrule{1126--1350}&
	\rewordedrule{1351--1575}&
	\rewordedrule{1576--1800}&
	>\rewordedrule{1800}\\
	\textbf{#3}&
	\rewordedrule{\minuss{}3}&
	\rewordedrule{\minuss{}2}&
	\rewordedrule{\minuss{}1}&
	\rewordedrule{0}&
	\rewordedrule{+1}&
	\rewordedrule{+2}&
	\rewordedrule{+3}\\
	\textbf{#4}&
	\rewordedrule{+3}&
	\rewordedrule{+2}&
	\rewordedrule{+1}&
	\rewordedrule{0}&
	\rewordedrule{\minuss{}1}&
	\rewordedrule{\minuss{}2}&
	\rewordedrule{\minuss{}3}\\
	\hline
	\end{tabular}
	\endgroup
}

\newcommand{\startweatherthestormtable}{%
	\vspace*{10pt}
	\begin{center}\begin{tabular}{>{\bfseries}M{0.02\textwidth}>{\bfseries}M{0.15\textwidth}m{0.73\textwidth}}
	\hline
}

\newcommand{\closeweatherthestormtable}{%
	\hline
	\end{tabular}\end{center}
}

\newcommand{\weatherthestormtableentry}[3]{%
	#1 & #2 & #3\\
}
