
\newgamingscenario{1}{3}{Dark Visions}
\label{DarkVisions}

\flufffont{An evil presence haunts these lands. Its will at times manifests itself to you, compelling you to lead its next victim to feed its insatiable hunger. Please the spirit and you may go safely.}

\subsection*{Deployment}

Standard Deployment Type: Encircle.

\subsection*{Pre-Game Set-up}

Place a small Impassable Terrain Feature (approximately \distance{3\timess{}3}) suitable to represent the source of The Gaze in the centre of the board, or, if there already is another Terrain Feature there, as close as possible to the centre of the board, but at least \distance{1} away from other Terrain Features. In case you don't have any suitable Impassable Terrain Feature available, you may mark the centre of the board as the source of The Gaze with a marker instead.

\newcommand{\attacker}{Attacker}
\newcommand{\defender}{Defender}
\printmap{pics/deployment_1_3_dark_visions.pdf_tex}

\subsection*{Special Scenario Rules}

At the start of each Player Turn, roll a D6 and consult the table below to determine which enemy unit is targeted by The Gaze.

\startswiftstrikingtable
\textbf{D6 Result}&
\textbf{Target for The Gaze}\\
1--2& 
The enemy Scoring Unit closest to the centre of the source of The Gaze\\
3--4& 
The enemy War Machine or Gigantic model closest to the centre of the source of The Gaze\\
5--6& 
The enemy unit (of any kind) closest to the centre of the source of The Gaze\\
\hline
\closeswiftstrikingtable

If a result is not applicable, roll the D6 again until it is. The Gaze remains on a unit until the end of the game. If two or more potential targets of The Gaze are equally close to the centre of the source of The Gaze, randomise which of these units is targeted.

\subsection*{Winning the Secondary Objective}

The player who has the lowest number of units under The Gaze removed as casualties at the end of the game wins this Secondary Objective.


