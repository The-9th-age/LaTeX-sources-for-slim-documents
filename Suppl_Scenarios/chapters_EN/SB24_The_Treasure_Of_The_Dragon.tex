
\newcommand{\greatreddragon}{Great Red Dragon}
\newcommand{\dragonaegis}{\aegis{} (4+, against Ranged Attacks)}
\newcommand{\proud}{\textbf{Proud}}
\newcommand{\prouddef}{Each Health Point lost by the Dragon counts twice when calculating the Combat Score.}
\newcommand{\territorial}{\textbf{Territorial}}
\newcommand{\territorialdef}{If a Scoring unit comes within \distance{6} of the Dragon or the Treasure Marker during any move other than a Charge Move and the Dragon isn't Engaged in Combat, immediately stop that unit. Pivot the Dragon in the direction of the Centre of the unit, and then move the Dragon into base contact with the unit, as if it was Charging using a Flying Movement. The Dragon counts as Charging and this move follows all the rules for Charge Moves (i.e. the target has to declare a Charge Reaction, the rules for Maximising Contact apply, etc.). The When Charged, the Dragon's Charge Reaction is always Hold, and its Pursuit and Overrun distance is always \distance{0}.}
\newcommand{\toughskin}{\textbf{Tough Skin}}
\newcommand{\toughskindef}{Attacks against the Dragon with Multiple Wounds lose this Attack Attribute.}
\newcommand{\savage}{\textbf{Savage}}
\newcommand{\savagedef}{The target of the Dragon's Special Attacks, as well as each of its Close Combat Attacks, must be randomly determined between all possible targets available. Its Close Combat Attacks are always allocated towards the \rnf{} models within the targeted unit, if possible (remember that the Dragon can benefit from Swirling Melee). The Dragon can only use its Breath Attack as a Melee Attack, and it does so the first time it fights a second Round of Combat.}

\newnarrativescenario{2}{The Treasure of the Dragon}
\label{TheTreasureoftheDragon}

\flufffont{No creature loves gold as dragons do.}

\subsection*{Deployment}

Standard Deployment Type: Marching Columns, with the exception that both players must choose the Dragon's short Board Edge (see below) and that no unit, not even units using Special Deployment, can be deployed in the third of the board that contains the Dragon.

\printmap{pics/deployment_2_the_treasure_of_the_dragon.pdf_tex}

\subsection*{Pre-Game Set-up}

We recommend you use armies of 4500 Army Points or less in this scenario.

Build one side of the Battlefield according to the figure above. Create a half circlewith Impassable Terrain on one of the short Board Edges, and place a Hill in the middle of this half circle. Place D3+1 random Terrain Features and a Forest on the rest of the board. Randomly distribute these Terrain Features in the four quarters A--D and move them \distance{2D6} in a random direction. 

Place a Dragon model where the red rectangle is and a Treasure Marker on the centre of the Hill.

\subsection*{Scenario Special Rules}

The Dragon is a Great Red Dragon, whose profile you'll find below. It is treated as an enemy model by both players. Rounds of Combat with the Dragon are only fought in the Player Turns of the player whose units are Engaged in Combat with the Dragon, except when units of both players are Engaged in the same Combat involving the Dragon. When calculating the Combat Score for such combats, the side with the highest Combat Score is considered to be the only winner, while all other sides count as losing the combat.


\unitentry{%
	name=\greatreddragon{},
	logo=monster,
	cost=,
	unitsize=1,
	type=\beast{},
	size=\sizegigantic{},
	basesize=50\timess{}100,
	global@Ad=0,
	global@Ma=0,
	global@Di=9,
	globalrules={\textbf{\proud{}},\textbf{\territorial{}},\unbreakable{},\fly{0}{0},\lighttroops{}},
	defense@HP=6,
	defense@Df=6,
	defense@Re=6,
	defense@Arm=4,
	defenserules={\dragonaegis{}, \hardtarget{} (1), \textbf{\toughskin{}}},
	offensename=\greatreddragon{},
	offense@At=5,
	offense@Of=6,
	offense@St=6,
	offense@AP=3,
	offense@Ag=3,
	offenserules={\breathattack{\St{} 4, \AP{} 1, \flamingattacks{}}, \textbf{\savage{}}},
	modelrulesdef={%
		\modelruledef{\proud}{\ruletype{\universalrule}\prouddef}
		\modelruledef{\territorial}{\ruletype{\universalrule}\territorialdef}
		\modelruledef{\toughskin}{\ruletype{\personalprotection}\toughskindef}
		\modelruledef{\savage}{\ruletype{\attackattribute}\savagedef}
	},
} % END UNIT ENTRY 


\subsection*{Winning the Scenario}

At the start of each Player Turn, each player gains a Victory Counter for each of their own Scoring Units that are within \distance{3} of the Treasure Marker and not Engaged in Combat.

You gain a Victory Counter for each unsaved wound inflicted by your units on the Dragon.

At the end of the game, the player with the most Victory Counters wins. 
