
\newcommand{\greatreddragon}{Great Red Dragon}
\newcommand{\dragonaegis}{\aegis{} (4+\rewordedrule{,} against \rewordedrule{R}anged \rewordedrule{A}ttacks)}
\newcommand{\proud}{\textbf{Proud}}
\newcommand{\prouddef}{Each \rewordedrule{Health Point lost by} the Dragon counts twice \rewordedrule{when calculating the Combat Score}.}
\newcommand{\territorial}{\textbf{Territorial}}
\newcommand{\territorialdef}{If a Scoring unit comes within \distance{6} of the Dragon or the Treasure Marker \rewordedrule{during any move other than a Charge Move} and the Dragon isn't Engaged in Combat, immediately stop that unit. \rewordedrule{Pivot the Dragon in the direction of the Centre of the unit}, and then move the \rewordedrule{D}ragon into \rewordedrule{base} contact with the unit\rewordedrule{,} as if it was \rewordedrule{C}harging using \rewordedrule{a} Flying Movement. \rewordedrule{The Dragon counts as Charging and this move follows all the rules for Charge Moves (i.e. the target has to declare a Charge Reaction, the rules for Maximising Contact apply, etc.).} The \rewordedrule{When Charged, the Dragon's Charge Reaction is always Hold, and its Pursuit and Overrun distance is always \distance{0}.}}
\newcommand{\toughskin}{\textbf{Tough Skin}}
\newcommand{\toughskindef}{Attacks against the Dragon with Multiple Wounds lose this Attack Attribute.}
\newcommand{\savage}{\textbf{Savage}}
\newcommand{\savagedef}{The target of the Dragon's Special Attacks, as well as each of its Close Combat Attack\rewordedrule{s}, must be randomly \rewordedrule{determined} between all \rewordedrule{possible targets available}. Its Close Combat Attacks are always allocated towards the \rnf{} models within the targeted unit\rewordedrule{,} if possible (remember that the Dragon \rewordedrule{can benefit from} Swirling Melee). The Dragon \rewordedrule{can only} use its Breath Attack as a Melee Attack\rewordedrule{, and it does so} the first \rewordedrule{time it fights a} second \rewordedrule{Round of C}ombat.}

\newnarrativescenario{2}{The Treasure of the Dragon}
\label{TheTreasureoftheDragon}

\flufffont{No creature loves gold as dragons do.}

\subsection*{Deployment}

\rewordedrule{Standard Deployment Type:} Marching Columns, with the exception that both players must cho\rewordedrule{o}se the Dragon's short \rewordedrule{Board Edge (see below)} and that no unit\rewordedrule{,} not even units \rewordedrule{using} \rewordedrule{S}pecial \rewordedrule{D}eployment\rewordedrule{,} can be deployed in the third of the \rewordedrule{board} that contains the Dragon.

\printmap{pics/deployment_2_the_treasure_of_the_dragon.pdf_tex}

\subsection*{Pre-Game Set-up}

\rewordedrule{We recommend you use armies of 4500 Army Points or less in this scenario.}

Build one side of the Battlefield \rewordedrule{according to the figure above}. \rewordedrule{C}reate a half circle\rewordedrule{with Impassable Terrain on one of the short Board Edges,} and place a Hill in the middle of th\rewordedrule{is half} circle. \rewordedrule{P}lace D3+1 random Terrain Feature\rewordedrule{s} and a Forest \rewordedrule{on the rest of the board}. \rewordedrule{Randomly distribute} these Terrain Features in the four quarters A--D and move them \distance{2D6} in a random direction. 

Place a Dragon model where the red rectangle is and a Treasure Marker on the cent\rewordedrule{re} of the \rewordedrule{H}ill.

\subsection*{Scenario Special Rules}

The Dragon is a Great Red Dragon\rewordedrule{, whose} profile \rewordedrule{you'll find} below. It is treated as an enemy model by both players. \rewordedrule{Rounds of} Combat with the Dragon \rewordedrule{are} only fought in the Player Turns of the player \rewordedrule{whose units are Engaged} in Combat with the Dragon\rewordedrule{, except when units of} both players \rewordedrule{are E}ngaged \rewordedrule{in the same Combat involving} the Dragon. When \rewordedrule{calculating the Combat Score} for such combats, the side with the highest \rewordedrule{Combat S}core is \rewordedrule{considered to be} the \rewordedrule{only} winner, \rewordedrule{while} all other sides count as losing the combat.

\input{layout/greatreddragon.tex}

\subsection*{Winning the Scenario}

At the start of each Player Turn, \rewordedrule{each player} gain\rewordedrule{s} a Victory Counter for each of \rewordedrule{their own} Scoring Units that are within \distance{3} of the Treasure \rewordedrule{Marker} and not Engaged in Combat.

\rewordedrule{You gain a Victory Counter f}or each unsaved wound inflicted by your units on the Dragon.

At the end of the game, the player with the most Victory Counters wins. 
