
\newnarrativescenario{7}{March to Victory}\label{MarchtoVictory}

\flufffont{With scheduled reinforcements inbound for both sides, claiming strategically sound positions in the local area becomes all the more important to make sure a decisive victory can be had at a later point.}

\subsection*{Deployment}

Standard Deployment Type: Frontline Clash, with the exception that units may not be deployed within \distance{12} of the short Board Edges.

\printmap{pics/deployment_7_march_to_victory.pdf_tex}

\subsection*{Pre-Game Set-up}

One player is the Attacker and the other player is the Defender. This is randomised at the beginning of step 6 of the Pre-Game Sequence (before determining Deployment Zones). The Attacker gets the first turn.

\subsection*{Winning the Scenario}

The Attacker receives bonus Battle Points for having friendly units inside the Defender's Deployment Zone at the end of the game, while the Defender receives points for preventing that. Determine the Battle Points as usual, add up the combined Point Costs of the Attacker's non-Fleeing units inside the Defender's Deployment Zone, and use the table below to calculate the amount of bonus Battle Points scored. Units at or below \SI{25}{\percent} of their starting Health Points (excluding Characters joined to the unit) count towards these bonus points with only half their Point Costs, rounding fractions up.

\marchtovictorytable{%
	Percentage of Army Points in Zone}{%
	if playing with 4500 pts}{%
	Attacker}{%
	Defender}
