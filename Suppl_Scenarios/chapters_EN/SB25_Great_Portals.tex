
\newnarrativescenario{3}{Great Portals of the Barren Mountains}
\label{GreatPortalsoftheBarrenMountains}

\flufffont{High up in the Barren Mountains, there exist two portals, one on each side of a narrow valley. No one knows who built them or their original purpose, though their value is obvious. Beware though, for the balance of the portal\rewordedrule{s'} energy must be maintained.}

\subsection*{Deployment}

The Deployment Zones are \rewordedrule{referred as} \rewordedrule{R}ed and \rewordedrule{B}lue. Not\rewordedrule{e} their placement relative to the \rewordedrule{P}ortals \rewordedrule{in the figure below}. 

\newcommand{\reddeploymentzone}{Red Deployment Zone}
\newcommand{\bluedeploymentzone}{Blue Deployment Zone}

\printmapportals{pics/deployment_3_the_portals_of_the_barren_mountains.pdf_tex}

\subsection*{Pre-Game Set-up}

Place \rewordedrule{the} Portal\rewordedrule{s} in two \rewordedrule{diagonally} opposite corners of the \rewordedrule{board}. \rewordedrule{T}he bases should be triangular in shape with sides of \rewordedrule{approximately \distance{18\timess{}8}}, assuming a \distance{72\rewordedrule{\timess{}48}} \rewordedrule{board}.

\subsection*{Scenario Special Rules}

There are two kinds of Ionisation Tokens, Red and Blue. Units that receive tokens are \rewordedrule{referred to} as Ioni\rewordedrule{s}ed Units and gain \textbf{Electric Charge} (see below). A unit can only have a single Ionisation Token.

\rewordedrule{During step 7 of the Deployment Phase Sequence (after moving Vanguarding units)}, the players take turn\rewordedrule{s marking 3 of their units each with} Ionisation Tokens of the same colour as their Deployment Zone, starting with the player who finished deploying first.

\subsubsection*{Electric Charge\titleruletype{\attackattribute}}

When \rewordedrule{C}harging, Ioni\rewordedrule{s}ed Units have special interactions:
\begin{itemize}%					
	\item Charge Range rolls against units of opposite Ionisation are subject to Maximised Roll. If such a Charge is successful, the \rewordedrule{owner of} the \rewordedrule{C}harging unit can choose to immediately switch the Ionisation \rewordedrule{Tokens between} the \rewordedrule{C}harging unit and the \rewordedrule{Charged} unit (\rewordedrule{once the Charge Move is completed}).
	\item Charge Range rolls against units of equal Ionisation are subject to Minimised Roll.
\end{itemize}
	
\subsubsection*{Portal Jumping}

When a unit ends an Advance Move fully within \distance{5} of a \rewordedrule{P}ortal, \rewordedrule{it} may choose to \rewordedrule{either:}
\begin{itemize}
	\item \rewordedrule{Enter it, unless is has an Ionisation of the same colour as the \rewordedrule{P}ortal}
	\item \rewordedrule{Lose its Ionisation, if it is Ionized}
\end{itemize}

If the unit enters the \rewordedrule{P}ortal, it is removed from the \rewordedrule{Battlefield} and then placed back in the same formation, \rewordedrule{facing any direction, following the Unit Spacing rule, and} fully within \distance{5} of the other \rewordedrule{P}ortal\rewordedrule{.} \rewordedrule{I}f this is impossible, the unit may not enter the \rewordedrule{P}ortal. \rewordedrule{If t}he teleported unit \rewordedrule{was not Ionised, it} gains an Ionisation Token \rewordedrule{of the colour of} the \rewordedrule{P}ortal it exited (\rewordedrule{e.g. exiting} through \rewordedrule{a R}ed \rewordedrule{P}ortal gives a Red Ionisation).

\subsection*{Winning the Scenario}

At the end of each \rewordedrule{P}layer \rewordedrule{T}urn\rewordedrule{, starting on} Game Turn 3, the \rewordedrule{A}ctive \rewordedrule{P}layer may choose to discard Ionisation Tokens from units within \distance{6} of the centre of the \rewordedrule{board}. Discarded tokens are placed in a discard pool at the side of the \rewordedrule{board}. For each \rewordedrule{discarded} Ionisation Token, the player gains a Victory Counter. If the discard pool (which is charged by \rewordedrule{both} players) contains an equal number of Red and Blue Ionisation Tokens after this is done, the player gains an additional Victory Counter.

At the end of the game, the player with the most Victory Counter\rewordedrule{s} wins the game.
