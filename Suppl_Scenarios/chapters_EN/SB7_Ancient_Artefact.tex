
\newgamingscenario{1}{5}{Ancient Art\rewordedrule{e}fact}
\label{AncientArtefact}

\flufffont{Your army was drawn here by the rumours of a valuable art\rewordedrule{e}fact of a forgotten age, but so was the army of your foe. Show them who's boss!}

\subsection*{Deployment}

\rewordedrule{Standard Deployment Type:} Refused Flank\rewordedrule{.}

\subsection*{Pre-Game Set-up}

\rewordedrule{Place a small Impassable Terrain Feature (approximately \distance{3\timess{}3}) suitable to represent the Ancient Artefact in the centre of the board, or, if there already is another Terrain Feature there, as close as possible to the centre of the board, but at least \distance{1} away from other Terrain Features. In case you don't have any suitable Impassable Terrain Feature available, you may mark the centre of the board as the Ancient Artefact with a marker instead.}

\printmap{pics/deployment_1_5_ancient_artifact.pdf_tex}

\subsection*{Scenario Special Rules}

The game lasts for a random \rewordedrule{number of turns}. At the end of the \nth{4} Game Turn\rewordedrule{,} roll a D6. On \rewordedrule{a roll of} 2+, play a \nth{5} Game Turn. At the end of the 5th Game Turn roll a D6. On \rewordedrule{a roll of} 3+, play a \nth{6} Game Turn, etc.

At the end of each Game Turn after the first, the player with the most Scoring Units within \distance{6} of the Ancient Art\rewordedrule{e}fact gains a counter.

The General and the Battle Standard Bearer \rewordedrule{of} both \rewordedrule{armies} gain \textbf{Scoring}.

\subsection*{Winning the Secondary Objective}

At the end of the game\rewordedrule{,} the player with the most counters wins th\rewordedrule{is} Secondary Objective.


