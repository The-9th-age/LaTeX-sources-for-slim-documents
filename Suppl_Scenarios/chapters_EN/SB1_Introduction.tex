\section*{Introduction}
\label{introduction}

While you can find several scenarios in the main \rewordedrule{R}ulebook for \nameofthegame{} \rewordedrule{in the form of the standard Deployment Types and Secondary Objectives,} we have received requests for additional scenarios.

In this \rewordedrule{Supplement}, you will find a number of\rewordedrule{scenarios} that are less focused on balanced tournament play \rewordedrule{but} have a stronger focus on narrative play.

This \rewordedrule{Supplement consists} of three different sections, for three different types of scenarios\rewordedrule{: Gaming Scenarios, Narrative Scenarios, and Add-Ons}.

The first \rewordedrule{section features} a list of \rewordedrule{simple} Gaming Scenarios \rewordedrule{that are} intended to be used as an alternative to the default scenarios of \nameofthegame{}. Usually, they only change the \rewordedrule{rules for D}eployment and the \rewordedrule{S}econdary \rewordedrule{O}bjective and thus can be used in everyday gaming, without \rewordedrule{any} special preparation\rewordedrule{s}.

The second part consists of Narrative Scenarios\rewordedrule{, which} completely change \rewordedrule{Deployment Types and victory} conditions, have special rules, and \rewordedrule{may} even change \rewordedrule{the rules for} army building. \rewordedrule{Therefore y}ou need to agree on such a scenario with your opponent before building your \rewordedrule{army}. \rewordedrule{Narrative Scenarios} offer more narrative aspects, and can easily be included in a campaign. \rewordedrule{Note that t}he \rewordedrule{victory} conditions are not necessarily symmetric\rewordedrule{al}, and the balance is less guaranteed than in Gaming Scenario\rewordedrule{s}.

\rewordedrule{And f}inally, you can find \rewordedrule{several} Add-Ons \rewordedrule{at the end of this document that comprise} small changes of the rules, \rewordedrule{so that} you can use them in any standard game or scenario to bring \rewordedrule{even more} flavour and fun to your game.

\rewordedrule{You will find figures illustrating the deployment rules for each scenario in the corresponding section. Note that the scenarios are intended to be played on a standard board (\distance{72} wide and \distance{48} deep). If you wish to play on a smaller or larger board, you will need to adjust the length specifications in the figures accordingly.}

\section*{Table of Contents}

% TOC
%%%

\newcommand{\tocfirstcolumn}{%
\tocentry{SEArmySpecificRules}{\armyspecificrules}\par
\tocentry{SEModelRules}{\armymodelrules}\par
\tocentry{SEHereditarySpell}{\hereditaryspell}\par
\tocentry{SEKindreds}{\kindreds}\par
\tocentry{SEAspectsOfNature}{\aspectsofnature}\par
\tocentry{SESpecialEquipment}{\specialequipment}\par
\tocentry{SEArmyOrganisation}{\armyorganisation}\par
\tocentry{SEQRS}{\quickrefsheet}\par
}

\newcommand{\tocsecondcolumn}{%
\tocentry{SECharacters}{\characters}\par
\tocentry{SEMounts}{\charactermounts}\par
\tocentry{SECore}{\core}\par
\tocentry{SESpecial}{\specialcategory}\par
\tocentry{SEUnseenArrows}{\unseenarrows}\par
}

