\section*{Introduction}
\label{introduction}

While you can find several scenarios in the main Rulebook for \nameofthegame{} in the form of the standard Deployment Types and Secondary Objectives, we have received requests for additional scenarios.

In this Supplement, you will find a number ofscenarios that are less focused on balanced tournament play but have a stronger focus on narrative play.

This Supplement consists of three different sections, for three different types of scenarios: Gaming Scenarios, Narrative Scenarios, and Add-Ons.

The first section features a list of simple Gaming Scenarios that are intended to be used as an alternative to the default scenarios of \nameofthegame{}. Usually, they only change the rules for Deployment and the Secondary Objective and thus can be used in everyday gaming, without any special preparations.

The second part consists of Narrative Scenarios, which completely change Deployment Types and victory conditions, have special rules, and may even change the rules for army building. Therefore you need to agree on such a scenario with your opponent before building your army. Narrative Scenarios offer more narrative aspects, and can easily be included in a campaign. Note that the victory conditions are not necessarily symmetrical, and the balance is less guaranteed than in Gaming Scenarios.

And finally, you can find several Add-Ons at the end of this document that comprise small changes of the rules, so that you can use them in any standard game or scenario to bring even more flavour and fun to your game.

You will find figures illustrating the deployment rules for each scenario in the corresponding section. Note that the scenarios are intended to be played on a standard board (\distance{72} wide and \distance{48} deep). If you wish to play on a smaller or larger board, you will need to adjust the length specifications in the figures accordingly.

\section*{Table of Contents}

% TOC
%%%

\newcommand{\tocfirstcolumn}{%
\tocentry{SEArmySpecificRules}{\armyspecificrules}\par
\tocentry{SEModelRules}{\armymodelrules}\par
\tocentry{SEHereditarySpell}{\hereditaryspell}\par
\tocentry{SEKindreds}{\kindreds}\par
\tocentry{SEAspectsOfNature}{\aspectsofnature}\par
\tocentry{SESpecialEquipment}{\specialequipment}\par
\tocentry{SEArmyOrganisation}{\armyorganisation}\par
\tocentry{SEQRS}{\quickrefsheet}\par
}

\newcommand{\tocsecondcolumn}{%
\tocentry{SECharacters}{\characters}\par
\tocentry{SEMounts}{\charactermounts}\par
\tocentry{SECore}{\core}\par
\tocentry{SESpecial}{\specialcategory}\par
\tocentry{SEUnseenArrows}{\unseenarrows}\par
}

