
\newgamingscenario{1}{2}{Hidden Treasures}
\label{HiddenTreasures}

\flufffont{You didn't have time to decipher the treasure map fully before setting off. Take what\rewordedrule{ever} opportunit\rewordedrule{y} you can as more secrets are revealed.}

\subsection*{Deployment}

\rewordedrule{Standard Deployment Type:} Frontline Clash\rewordedrule{.}

\subsection*{Pre-Game Set-up}

\rewordedrule{M}ark the positions \rewordedrule{1, 2, and 3 on the board as} shown \rewordedrule{i}n the \rewordedrule{figure} below.

\printmap{pics/deployment_1_2_hidden_treasures.pdf_tex}

\subsection*{Special Scenario Rules}

At the start of each Game Turn, roll a D3\rewordedrule{.} \rewordedrule{Place a Treasure Marker with its centre on the Centre Line, at the position corresponding to the result of the dice. If there is already one marker at the position, don't place any markers during this Game Turn.}

\subsection*{Winning the Secondary Objective}

At the end of the game, the player who controls the most Treasure Markers wins th\rewordedrule{is} Secondary Objective. A Treasure Marker is controlled by the player with the most Scoring Units within \distance{6} of \rewordedrule{the centre of the marker}. If a unit is within \distance{6} of \rewordedrule{the centres of} more than one \rewordedrule{marker}, it only counts as within \distance{6} of \rewordedrule{the centre of} the \rewordedrule{marker} which is closest to its \rewordedrule{C}entre (randomi\rewordedrule{s}e if both markers\rewordedrule{' centres} are equally close).
